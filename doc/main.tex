\documentclass[twocolumn]{article}
\usepackage{microtype}
\usepackage{alphabeta}
\usepackage[greek,english]{babel}
\languageattribute{greek}{polutoniko}
\usepackage[dvipsnames,table]{xcolor}
\colorlet{tm}{Black}
\colorlet{ty}{Black}

\usepackage{fullpage}
\usepackage{amsmath, amscd, amsthm, amssymb, mathrsfs,amsfonts}
\usepackage{textcomp}
\usepackage{txfonts}
\usepackage{graphicx}

%%% TERM CALCULUS
\newcommand{\link}[2]{#1{\leftrightarrow}#2}
\newcommand{\wait}[1]{#1().}
\newcommand{\halt}[1]{#1[].0}
\newcommand{\send}[2]{#1[#2].}
\newcommand{\recv}[2]{#1(#2).}
\newcommand{\case}[3]{%
\toks0={#2#3}%
\edef\param{\the\toks0}%
\ifx\param\empty
  \ensuremath{\text{case} \; #1 \; \{\}}
\else
  \ensuremath{\text{case} \; #1 \; \{ #2 ; #3 \}}
\fi}
\newcommand{\inl}[1]{#1[\text{inl}].}
\newcommand{\inr}[1]{#1[\text{inr}].}
\newcommand{\expn}[2]{#1 \uparrow #2.}
\newcommand{\intl}[2]{#1 \downarrow #2.}
\newcommand{\cut}[1]{\nu #1.}

\def\parr{\ensuremath{\mathbin{\rotatebox[origin=c]{180}{\&}}}}
\def\with{\ensuremath{\mathbin{\text{\&}}}}
\def\plus{\ensuremath{\oplus}}
\def\tens{\ensuremath{\otimes}}
\def\limp{\ensuremath{\multimap}}
\def\one{\ensuremath{\mathbf{1}}}
\def\nil{\ensuremath{\mathbf{0}}}
\def\lequiv{\ensuremath{\multimapboth}}

\DeclareGraphicsRule{.ai}{pdf}{.ai}{}
\newcommand{\emoji}[2][1em]{\ensuremath{\vcenter{\hbox{\includegraphics[width=#1]{twemoji/assets/#2.ai}}}}}
\newcommand{\mary}[1][1em]{\emoji[#1]{1f469}}
\newcommand{\john}[1][1em]{\emoji[#1]{1f466}}
\newcommand{\cake}[1][1em]{\emoji[#1]{1f382}}
\newcommand{\plato}[1][1em]{\emoji[#1]{1f381}}
\newcommand{\nocake}[1][1em]{\emoji[#1]{1f64c}}
\newcommand{\money}[1][1em]{\emoji[#1]{1f4b0}}
\newcommand{\dollar}[1][1em]{\emoji[#1]{1f4b5}}
\newcommand{\ptis}[1][1em]{\emoji[#1]{1f3ea}}
\newcommand{\good}[1][1em]{\emoji[#1]{2714}}
\newcommand{\bad}[1][1em]{\emoji[#1]{2716}}

\newcommand{\tm}[2][]{%
\toks0={#1}%
\edef\param{\the\toks0}%
\ifx\param\empty
  \ensuremath{{\color{ty}#2}}%
\else
  \ensuremath{{\color{tm}#1}\colon{\color{ty}#2}}%
\fi}
\newcommand{\seq}[2][]{\ensuremath{{\color{tm}#1}\vdash{\color{ty}#2}}}
\newcommand{\subst}[3]{\ensuremath{#1 \{ #2 / #3 \}}}
\newcommand{\nod}[2][]{\ensuremath{{\star}_{#1}{#2}}}
\newcommand{\give}[2][]{\ensuremath{{ ? }_{#1}{#2}}}
\newcommand{\take}[2][]{\ensuremath{{ ! }_{#1}{#2}}}

\usepackage{bussproofs}
\def\fCenter{\ensuremath{\;\vdash\;}}
\newcommand{\NOM}[1]{\RightLabel{\textsc{#1}}}
\newcommand{\SYM}[1]{\RightLabel{\ensuremath{#1}}}
\EnableBpAbbreviations
\newenvironment{proofbox}[1][0.9]%
  {\gdef\scalefactor{#1} \leavevmode\hbox\bgroup}
  {\scalebox{\scalefactor}{\DisplayProof} \egroup}
\newenvironment{proofblock}[1][0.9]%
  {\gdef\scalefactor{#1}\begin{center}\proofSkipAmount \leavevmode}%
  {\scalebox{\scalefactor}{\DisplayProof}\proofSkipAmount \end{center} }

%%% Local Variables:
%%% TeX-master: "main"
%%% End:
\usepackage{lipsum}

\title{Give or take \\
  {\large Non-Determinism and Session Types}}
\author{Pepijn Kokke \\
  CDT Pervasive Parallelism \\
  \url{pepijn.kokke@ed.ac.uk}}

\begin{document}
\maketitle

\begin{abstract}
  The aim of this project is to extend the session-typed \textpi-calculus CP
  \citep{wadler2012} with \emph{global} non-determinism---i.e.\ without using an
  explicit program construct for non-determinism---which preserves the strong
  guarantees of termination that are offered by the correspondence with
  classical linear logic. 
   
  Our extended version of CP will allow us to model more complex concurrent
  processes, such as an online ticket vendor which can sell some finite number
  of tickets, with customers racing to obtain a ticket.
  %
  We will be able to model further complications, such as a ticket vendor for
  which all tickets are distinct---e.g.\ by being labeled with distinct serial
  numbers---and vendors who maintain a list detailing which customer bought
  which ticket.
  %
  All of these are open problems for session-typed systems which guarantee
  termination.
\end{abstract}

\section{Introduction}

\section{Background}

\subsection{Session Types and Linear Logic}
\citet{caires2010}
\citet{wadler2012}

\subsection{Session Types and Non-Determinism}
\citet{caires2014,atkey2016,caires2017}

\section{Methodology}
Classical Processes~\citep[CP;][]{wadler2012} is a process calculus with a
Curry-Howard correspondence to classical linear logic, and for which computation
is identified with cut elimination. One consequence of this is that computation
is entirely deterministic.

In this section, we will discuss which changes we will make to CP to be able to
model non-deterministic computation. In addition, we will sketch process
semantics for these constructs.

\subsection{Give and take}
Our aim is to add some construct for global non-determinism to CP.
In the \textpi-calculus, the non-determinism comes from two things: channel
sharing and the structural congruence relation.
This allows you to write a process such as e.g.\ %
\[
  (x[u].x[v].P \mid x(y).Q \mid x(z).R)
\]
This process is non-deterministic, because it is not determined a priori which
value sent along $x$ will be received by which process. 

In CP, however, channels are strictly binary---all communication is introduced
by the \textsc{Cut}-rule, which creates a channel with two endpoints, and hands
each endpoint to \emph{one} process.
\begin{proofblock}
  \AXC{$\seq[P]{ \Gamma , \tm[x]{A} }$}
  \AXC{$\seq[Q]{ \Delta , \tm[x]{A^\bot} }$}
  \NOM{Cut}
  \BIC{$\seq[{\nu x.(P \mid Q)}]{ \Gamma , \Delta }$}
\end{proofblock}
Furthermore, CP defines its reduction semantics on \emph{typed} processes, and
does away with structural congruence.
As a consequence, computation in CP is completely deterministic.

Our aim, therefore, is to capture safe name sharing.
We do this by extending CP with a pair of exponentials---$\give[n]{A}$ (give)
and $\take[n]{A}$ (take).
We refer to the resulting system as \gtcp.
The type $\give[n]{A}$ represents a process which communicates as $A$, $n$
times, and these communications can depend on one another.
For instance, a cake store could use the income from its sales to buy the
ingredients for new cakes.
The type $\take[n]{A}$ represents a \emph{pool} of $n$ processes, each of which
communicates as $A$, and each of these communications should be independent.
It is this \emph{pooling} which allows for name sharing, and therefore
non-determinism.
\begin{center}
  \begin{proofbox}
    \AXC{$\seq[{P}]{ \Gamma , \tm[y]{A} }$}
    \SYM{\give{}}
    \UIC{$\seq[{?x[y].P}]{ \Gamma , \tm[x]{\give[1]{A}} }$}
  \end{proofbox}
  \begin{proofbox}
    \AXC{$\seq[P]{ \Gamma , \tm[y]{A} }$}
    \SYM{\take{}}
    \UIC{$\seq[{!x(y).P}]{ \Gamma , \tm[x]{\take[1]{A}} }$}
  \end{proofbox}
  \\[1\baselineskip]
  \begin{proofbox}
    \AXC{$\seq[P]{ \Gamma , \tm[x]{\give[m]{A}} , \tm[x']{\give[n]{A}} }$}
    \NOM{Cont}
    \UIC{$\seq[\subst{P}{x}{x'}]{ \Gamma , \tm[x]{\give[m+n]{A}} }$}
  \end{proofbox}
  \\[1\baselineskip]
  \begin{proofbox}
    \AXC{$\seq[P]{ \Gamma , \tm[x]{\take[m]{A}} }$}
    \AXC{$\seq[Q]{ \Delta , \tm[x]{\take[n]{A}} }$}
    \NOM{Pool}
    \BIC{$\seq[P \mid Q]{ \Gamma , \Delta , \tm[x]{\take[m+n]{A}} }$}
  \end{proofbox}
\end{center}
We eliminate the cut on a channel of type $\give[n]{A^\bot} / \take[n]{A}$
by linking each communication of the process $\give[n]{A^\bot}$ to an arbitrary
process from the pool of processes $\take[n]{A}$. 
Thus, we can rewrite the above cut into $n$ smaller cuts.
In addition, if the endsequent does not contain any channels of type
$\give[n]{A}$ or $\take[n]{A}$, then we should be able to eliminate \emph{all}
usage of the rules for non-determinism. This isn't surprising---after running a
non-deterministic program, we should end up in \emph{some} deterministic end
state. 

\subsection{Relation with bounded linear logic}
The connectives $\give[n]{A}$ and $\take[n]{A}$ were first used in the context
of bounded linear logic~\citep[BLL;][]{girard1992}, a type system based on
intuitionistic linear logic which guarantees polynomial time complexity.

Here we use the connectives in a classical setting.
However, this is not the only difference.
The connectives in BLL do not have a pooling rule---instead, they have a
\emph{storage} rule, which allows for arbitrary---but bounded---duplication.
\begin{center}
  \begin{proofbox}
    \AXC{$\seq{ \give[m]{\Gamma} , \take[m]{A} }$}
    \NOM{Store}
    \UIC{$\seq{ \give[mn]{\Gamma} , \take[mn]{A} }$}
  \end{proofbox}
\end{center}
We can show that, for $n \geq 1$, pooling is a generalization of
storage.\footnote{%
  BLL also has a weakening rule, which introduces a channel of type
  $\give[0]{A}$. We have, so far, not found a use for weakening or the $0$-case.
}
However, this does require making $n$ copies of the proof.
By induction.
\begin{proofblock}
  \AXC{IH}
  \noLine
  \UIC{$\vphantom{!}\smash[t]{\vdots}$}
  \noLine
  \UIC{$\seq{ \give[mn]{\Gamma} , \take[mn]{A} }$}
  \AXC{ARG}
  \noLine
  \UIC{$\vphantom{!}\smash[t]{\vdots}$}
  \noLine
  \UIC{$\seq{ \give[m]{\Gamma} , \take[m]{A} }$}
  \NOM{Pool}
  \BIC{$\seq{ \give[mn]{\Gamma} , \give[m]{\Gamma} , \take[m(n+1)]{A} }$}
  \NOM{Cont}
  \UIC{$\seq{ \give[m(n+1)]{\Gamma} , \take[m(n+1)]{A} }$}
\end{proofblock}
In our system, the processes pooled under $\take[n]{A}$ can be distinct.
Therefore, we most likely lose the time complexity guarantees offered by BLL.

\subsection{A cake store and its customer}
We can use the new connectives to implement non-deterministic programs.
For instance, in \autoref{fig:cake-store}, we model a cake store which sells
exactly one cake, yet can have two customers race for the cake.

The first process represents a customer. Customers want a channel over which
they either can buy a cake---trading a coin for a cake---or which tells them
when cake is unavailable. This is represented using the type
$(\coin\mult\cake^\bot)\with\bot$.
The second process is the cake store itself. It offers up a channel of type
$\give[2]{(\coin^\bot\parr\cake)\plus{1}}$---a channel over which two customers
can try to purchase a cake simultaneously.

In order to run the program, we pool two customers together, and have them
communicate with the store.
\begin{figure*}
  \begin{mdframed}
    \vspace*{1em}
    \begin{proofblock}  
      \AXC{$\vphantom{!}\smash[t]{\vdots}$}
      \noLine
      \UIC{$\seq[P]{ \Gamma , \tm[y]{\coin} }$}
      \AXC{$\vphantom{!}\smash[t]{\vdots}$}
      \noLine
      \UIC{$\seq[Q]{ \Delta , \tm[z]{\cake^\bot} }$}
      \BIC{$\seq[{y[z].(P \mid Q)}]{
          \Gamma , \Delta , \tm[y]{\coin\mult\cake^\bot} }$}
      \AXC{$\vphantom{!}\smash[t]{\vdots}$}
      \noLine
      \UIC{$\seq[R]{ \Gamma , \Delta }$}
      \SYM{\bot}
      \UIC{$\seq[{y().R}]{ \Gamma , \Delta , \tm[y]{\bot} }$}
      \SYM{\with}
      \BIC{$\seq[{\case{y}{y[z].(P \mid Q)}{y().R}}]{
          \Gamma , \Delta , \tm[y]{(\coin\mult\cake^\bot) \with \bot} }$}
      \SYM{\take{}}
      \UIC{$\seq[{!x(y).\case{y}{y[z].(P \mid Q)}{y().R}}]{
          \Gamma , \Delta ,
          \tm[x]{\take[1]{((\coin\mult\cake^\bot) \with \bot)}} }$}
    \end{proofblock}
    \vspace*{0pt}
    \begin{proofblock} 
      \AXC{}
      \SYM{1}
      \UIC{$\seq[{y[].0}]{ \tm[y]{1} }$}
      \SYM{\plus_2}
      \UIC{$\seq[{\inr{y}{y[].0}}]{
          \tm[y]{(\coin^\bot\parr\cake)\plus{1}} }$}
      \SYM{\give{}}
      \UIC{$\seq[{?x'[y].\inr{y}{y[].0}}]{
          \tm[x']{\give[1]{((\coin^\bot\parr\cake)\plus{1})}} }$}
      \noLine
      \UIC{$\vphantom{!}\smash[t]{\vdots}$}
      \noLine
      \UIC{$\seq[{\mathcal{E}[{\inr{x'}{x'[].0}}]}]{
          \Theta ,
          \tm[z]{\coin^\bot\parr\cake} ,
          \tm[x']{\give[1]{((\coin^\bot\parr\cake)\plus{1})}} }$}
      \SYM{\plus_1}
      \UIC{$\seq[\inl{z}{\mathcal{E}[{?x'(y).\inr{y}{y[].0}}]}]{
          \Theta ,
          \tm[z]{(\coin^\bot\parr\cake)\plus{1}} ,
          \tm[x']{\give[1]{((\coin^\bot\parr\cake)\plus{1})}} }$}
      \SYM{\give{}}
      \UIC{$\seq[{?x[z].\inl{z}{\mathcal{E}[{?x'[y].\inr{y}{y[].0}}]}}]{
          \Theta ,
          \tm[x]{\give[1]{((\coin^\bot\parr\cake)\plus{1})}} ,
          \tm[x']{\give[1]{((\coin^\bot\parr\cake)\plus{1})}} }$}
      \NOM{Cont}
      \UIC{$\seq[{?x[z].\inl{z}{\mathcal{E}[{?x[y].\inr{y}{y[].0}}]}}]{
          \Theta ,
          \tm[x]{\give[2]{((\coin^\bot\parr\cake)\plus{1})}} }$}
    \end{proofblock}
    \vspace*{0pt}
  \end{mdframed}
  \caption{A cake store and its customer.}\label{fig:cake-store}
\end{figure*}

\subsection{Take tensors, give pars}
The bounded modalities give ($\give[n]{A}$) and take ($\take[n]{A}$) have a
strong connection to par ($\parr$) and tensor ($\mult$).
In some sense, they are vectors of pars and tensors.
We can demonstrate this by showing that we can ``unroll'' a give into a series
of pars.
\begin{proofblock}
  \AXC{}
  \NOM{Ax}
  \UIC{$\seq{ A^\bot , A }$}
  \SYM{\take{}}
  \UIC{$\seq{ \take[1]{A^\bot} , A }$}
  \AXC{}
  \NOM{Ax}
  \UIC{$\seq{ \take[n]{A^\bot} , \give[n]{A} }$}
  \NOM{Pool}
  \BIC{$\seq{ \take[n+1]{A^\bot} , A , \give[n]{A} }$}
  \SYM{\parr}
  \UIC{$\seq{ \take[n+1]{A^\bot} , A \parr \give[n]{A} }$}
  \SYM{\parr}
  \UIC{$\seq{ \give[n+1]{A^\bot} \limp A \parr \give[n]{A} }$}
\end{proofblock}
And vice versa---we can roll a series of pars into a give.
\begin{proofblock}
  \AXC{}
  \NOM{Ax}
  \UIC{$\seq{ A^\bot , A }$}
  \SYM{\give{}}
  \UIC{$\seq{ A^\bot , \give[1]{A} }$}
  \AXC{}
  \NOM{Ax}
  \UIC{$\seq{ \take[n]{A^\bot} , \give[n]{A} }$}
  \SYM{\mult}
  \BIC{$\seq{ A^\bot \mult \take[n]{A^\bot} , \give[1]{A} , \give[n]{A} }$}
  \NOM{Cont}
  \UIC{$\seq{ A^\bot \mult \take[n]{A^\bot} , \give[n+1]{A} }$}
  \SYM{\parr}
  \UIC{$\seq{ A \parr \give[n]{A} \limp \give[n+1]{A} }$}
\end{proofblock}
Thus, we have
\[
  \give[n+1]{A} \lequiv A \parr \give[n]{A}
\]
Similarly, for take and tensor we have
\[
  \take[n+1]{A} \lequiv A \mult \give[n]{A}
\]
And---both given the semantics, and the rules of the type system---it should
come as no surprise that we have
\begin{proofblock}
  \AXC{}
  \RightLabel{Ax}
  \UIC{$\seq{A^\bot, A}$}
  \RightLabel{\take{}(2)}
  \UIC{$\seq{\take[1]{A^\bot}, \take[1]{A}}$}
  \RightLabel{\parr}
  \UIC{$\seq{\give[1]{A} \limp \take[1]{A}}$}
  \AXC{}
  \RightLabel{Ax}
  \UIC{$\seq{A^\bot, A}$}
  \RightLabel{\give{}(2)}
  \UIC{$\seq{\give[1]{A^\bot}, \give[1]{A}}$}
  \RightLabel{\parr}
  \UIC{$\seq{\take[1]{A} \limp \give[1]{A}}$}
  \RightLabel{\mult}
  \BIC{$\seq{\give[1]{A} \lequiv \take[1]{A}}$}
\end{proofblock}
For unit vectors, it should not matter whether we build them using
tensors or pars.

\subsection{Conflating $\give[n]{A}$ and $\take[n]{A}$}
Give and take impose a strict structure on the kinds of non-deterministic which
we can model. Specifically, we always need one server process which offers to
communicate $n$ times, and $n$ independent client processes
There is a good reason for this. After cut elimination, we are left with a
single, central process---the server process.
However, this may not always be what we desire.
Imagine a matchmaking service.
This could be a dating site, an online market place---any process whose purpose
it is to link \emph{two} pools of independent processes.
In such a scenario, it is entirely reasonable to write programs which end up as
several independent processes. For instance,
\[
  (x[u].P \mid x[v].Q \mid x(y).R \mid x(z).S)
\]
In the above program, the names in $P$, $Q$, $R$ and $S$ are disjoint, aside
from $x$. This means that whichever way we choose to communicate along $x$, we
will end up with two completely independent processes.
Logically, allowing such processes corresponds to adding the \textsc{Mix}-rule.
\begin{proofblock}
  \AXC{$\seq[P]{ \Gamma }$}
  \AXC{$\seq[Q]{ \Delta }$}
  \NOM{Mix}
  \BIC{$\seq[P \mid Q]{ \Gamma , \Delta }$}
\end{proofblock}
It is a well-known consequence of \textsc{Mix} that we get proofs of $\bot \limp
1$ and $A \mult B \limp A \parr B$---a phenomenon which was recently
investigated in the context of CP by~\citet{atkey2016}.
\textsc{Mix} also gives us $\take[n]{A}\limp\give[n]{A}$, by induction on $n$.
\begin{proofblock}
  \AXC{}
  \NOM{Ax}
  \UIC{$\seq{ A^\bot , A }$}
  \SYM{\give{}(2)}
  \UIC{$\seq{ \give[1]{A^\bot} , \give[1]{A} }$}
  \AXC{IH}
  \noLine
  \UIC{$\vphantom{()}{\smash[t]{\vdots}}$}
  \noLine
  \UIC{$\seq{ \give[n]{A^\bot} , \give[n]{A} }$}
  \NOM{Mix}
  \BIC{$\seq{
      \give[1]{A^\bot} , \give[1]{A} , \give[n]{A^\bot} , \give[n]{A} }$}
  \NOM{Cont(2)}
  \UIC{$\seq{ \give[n+1]{A^\bot} , \give[n+1]{A} }$}
\end{proofblock}


% Why do we want multicut? Because it's interesting, but also because it allows
% us to do some new things. Namely, it allows us to construct the following
% process:
%
% (x[u].P | x[v].Q)
%

\begin{proofblock}
  \AXC{$\seq{ \Gamma , A_1^\bot , \ldots , A_n^\bot }$}
  \AXC{$\seq{ \Delta , A_1     , \ldots , A_n     }$}
  \NOM{MultiCut}
  \BIC{$\seq{ \Gamma , \Delta }$}
\end{proofblock}
\begin{figure*}
  \begin{mdframed}
    \begin{proofblock}
      \AXC{}
      \NOM{Ax}
      \UIC{$\seq{ A^\bot , A }$}
      \SYM{\take{}}
      \UIC{$\seq{ \take[1]{A^\bot} , A }$}
      \AXC{$\ldots$}
      \AXC{}
      \NOM{Ax}
      \UIC{$\seq{ A^\bot , A }$}
      \SYM{\take{}}
      \UIC{$\seq{ \take[1]{A^\bot} , A }$}
      \NOM{Pool}
      \TIC{$\seq{ \take[n]{A^\bot} , A , \ldots , A }$}
      \AXC{}
      \NOM{Ax}
      \UIC{$\seq{ A^\bot , A }$}
      \SYM{\take{}}
      \UIC{$\seq{ A^\bot , \take[1]{A} }$}
      \AXC{$\ldots$}
      \AXC{}
      \NOM{Ax}
      \UIC{$\seq{ A^\bot , A }$}
      \SYM{\take{}}
      \UIC{$\seq{ A^\bot , \take[1]{A} }$}
      \NOM{Pool}
      \TIC{$\seq{ A^\bot , \ldots , A^\bot , \take[n]{A} }$}
      \NOM{MultiCut}
      \BIC{$\seq{ \take[n]{A^\bot} , \take[n]{A} }$}
      \SYM{\parr}
      \UIC{$\seq{ \give[n]{A} \limp \take[n]{A} }$}
    \end{proofblock}
  \end{mdframed}
  \caption{Conflation of $\give[n]{A}$ and $\take[n]{A}$ in the presence of
    \textsc{MultiCut}.}
  \label{fig:conflate-give-and-take}
\end{figure*}
\begin{center}
  \begin{proofbox}
    \AXC{$\seq[P]{ \Gamma , \tm[y]{A} }$}
    \SYM{\nod{}}
    \UIC{$\seq[x\nod{y}.P]{ \Gamma , \tm[x]{\nod[1]{A}} }$}
  \end{proofbox}
  \begin{proofbox}
    \AXC{$\seq[P]{ \Gamma , \tm[y]{\nod[m]{A}} , \tm[z]{\nod[n]{A}} }$}
    \NOM{Cont}
    \UIC{$\seq[\subst{P}{x}{y,z}]{ \Gamma , \tm[x]{\nod[m+n]{A}} }$}
  \end{proofbox}
\end{center}
\begin{proofblock}
  \AXC{$\seq{ \Gamma , \nod[m]{A} }$}
  \AXC{$\seq{ \Delta , \nod[n]{A} }$}
  \NOM{Mix}
  \BIC{$\seq{ \Gamma , \Delta , \nod[m]{A} , \nod[n]{A} }$}
  \NOM{Cont}
  \UIC{$\seq{ \Gamma , \Delta , \nod[m+n]{A} }$}
\end{proofblock}
\begin{proofblock}
  \AXC{$\seq{ \Gamma , A^\bot , \ldots , A^\bot  }$}
  \SYM{\nod{}}
  \UIC{$\seq{ \Gamma , \nod[1]{A^\bot} , \ldots , \nod[1]{A^\bot} }$}
  \NOM{Cont}
  \UIC{$\seq{ \Gamma , \nod[n]{A^\bot} }$}
  \AXC{$\seq{ \Delta , A , \ldots , A }$}
  \SYM{\nod{}}
  \UIC{$\seq{ \Delta , \nod[1]{A} , \ldots , \nod[1]{A}     }$}
  \NOM{Cont}
  \UIC{$\seq{ \Delta , \nod[n]{A} }$}
  \NOM{Cut}
  \BIC{$\seq{ \Gamma , \Delta }$}
\end{proofblock}

\section{Evaluation}

\section{Research outputs}
By the end of this project, I hope to have:
\begin{itemize}
\renewcommand\labelitemi{--}
\item%
  formulated the system \gtcp, which extends CP with a pair of dual
  exponentials, $\give[n]{A}$ and $\take[n]{A}$, for non-deterministic
  computation; 
\item%
  given a cut-elimination procedure for \gtcp;
\item%
  characterized the non-determinism in \gtcp;
\item%
  formulated the system \ndcp, which collapses the dual exponentials
  into a single, self-dual exponential $\nod[n]{A}$; 
\item%
  given a cut-elimination procedure for \ndcp;
\item%
  characterized the non-determinism in \ndcp;
\end{itemize}

\section{Workplan}

\bibliographystyle{apalike}
\bibliography{main}
\nocite{*}

\end{document}