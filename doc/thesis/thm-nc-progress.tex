\begin{theorem}[Progress]\label{thm:nc-progress}
  If $\seq[{ P }]{ \Gamma }$, then $\tm{P}$ is in canonical form, or there
  exists a $\tm{P'}$ s.t.\ $\reducesto{P}{P'}$.
\end{theorem}
\begin{proof}
  By induction on the structure of derivation for $\seq[{ P }]{ \Gamma }$.
  There only interesting cases are when the last rule of the derivation is
  \textsc{Cut} or \textsc{Pool}. In every other case, the typing rule constructs
  a term in which is in canonical form. 
  \\
  If the last rule in the derivation is \textsc{Cut} or \textsc{Pool}, we
  consider the prefix of the derivation for $\seq[{ P }]{ \Gamma}$ which
  consists of all top-level cuts and pooling rules. A prefix of $n$ cuts and $m$
  pooling rules introduces $n$ variables, but composes $n+m+1$ actions, at most
  $m+1$ of which are on the same side of all cut rules.
  Therefore, one of the following must be true:
  \begin{itemize}
  \item
    One of these actions was introduced by an application of \textsc{Ax}.
    \\
    We proceed as in \cref{thm:cp-progress}. 
  \item
    Two of these actions, on different sides of a \textsc{Cut}, act on the same
    channel. Let us name these processes \tm{P_i} and \tm{P_j}, and their shared
    channel \tm{y}. We have
    $\tm{P} = \tm{\cpPlug{G}{\cpCut{y}{\cpPlug{H_i}{P_i}}{\cpPlug{H_j}{P_j}}}}$.
    We distinguish the following cases:
    \begin{itemize}
    \item
      We have either
      $\seq[{ \cpPlug{H_i}{P_i}}]{ \Delta, \tmty{y}{\take[n]{A}} }$ or
      $\seq[{ \cpPlug{H_j}{P_j} }]{\Delta, \tmty{y}{\take[n]{A}} }$.
      \\
      We rewrite by \cref{thm:nc-display-3}, then apply one of \ncRedBetaStar{1}
      and \ncRedBetaStar{n+1}. 
    \item
      Otherwise, we can infer $\notFreeIn{y}{H_i}$ and $\notFreeIn{y}{H_j}$.
      \\
      We proceed as in \cref{thm:cp-progress}. 
    \end{itemize}
  \item
    The process is in canonical form \tm{
      \ncPool{\ncCnt{x_1}{y_1}{P_1}}{
        \ncPool{\dots}{\ncCnt{x_n}{y_n}{P_n}}\dots}}. 
  \item 
    Otherwise (at least) one of the actions acts on a free variable.
    \\
    We apply one of the commutative conversions.
  \end{itemize}
\end{proof}
%%% Local Variables:
%%% TeX-master: "main"
%%% End:
