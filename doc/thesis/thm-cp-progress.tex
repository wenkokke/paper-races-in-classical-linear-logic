\begin{theorem}[Progress]\label{thm:cp-progress}
  If $\seq[{ P }]{ \Gamma }$, then $\tm{P}$ is in canonical form, or there
  exists a $\tm{P'}$ s.t.\ $\reducesto{P}{P'}$. 
\end{theorem}
\begin{proof}
  By induction on the structure of derivation for $\seq[{ P }]{ \Gamma }$.
  The only interesting case is when the last rule of the derivation is
  \textsc{Cut}. In every other case, the typing rule constructs a term in which
  is in canonical form. 
  \\
  If the last rule in the derivation is \textsc{Cut}, we consider the prefix of
  the derivation for $\seq[{ P }]{ \Gamma}$ which consists of all top-level
  cuts. A prefix of $n$ cuts introduces $n$ variables, but composes $n+1$
  actions. Therefore, one of the following must be true: 
  \begin{itemize}
  \item
    One of these actions is a link \tm{\cpLink{x}{y}}. At least one of \tm{x}
    and \tm{y} is a bound name. Let us assume \tm{x} is bound. We have:
    \begin{gather*}
      \begin{array}{ll}
        \tm{P}
        & = \quad \text{see above}
        \\
        \tm{\cpPlug{G}{\cpCut{x}{\cpPlug{H}{\cpLink{x}{y}}}{Q}}}
        & \equiv \quad \text{by \cref{thm:cp-display}}
          \hphantom{\; \text{and} \; \cref{thm:cp-preservation-equiv}}
        \\
        \tm{\cpPlug{G}{\cpPlug{H}{\cpCut{x}{\cpLink{x}{y}}{Q}}}}
      \end{array}
    \end{gather*}
    Similarly if \tm{y} is bound. We then apply one of \cpRedAxCut1 or
    \cpRedAxCut2.
  \item
    Two of these actions, on different sides of a \textsc{Cut}, act on the same
    channel. 
    \\
    Let us name these processes \tm{P_i} and \tm{P_j}, and their shared channel
    \tm{y}. We have:
    \begin{gather*}
      \begin{array}{ll}
        \tm{P}
        & = \quad \text{see above}
        \\
        \tm{\cpPlug{G}{\cpCut{y}{\cpPlug{H_i}{P_i}}{\cpPlug{H_j}{P_j}}}}
        & \equiv \quad \text{by} \; \cref{thm:cp-display}
        \\
        \tm{\cpPlug{G}{\cpPlug{H_i}{\cpCut{y}{P_i}{\cpPlug{H_j}{P_j}}}}}
        & \equiv \quad \text{by} \; \cref{thm:cp-display} \; \text{and} \;
          \cref{thm:cp-preservation-equiv}
        \\
        \tm{\cpPlug{G}{\cpPlug{H_i}{\cpPlug{H_j}{\cpCut{y}{P_i}{P_j}}}}} 
      \end{array}
    \end{gather*}
    We then apply one of the \textbeta-reduction rules.
  \item
    Otherwise (at least) one of the actions acts on a free variable.
    \\
    We then apply one of the commutative conversions.
  \end{itemize}
\end{proof}
%%% Local Variables:
%%% TeX-master: "main"
%%% End:
