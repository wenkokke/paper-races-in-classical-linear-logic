%% Background
\chapter{Background}\label{sec:background}
% * Introduction
\section{Related work}
% * - Work on πDILL and CP
\section{Classical Processes}
In this section, we will discuss a rudimentary subset of the typed process calculus \cp~\cite{wadler2012}.
We do this in order to keep the discussion of our extension in \cref{sec:main} as simple as possible.
We will refer to the subset as \rcp, after its corresponding logic, rudimentary linear logic~\cite[RLL]{girard1992}, also known as multiplicative-additive linear logic.
However, as we foresee no problems in extending the proofs given in \cref{sec:main} to the full version of \cp, we will occasionally use \cp where saying \rcp would be more accurate. 

% * Terms
The process calculus used by \rcp is a variant of the \textpi-calculus~\cite{milner1992b}, and its terms are defined by the following grammar:
\begin{definition}[Terms]\label{def:cp-terms}
  \[\!
    \begin{aligned}
      \tm{P}, \tm{Q}, \tm{R}
           :=& \; \tm{\cpLink{x}{y}}       &&\text{link}
      \\ \mid& \; \tm{\cpCut{x}{P}{Q}}     &&\text{parallel composition}
      \\ \mid& \; \tm{\cpSend{x}{y}{P}{Q}} &&\text{``output''}
      \\ \mid& \; \tm{\cpRecv{x}{y}{P}}    &&\text{``input''}
      \\ \mid& \; \tm{\cpHalt{x}}          &&\text{halt}
      \\ \mid& \; \tm{\cpWait{x}{P}}       &&\text{wait}
      \\ \mid& \; \tm{\cpInl{x}{P}}        &&\text{select left choice}
      \\ \mid& \; \tm{\cpInr{x}{P}}        &&\text{select right choice}
      \\ \mid& \; \tm{\cpCase{x}{P}{Q}}    &&\text{offer binary choice}
      \\ \mid& \; \tm{\cpAbsurd{x}}        &&\text{offer nullary choice}
    \end{aligned}
  \]  
\end{definition}
%%% Local Variables:
%%% TeX-master: "main"
%%% End:

The construct \tm{\cpLink{x}{y}} links two channels~\cite{sangiorgi1996,boreale1998}, forwarding messages received on \tm{x} to \tm{y} and vice versa.
The construct \tm{\cpCut{x}{P}{Q}} creates a new channel \tm{x}, and composes two processes, \tm{P} and \tm{Q}, which communicate on \tm{x}, in parallel. Therefore, in \tm{\cpCut{x}{P}{Q}} the name \tm{x} is bound in both \tm{P} and \tm{Q}. 
In \tm{\cpRecv{x}{y}{P}} and \tm{\cpSend{x}{y}{P}{Q}}, round brackets are used for input, square brackets for output.
We use bound output~\cite{sangiorgi1996}.
This means that unlike in the \textpi-calculus, both input and output bind a new name. 
In \tm{\cpRecv{x}{y}{P}} the new name \tm{y} is bound in \tm{P}.
In \tm{\cpSend{x}{y}{P}{Q}}, the new name \tm{y} is only bound in \tm{P}, while \tm{x} is only bound in \tm{Q}.

% * Types 
Processes in \rcp are typed using a session type system which corresponds to
RLL, the multiplicative and additive connectives from linear logic.
These are defined using the following grammar:
\begin{definition}[Types]\label{def:cp-types}
  \[\!
    \begin{aligned}
      \ty{A}, \ty{B}, \ty{C}
           :=& \; \ty{A \tens B} &&\text{pair of independent processes}
      \\ \mid& \; \ty{A \parr B} &&\text{pair of interdependent processes}
      \\ \mid& \; \ty{\one}      &&\text{unit for} \; {\tens}
      \\ \mid& \; \ty{\bot}      &&\text{unit for} \; {\parr}
      \\ \mid& \; \ty{A \plus B} &&\text{internal choice}
      \\ \mid& \; \ty{A \with B} &&\text{external choice}
      \\ \mid& \; \ty{\nil}      &&\text{unit for} \; {\plus}
      \\ \mid& \; \ty{\top}      &&\text{unit for} \; {\with}
    \end{aligned}
  \]  
\end{definition}
%%% Local Variables:
%%% TeX-master: "main"
%%% End:

We define the interpretations of each connective below, together with the typing
rules which introduce them.

It is commonplace, in linear logic, to define negation as a function on types,
instead of including it in the grammar.
In this thesis, we follow that practice.
\begin{definition}[Duality]\label{def:cp-negation}
  \[\!
    \begin{array}{lclclcl}
              \ty{(A \tens B)^\bot} &=& \ty{A^\bot \parr B^\bot}
      &\quad& \ty{\one^\bot}        &=& \ty{\bot}
      \\      \ty{(A \parr B)^\bot} &=& \ty{A^\bot \tens B^\bot}
      &\quad& \ty{\bot^\bot}        &=& \ty{\one}
      \\      \ty{(A \plus B)^\bot} &=& \ty{A^\bot \with B^\bot}
      &\quad& \ty{\nil^\bot}        &=& \ty{\top}
      \\      \ty{(A \with B)^\bot} &=& \ty{A^\bot \plus B^\bot}
      &\quad& \ty{\top^\bot}        &=& \ty{\nil}
    \end{array}
  \]
\end{definition}
%%% Local Variables:
%%% TeX-master: "main"
%%% End:

We define our negation such that it is involutive.
\begin{lemma}[Involutive]\label{thm:cp-negation-involutive}
  We have $\ty{A^{\bot\bot}} = \ty{A}$.
\end{lemma}
  \begin{proof}
    By induction on the structure of the type $\ty{A}$.
  \end{proof}
%%% Local Variables:
%%% TeX-master: "main"
%%% End:

Environments associate channels with types. They are defined as follows:
\begin{definition}[Environments]\label{def:cp-environments}
  We define environments as follows:
  \[
    \ty{\Gamma}, \ty{\Delta}, \ty{\Theta}
    ::= \tmty{x_1}{A_1}\dots\tmty{x_n}{A_n}
  \] 
  Names in environments must be unique, and environments \ty{\Gamma} and
  \ty{\Delta} can only be combined as $\ty{\Gamma}, \ty{\Delta}$ if
  $\text{fv}(\ty{\Gamma}) \cap \text{fv}(\ty{\Delta}) = \varnothing$. 
\end{definition}
%%% Local Variables:
%%% TeX-master: "main"
%%% End:

Typing judgements associative processes with their collection of channels, and
enforce the communication protocols specified by the types of those channels.
They are defined as follows:
\input{def-cp-typing-judgement}
\begin{figure*}[b]
  \begin{center}
    \cpInfAx
    \cpInfCut
  \end{center}
  \begin{center}
    \cpInfTens
    \cpInfParr
  \end{center}
  \begin{center}
    \cpInfOne
    \cpInfBot
  \end{center}
  \begin{center}
    \cpInfPlus1
    \cpInfPlus2
  \end{center}
  \begin{center}
    \cpInfWith
  \end{center}
  \begin{center}
    \cpInfNil
    \cpInfTop
  \end{center}
  \caption{Typing judgement for the multiplicative applicative subset of \rcp.}
  \label{fig:cp-typing-judgement}
\end{figure*}
%%% Local Variables:
%%% TeX-master: "main"
%%% End:

The multiplicatives ($\ty{\tens}, \ty{\parr}$) deal with independence and
interdependence. A channel of type \ty{A \tens B} represents a pair of channels,
which communicate with two \emph{independent} processes---that is to say, two
processes who share no channels. A channel of type \ty{A \parr B} represents a
pair of interdependent channels, which are used within a single process. This
means that their usages can depend on one another---e.g.\ the interaction of
type \ty{B} could depend on the result of the interaction of type \ty{A}, or
vise versa, and if \ty{A} and \ty{B} are complex types, then their interactions
could likewise interweave in complex ways.
While the rules for \ty{\tens} and \ty{\parr} introduce input and output
operations, these are inessential---the essential distinction lies two in the
fact that (\tens) composes two independent processes, and must split the
environment between them, whereas (\parr) uses a single process, which can use
all the channels in the environment.

The rules for the multiplicative units ($\ty{\one}, \ty{\bot}$) follow the same
pattern: (\one) composes \emph{zero} independent processes, and thus must halt;
and (\bot) uses a single process, which is not further restricted.

% * Reduction
Terms in \rcp are identified up to structural congruence, which states that parallel compositions \tm{\cpCut{x}{P}{Q}} are associative and commutative.
It is defined as follows:
\begin{definition}[Structural congruence]\label{def:cp-equiv}
  We define the structural congruence $\equiv$ as reflexivity, transitivity, and
  congruence over terms, plus the following two axioms:
  \[
    \begin{array}{llll}
      \cpEquivCutComm
      & \tm{\cpCut{x}{P}{Q}}
      & \equiv \;
      & \tm{\cpCut{x}{Q}{P}}
      \\
      \cpEquivCutAss1
      & \tm{\cpCut{x}{P}{\cpCut{y}{Q}{R}}}
      & \equiv \;
      & \tm{\cpCut{y}{\cpCut{x}{P}{Q}}{R}}
        \quad \text{if} \; \notFreeIn{x}{R} \; \text{and} \; \notFreeIn{y}{P}
    \end{array}
  \]
\end{definition}
%%% Local Variables:
%%% TeX-master: "main"
%%% End:

We do not add an axiom for \cpEquivCutAss2, as it follows from \cref{def:cp-equiv}, which we prove below.
Note that, throughout this thesis, we will leave uses of the transitivity and congruence rules implicit.
\begin{lemmaB}[\cpEquivCutAssNoParen2]\label{thm:cp-cut-assoc2}
  If $\tm{x}\not\in\tm{R}$ and $\tm{y}\not\in\tm{P}$, then 
  \(
    \tm{\cpCut{y}{\cpCut{x}{P}{Q}}{R}} \equiv
    \tm{\cpCut{x}{P}{\cpCut{y}{Q}{R}}}
  \).
\end{lemmaB}
  \begin{proof}
    \begin{align*}
      \tm{\cpCut{y}{\cpCut{x}{P}{Q}}{R}} &\equiv \qquad \text{by \cpEquivCutComm} \\
      \tm{\cpCut{y}{\cpCut{x}{Q}{P}}{R}} &\equiv \qquad \text{by \cpEquivCutComm} \\
      \tm{\cpCut{y}{R}{\cpCut{x}{Q}{P}}} &\equiv \qquad \text{by \cpEquivCutAss1} \\
      \tm{\cpCut{x}{\cpCut{y}{R}{Q}}{P}} &\equiv \qquad \text{by \cpEquivCutComm} \\
      \tm{\cpCut{x}{P}{\cpCut{y}{R}{Q}}} &\equiv \qquad \text{by \cpEquivCutComm} \\
      \tm{\cpCut{x}{P}{\cpCut{y}{Q}{R}}}
    \end{align*}
    The side conditions for \cpEquivCutAss1 are given.
  \end{proof}
%%% Local Variables:
%%% TeX-master: "main"
%%% End:

Furthermore, the structural congruence from \cref{def:cp-equiv} is a symmetric
relation.
\begin{theorem}[Symmetry]\label{thm:cp-symmetry}
  If $\tm{P} \equiv \tm{Q}$, then $\tm{Q} \equiv \tm{P}$.
\end{theorem}
  \begin{proof}
    By induction on the structure of the equivalence proof.
    The only interesting case is \cpEquivCutAss1, which follows from
    \cref{thm:cp-cut-assoc2}.
  \end{proof}
%%% Local Variables:
%%% TeX-master: "main"
%%% End:

\begin{definition}[Term reduction]\label{def:cp-reduction}
  \[
    \begin{array}{llll}
      \cpRedAxCut1
      & \tm{\cpCut{x}{\cpLink{w}{x}}{P}}
      & \Longrightarrow \;
      & \tm{\cpSub{w}{x}{P}} 
      \\
      \cpRedAxCut2
      & \tm{\cpCut{x}{\cpLink{x}{w}}{P}}
      & \Longrightarrow \;
      & \tm{\cpSub{w}{x}{P}} 
      \\
      \\
      \cpRedBetaTensParr
      & \tm{\cpCut{x}{\cpSend{x}{y}{P}{Q}}{\cpRecv{x}{z}{R}}}
      & \Longrightarrow \;
      & \tm{\cpCut{y}{P}{\cpCut{x}{Q}{\cpSub{y}{z}{R}}}}
      \\
      \cpRedBetaOneBot
      & \tm{\cpCut{x}{\cpHalt{x}}{\cpWait{x}{P}}}
      & \Longrightarrow \;
      & \tm{P}
      \\
      \cpRedBetaPlusWith1
      & \tm{\cpCut{x}{\cpInl{x}{P}}{\cpCase{x}{Q}{R}}}
      & \Longrightarrow \;
      & \tm{\cpCut{x}{P}{Q}}
      \\
      \cpRedBetaPlusWith2
      & \tm{\cpCut{x}{\cpInr{x}{P}}{\cpCase{x}{Q}{R}}}
      & \Longrightarrow \;
      & \tm{\cpCut{x}{P}{R}}
      \\
      \\
      \cpRedKappaTens1
      & \tm{\cpCut{x}{\cpSend{y}{z}{P}{Q}}{R}}
      & \Longrightarrow \;
      & \tm{\cpSend{y}{z}{\cpCut{x}{P}{R}}{Q}} \quad \text{if} \; \notFreeIn{x}{Q}
      \\
      \cpRedKappaTens2
      & \tm{\cpCut{x}{\cpSend{y}{z}{P}{Q}}{R}}
      & \Longrightarrow \;
      & \tm{\cpSend{y}{z}{P}{\cpCut{x}{Q}{R}}} \quad \text{if} \; \notFreeIn{x}{P}
      \\
      \cpRedKappaParr
      & \tm{\cpCut{x}{\cpRecv{y}{z}{P}}{R}}
      & \Longrightarrow \;
      & \tm{\cpRecv{y}{z}{\cpCut{x}{P}{R}}}
      \\
      \cpRedKappaBot
      & \tm{\cpCut{x}{\cpWait{y}{P}}{R}}
      & \Longrightarrow \;
      & \tm{\cpWait{y}{\cpCut{x}{P}{R}}}
      \\
      \cpRedKappaPlus1
      & \tm{\cpCut{x}{\cpInl{y}{P}}{R}}
      & \Longrightarrow \;
      & \tm{\cpInl{y}{\cpCut{x}{P}{R}}}
      \\
      \cpRedKappaPlus2
      &\tm{\cpCut{x}{\cpInr{y}{P}}{R}}
      & \Longrightarrow \;
      & \tm{\cpInr{y}{\cpCut{x}{P}{R}}}
      \\
      \cpRedKappaWith
      & \tm{\cpCut{x}{\cpCase{y}{P}{Q}}{R}}
      & \Longrightarrow \;
      & \tm{\cpCase{y}{\cpCut{x}{P}{R}}{\cpCut{x}{Q}{R}}}
      \\
      \cpRedKappaTop
      & \tm{\cpCut{x}{\cpAbsurd{y}}{R}}
      & \Longrightarrow \;
      & \tm{\cpAbsurd{y}}
    \end{array}
  \]

  \begin{center}
    \begin{prooftree*}
      \AXC{\reducesto{P}{P^\prime}}
      \SYM{\cpRedGammaCut}
      \UIC{\reducesto{\cpCut{x}{P}{Q}}{\cpCut{x}{P^\prime}{Q}}}
    \end{prooftree*}
    \begin{prooftree*}
      \AXC{$\tm{P}\equiv\tm{Q}$}
      \AXC{\reducesto{Q}{Q^\prime}}
      \AXC{$\tm{Q^\prime}\equiv\tm{P^\prime}$}
      \SYM{\cpRedGammaEquiv}
      \TIC{\reducesto{P}{P^\prime}}
    \end{prooftree*}
  \end{center}
\end{definition}
%%% Local Variables:
%%% TeX-master: "main"
%%% End:


% * Preservation 
\begin{lemma}[Preservation for $\equiv$]\label{thm:cp-preservation-equiv}
  If $\tm{P} \equiv \tm{Q}$ and $\seq[{ P }]{ \Gamma }$, then $\seq[{ Q }]{ \Gamma }$.
\end{lemma}
\begin{proof}
  By induction on the structure of the equivalence. The cases for reflexivity,
  transitivity and congruence are trivial. The two interesting cases, for
  \cpEquivCutComm and \cpEquivCutAss1 are given in \cref{fig:cp-preservation-equiv}
\end{proof}
%%% Local Variables:
%%% TeX-master: "main"
%%% End:

\begin{figure*}[htb]
  \centering
  \begin{tabular}{ll}
    \cpEquivLinkComm
    &
      \begin{prooftree*}
        \AXC{}
        \NOM{Ax}
        \UIC{$\seq[{ \cpLink{x}{y} }]{ \tmty{x}{A}, \tmty{y}{A^\bot} }$}
      \end{prooftree*}
    \\[20pt]
    $\equiv$
    &
      \begin{prooftree*}
        \AXC{}
        \NOM{Ax}
        \UIC{$\seq[{ \cpLink{y}{x} }]{ \tmty{y}{A^\bot}, \tmty{x}{A^{\bot\bot}} }$}
        \RightLabel{$(\cdot)^\bot$-Involutive}
        \UIC{$\seq[{ \cpLink{y}{x} }]{ \tmty{y}{A^\bot}, \tmty{x}{A} }$}
      \end{prooftree*}
    \\[25pt]
    \cpEquivCutComm
    &
      \begin{prooftree*}
        \AXC{$\seq[{ P }]{ \Gamma, \tmty{x}{A} }$}
        \AXC{$\seq[{ Q }]{ \Delta, \tmty{x}{A^\bot} }$}
        \NOM{Cut}
        \BIC{$\seq[{ \cpCut{x}{P}{Q} }]{ \Gamma, \Delta }$}
      \end{prooftree*}
    \\[20pt]
    $\equiv$
    &
      \begin{prooftree*}
        \AXC{$\seq[{ Q }]{ \Delta, \tmty{x}{A^\bot} }$}
        \AXC{$\seq[{ P }]{ \Gamma, \tmty{x}{A} }$}
        \RightLabel{$(\cdot)^\bot$-Involutive}
        \UIC{$\seq[{ P }]{ \Gamma, \tmty{x}{A^{\bot\bot}} }$}
        \NOM{Cut}
        \BIC{$\seq[{ \cpCut{x}{Q}{P} }]{ \Gamma, \Delta }$}
      \end{prooftree*}
    \\[25pt]
    \cpEquivCutAss1
    &
      \begin{prooftree*}
        \AXC{$\seq[{ P }]{ \Gamma, \tmty{x}{A} }$}
        \AXC{$\seq[{ Q }]{ \Delta, \tmty{x}{A^\bot}, \tmty{y}{B} }$}
        \AXC{$\seq[{ R }]{ \Theta, \tmty{y}{B^\bot} }$}
        \NOM{Cut}
        \BIC{$\seq[{ \cpCut{y}{Q}{R} }]{ \Delta, \Theta, \tmty{x}{A^\bot} }$}
        \NOM{Cut}
        \BIC{$\seq[{ \cpCut{x}{P}{\cpCut{y}{Q}{R}} }]{ \Gamma, \Delta, \Theta }$}
      \end{prooftree*}
    \\[20pt]
    $\equiv$
    &
      \begin{prooftree*}
        \AXC{$\seq[{ P }]{ \Gamma, \tmty{x}{A} }$}
        \AXC{$\seq[{ Q }]{ \Delta, \tmty{x}{A^\bot}, \tmty{y}{B} }$}
        \NOM{Cut}
        \BIC{$\seq[{ \cpCut{x}{P}{Q} }]{ \Gamma, \Delta, \tmty{y}{B} }$}
        \AXC{$\seq[{ R }]{ \Theta, \tmty{y}{B^\bot} }$}
        \NOM{Cut}
        \BIC{$\seq[{ \cpCut{y}{\cpCut{x}{P}{Q}}{R} }]{ \Gamma, \Delta, \Theta }$}
      \end{prooftree*}
  \end{tabular}
  \caption{Type preservation for the structural congruence of \cp}
  \label{fig:cp-preservation-equiv}
\end{figure*}
%%% Local Variables:
%%% TeX-master: "main"
%%% End:

\begin{theorem}[Preservation]\label{thm:cp-preservation}
  If \reducesto{P}{Q} and $\seq[{ P }]{ \Gamma }$, then $\seq[{ Q }]{ \Gamma }$.
\end{theorem}
\begin{proof}
  By induction on the structure of the reduction. See
  \cref{fig:cp-preservation-1} for the \cpRedAxCut{i} and \textbeta-reduction
  rules, and \cref{fig:cp-preservation-2a,fig:cp-preservation-2b} for the
  commutative conversions.
  %
  The case for \cpRedGammaCut is trivial by the induction hypothesis, and the
  case for \cpRedGammaEquiv is trivial by the induction hypothesis and
  \cref{thm:cp-preservation-equiv}. 
\end{proof}
%%% Local Variables:
%%% TeX-master: "main"
%%% End:

\input{fig-cp-preservation-1}
\begin{figure*}[ht]
  \centering
  \begin{tabular}{ll}
    \cpRedKappaTens1
    &
      \begin{prooftree*}
        \AXC{$\seq[{ P }]{ \Gamma, \tmty{x}{A}, \tmty{z}{B} }$}
        \AXC{$\seq[{ Q }]{ \Delta, \tmty{y}{C} }$}
        \SYM{\tens}
        \BIC{$\seq[{ \cpSend{y}{z}{P}{Q} }]{ \Gamma, \Delta, \tmty{y}{B \tens C} }$}
        \AXC{$\seq[{ R }]{ \Theta, \tmty{x}{A^\bot} }$}
        \NOM{Cut}
        \BIC{$\seq[{ \cpCut{x}{\cpSend{y}{z}{P}{Q}}{R} }]{ \Gamma, \Delta, \Theta, \tmty{y}{B \tens C} }$}
      \end{prooftree*}
    \\[30pt]
    $\Longrightarrow$
    &
       \begin{prooftree*}
         \AXC{$\seq[{ P }]{ \Gamma, \tmty{x}{A}, \tmty{z}{B} }$}
         \AXC{$\seq[{ R }]{ \Theta, \tmty{x}{A^\bot} }$}
         \NOM{Cut}
         \BIC{$\seq[{ \cpCut{x}{P}{Q} }]{ \Gamma, \Theta, \tmty{z}{B} }$}
         \AXC{$\seq[{ Q }]{ \Delta, \tmty{y}{C} }$}
         \SYM{\tens}
         \BIC{$\seq[{ \cpSend{y}{z}{\cpCut{x}{P}{Q}}{R} }]{ \Gamma, \Delta, \Theta, \tmty{y}{B \tens C} }$}
       \end{prooftree*}
    \\[30pt]
    \cpRedKappaTens2
    &
      (as above)
    \\[20pt]
    \cpRedKappaParr
    &
      \begin{prooftree*}
        \AXC{$\seq[{ P }]{ \Gamma, \tmty{x}{A}, \tmty{z}{B}, \tmty{y}{C} }$}
        \SYM{\parr}
        \UIC{$\seq[{ \cpRecv{y}{z}{P} }]{ \Gamma, \tmty{x}{A}, \tmty{y}{B \parr C} }$}
        \AXC{$\seq[{ R }]{ \Theta, \tmty{x}{A^\bot} }$}
        \NOM{Cut} 
        \BIC{$\seq[{ \cpCut{x}{\cpRecv{y}{z}{P}}{R} }]{ \Gamma, \Theta, \tmty{y}{B \parr C} }$}
      \end{prooftree*}
    \\[30pt]
    $\Longrightarrow$
    &
      \begin{prooftree*}
        \AXC{$\seq[{ P }]{ \Gamma, \tmty{x}{A}, \tmty{z}{B}, \tmty{y}{C} }$}
        \AXC{$\seq[{ R }]{ \Theta, \tmty{x}{A^\bot} }$}
        \NOM{Cut}
        \BIC{$\seq[{ \cpCut{x}{P}{R} }]{ \Gamma, \Theta, \tmty{z}{B}, \tmty{y}{C} }$}
        \SYM{\parr}
        \UIC{$\seq[{ \cpRecv{y}{z}{\cpCut{x}{P}{R}} }]{ \Gamma, \Theta, \tmty{y}{B \parr C} }$}
      \end{prooftree*}
    \\[40pt]
    \cpRedKappaBot
    &
      \begin{prooftree*}
        \AXC{$\seq[{ P }]{ \Gamma, \tmty{x}{A} }$}
        \SYM{\bot}
        \UIC{$\seq[{ \cpWait{y}{P} }]{ \Gamma, \tmty{x}{A}, \tmty{y}{\bot} }$}
        \AXC{$\seq[{ R }]{ \Theta, \tmty{x}{A^\bot} }$}
        \NOM{Cut} 
        \BIC{$\seq[{ \cpCut{x}{\cpWait{y}{P}}{R} }]{ \Gamma, \Theta, \tmty{y}{\bot} }$}
      \end{prooftree*}
    \\[30pt]
    $\Longrightarrow$
    &
      \begin{prooftree*}
        \AXC{$\seq[{ P }]{ \Gamma, \tmty{x}{A} }$}
        \AXC{$\seq[{ R }]{ \Theta, \tmty{x}{A^\bot} }$}
        \NOM{Cut} 
        \BIC{$\seq[{ \cpCut{x}{P}{R} }]{ \Gamma, \Theta }$}
        \SYM{\bot}
        \UIC{$\seq[{ \cpWait{y}{\cpCut{x}{P}{R}} }]{ \Gamma, \Theta, \tmty{y}{\bot} }$}
      \end{prooftree*}
  \end{tabular}

  \caption{Type preservation for the commuting conversions of \cp}
  \label{fig:cp-preservation-2a}
\end{figure*}
\begin{figure*}[ht]
  \makebox[\textwidth][c]{
    \begin{tabular}{ll}
      \cpRedKappaPlus1
      &
        \begin{prooftree*}
          \AXC{$\seq[{ P }]{ \Gamma, \tmty{x}{A}, \tmty{y}{B} }$}
          \SYM{\plus_1}
          \UIC{$\seq[{ \cpInl{y}{P} }]{ \Gamma, \tmty{x}{A}, \tmty{y}{B \plus C} }$}
          \AXC{$\seq[{ R }]{ \Theta, \tmty{x}{A^\bot} }$}
          \NOM{Cut}
          \BIC{$\seq[{ \cpCut{x}{\cpInl{y}{P}}{R} }]{ \Gamma, \Theta, \tmty{y}{B \plus C} }$}
        \end{prooftree*}
      \\[30pt]
      $\Longrightarrow$
      &
        \begin{prooftree*}
          \AXC{$\seq[{ P }]{ \Gamma, \tmty{x}{A}, \tmty{y}{B} }$}
          \AXC{$\seq[{ R }]{ \Theta, \tmty{x}{A^\bot} }$}
          \NOM{Cut}
          \BIC{$\seq[{ \cpCut{x}{P}{R} }]{ \Gamma, \Theta, \tmty{y}{B} }$}
          \SYM{\plus_1}
          \UIC{$\seq[{ \cpInl{y}{\cpCut{x}{P}{R}} }]{ \Gamma, \Theta, \tmty{y}{B \plus C} }$}
        \end{prooftree*}
      \\[30pt] 
      \cpRedKappaPlus2
      &
        (as above)
      \\[20pt]
      \cpRedKappaWith
      &
        \begin{prooftree*}
          \AXC{$\seq[{ P }]{ \Gamma, \tmty{x}{A}, \tmty{y}{B} }$}
          \AXC{$\seq[{ Q }]{ \Gamma, \tmty{x}{A}, \tmty{y}{C} }$}
          \SYM{\with}
          \BIC{$\seq[{ \cpCase{y}{P}{Q} }]{ \Gamma, \tmty{x}{A}, \tmty{y}{B \with C} }$}
          \AXC{$\seq[{ R }]{ \Theta, \tmty{x}{A^\bot} }$}
          \NOM{Cut}
          \BIC{$\seq[{ \cpCut{x}{\cpCase{y}{P}{Q}}{R} }]{ \Gamma, \Theta, \tmty{y}{B \with C} }$}
        \end{prooftree*}
      \\[30pt]
      $\Longrightarrow$
      &
        \begin{prooftree*}
          \AXC{$\seq[{ P }]{ \Gamma, \tmty{x}{A}, \tmty{y}{B} }$}
          \AXC{$\seq[{ R }]{ \Theta, \tmty{x}{A^\bot} }$}
          \NOM{Cut}
          \BIC{$\seq[{ \cpCut{x}{P}{R} }]{ \Gamma, \Theta, \tmty{y}{B} }$}
          \AXC{$\seq[{ Q }]{ \Gamma, \tmty{x}{A}, \tmty{y}{C} }$}
          \AXC{$\seq[{ R }]{ \Theta, \tmty{x}{A^\bot} }$}
          \NOM{Cut}
          \BIC{$\seq[{ \cpCut{x}{Q}{R} }]{ \Gamma, \Theta, \tmty{y}{C} }$}
          \SYM{\with}
          \BIC{$\seq[{ \cpCase{y}{\cpCut{x}{P}{R}}{\cpCut{x}{Q}{R}} }]{ \Gamma, \Theta, \tmty{y}{B \with C} }$}
        \end{prooftree*}
      \\[40pt] 
      \cpRedKappaTop
      &
        \begin{prooftree*}
          \AXC{}
          \SYM{\top}
          \UIC{$\seq[{ \cpAbsurd{y} }]{ \Gamma, \tmty{x}{A}, \tmty{y}{\top} }$}
          \AXC{$\seq[{ R }]{ \Theta, \tmty{x}{A^\bot} }$}
          \NOM{Cut}
          \BIC{$\seq[{ \cpCut{x}{\cpAbsurd{y}}{R} }]{ \Gamma, \Theta, \tmty{y}{\top} }$}
        \end{prooftree*}
      \\[30pt]
      $\Longrightarrow$
      &
        \begin{prooftree*}
          \AXC{}
          \SYM{\top}
          \UIC{$\seq[{ \cpAbsurd{y} }]{ \Gamma, \Theta, \tmty{y}{\top} }$}
        \end{prooftree*}
    \end{tabular}
  }
  \caption{Type preservation for the commuting conversions of \cp (cont'd)}
  \label{fig:cp-preservation-2b}
\end{figure*}
%%% Local Variables:
%%% TeX-master: "main"
%%% End:

% * Progress
\begin{definition}[Action]\label{def:cp-action}
  A process $\tm{P}$ \emph{acts on} a channel $\tm{x}$ if it is of the form
  $\tm{\cpSend{x}{y}{P'}{Q'}}$, $\tm{\cpRecv{x}{y}{P'}}$, $\tm{\cpHalt{x}}$,
  $\tm{\cpWait{x}{P'}}$, $\tm{\cpInl{x}{P'}}$, $\tm{\cpInr{x}{P'}}$,
  $\tm{\cpCase{x}{P'}{Q'}}$ or $\tm{\cpAbsurd{x}}$. 
\end{definition}
%%% Local Variables:
%%% TeX-master: "main"
%%% End:

\begin{definition}[Canonical forms]\label{def:cp-canonical-forms}
  A process \tm{P} is in canonical form if it is in one of the following forms:
  \begin{multicols}{3}
    \begin{itemize}[noitemsep,topsep=0pt,parsep=0pt,partopsep=0pt]
    \item \tm{\cpLink{x}{y}}
    \item \tm{\cpSend{x}{y}{P'}{Q'}}
    \item \tm{\cpRecv{x}{y}{P'}}
    \item \tm{\cpHalt{x}}
    \item \tm{\cpWait{x}{P'}}
    \item \tm{\cpInl{x}{P'}}
    \item \tm{\cpInr{x}{P'}}
    \item \tm{\cpCase{x}{P'}{Q'}}
    \item \tm{\cpAbsurd{x}}
    \end{itemize}
  \end{multicols}
\end{definition}
%%% Local Variables:
%%% TeX-master: "main"
%%% End:

\begin{definition}[Evaluation contexts]\label{def:cp-evaluation-contexts}
  We define evaluation contexts as:
  \begin{align*}
    \tm{G}, \tm{H} := \tm{\Box}
    \mid \tm{\cpCut{x}{G}{P}}
    \mid \tm{\cpCut{x}{P}{G}}
  \end{align*}
\end{definition}
\begin{definition}[Plugging]\label{def:cp-evaluation-context-plugging}
  We define plugging for evaluation contexts as:
  \begin{gather*}
    \begin{array}{ll}
      \tm{\cpPlug{\Box}{R}}            
      & := \; \tm{R}
      \\
      \tm{\cpPlug{\cpCut{x}{G}{P}}{R}}
      & := \; \tm{\cpCut{x}{\cpPlug{G}{R}}{P}}
      \\
      \tm{\cpPlug{\cpCut{x}{P}{G}}{R}}
      & := \; \tm{\cpCut{x}{P}{\cpPlug{G}{R}}}
    \end{array}
  \end{gather*}
\end{definition}
%%% Local Variables:
%%% TeX-master: "main"
%%% End:

\begin{definition}[Plugging]\label{def:cp-evaluation-contexts-plugging}
  We define plugging for evaluation contexts as:
  \begin{gather*}
    \begin{array}{ll}
      \tm{\cpPlug{\Box}{R}}            
      & := \; \tm{R}
      \\
      \tm{\cpPlug{\cpCut{x}{G}{P}}{R}}
      & := \; \tm{\cpCut{x}{\cpPlug{G}{R}}{P}}
      \\
      \tm{\cpPlug{\cpCut{x}{P}{G}}{R}}
      & := \; \tm{\cpCut{x}{P}{\cpPlug{G}{R}}}
    \end{array}
  \end{gather*}
\end{definition}
%%% Local Variables:
%%% TeX-master: "main"
%%% End:

\input{thm-cp-display}
\begin{theorem}[Progress]\label{thm:cp-progress}
  If $\seq[{ P }]{ \Gamma }$, then $\tm{P}$ is in canonical form, or there
  exists a $\tm{P'}$ s.t.\ $\reducesto{P}{P'}$. 
\end{theorem}
\begin{proof}
  By induction on the structure of derivation for $\seq[{ P }]{ \Gamma }$.
  The only interesting case is when the last rule of the derivation is
  \textsc{Cut}. In every other case, the typing rule constructs a term in which
  is in canonical form. 
  \\
  If the last rule in the derivation is \textsc{Cut}, we consider the prefix of
  the derivation for $\seq[{ P }]{ \Gamma}$ which consists of all top-level
  cuts. A prefix of $n$ cuts introduces $n$ variables, but composes $n+1$
  actions. Therefore, one of the following must be true: 
  \begin{itemize}
  \item
    One of these actions is a link \tm{\cpLink{x}{y}}. At least one of \tm{x}
    and \tm{y} is a bound name. Let us assume \tm{x} is bound. We have:
    \begin{gather*}
      \begin{array}{ll}
        \tm{P}
        & = \quad \text{see above}
        \\
        \tm{\cpPlug{G}{\cpCut{x}{\cpPlug{H}{\cpLink{x}{y}}}{Q}}}
        & \equiv \quad \text{by \cref{thm:cp-display}}
          \hphantom{\; \text{and} \; \cref{thm:cp-preservation-equiv}}
        \\
        \tm{\cpPlug{G}{\cpPlug{H}{\cpCut{x}{\cpLink{x}{y}}{Q}}}}
      \end{array}
    \end{gather*}
    Similarly if \tm{y} is bound. We then apply one of \cpRedAxCut1 or
    \cpRedAxCut2.
  \item
    Two of these actions, on different sides of a \textsc{Cut}, act on the same
    channel. 
    \\
    Let us name these processes \tm{P_i} and \tm{P_j}, and their shared channel
    \tm{y}. We have:
    \begin{gather*}
      \begin{array}{ll}
        \tm{P}
        & = \quad \text{see above}
        \\
        \tm{\cpPlug{G}{\cpCut{y}{\cpPlug{H_i}{P_i}}{\cpPlug{H_j}{P_j}}}}
        & \equiv \quad \text{by} \; \cref{thm:cp-display}
        \\
        \tm{\cpPlug{G}{\cpPlug{H_i}{\cpCut{y}{P_i}{\cpPlug{H_j}{P_j}}}}}
        & \equiv \quad \text{by} \; \cref{thm:cp-display} \; \text{and} \;
          \cref{thm:cp-preservation-equiv}
        \\
        \tm{\cpPlug{G}{\cpPlug{H_i}{\cpPlug{H_j}{\cpCut{y}{P_i}{P_j}}}}} 
      \end{array}
    \end{gather*}
    We then apply one of the \textbeta-reduction rules.
  \item
    Otherwise (at least) one of the actions acts on a free variable.
    \\
    We then apply one of the commutative conversions.
  \end{itemize}
\end{proof}
%%% Local Variables:
%%% TeX-master: "main"
%%% End:

\begin{theorem}[Termination]\label{thm:cp-termination}
  If $\seq[{ P }]{ \Gamma }$, then there are no infinite $\Longrightarrow$
  reduction sequences.
\end{theorem}
  \begin{proof}
    Every reduction reduces a single cut to zero, one or two cuts.
    However, each of these cuts is \emph{smaller}, in the sense that the type of
    the channel on which the communication takes place is smaller, as each
    reduction eliminates a connective---see
    \cref{fig:cp-preservation-1,fig:cp-preservation-2a,fig:cp-preservation-2b}.
    Therefore, there cannot be an infinite reduction sequence.
  \end{proof}
%%% Local Variables:
%%% TeX-master: "main"
%%% End:

%%% Local Variables:
%%% TeX-master: "main"
%%% End:
