\documentclass[12pt,a4paper,UKenglish,mscres,logo,twoside,plainprepages,notimes,lfcs]{infthesis}
\shieldtype{3}
\usepackage{microtype}
\usepackage{alphabeta}
\usepackage[greek,english]{babel}
\languageattribute{greek}{polutoniko}
\usepackage[dvipsnames,table]{xcolor}
\colorlet{tm}{Black}
\colorlet{ty}{Black}

\usepackage{fullpage}
\usepackage{amsmath, amscd, amsthm, amssymb, mathrsfs,amsfonts}
\usepackage{textcomp}
\usepackage{txfonts}
\usepackage{graphicx}

%%% TERM CALCULUS
\newcommand{\link}[2]{#1{\leftrightarrow}#2}
\newcommand{\wait}[1]{#1().}
\newcommand{\halt}[1]{#1[].0}
\newcommand{\send}[2]{#1[#2].}
\newcommand{\recv}[2]{#1(#2).}
\newcommand{\case}[3]{%
\toks0={#2#3}%
\edef\param{\the\toks0}%
\ifx\param\empty
  \ensuremath{\text{case} \; #1 \; \{\}}
\else
  \ensuremath{\text{case} \; #1 \; \{ #2 ; #3 \}}
\fi}
\newcommand{\inl}[1]{#1[\text{inl}].}
\newcommand{\inr}[1]{#1[\text{inr}].}
\newcommand{\expn}[2]{#1 \uparrow #2.}
\newcommand{\intl}[2]{#1 \downarrow #2.}
\newcommand{\cut}[1]{\nu #1.}

\def\parr{\ensuremath{\mathbin{\rotatebox[origin=c]{180}{\&}}}}
\def\with{\ensuremath{\mathbin{\text{\&}}}}
\def\plus{\ensuremath{\oplus}}
\def\tens{\ensuremath{\otimes}}
\def\limp{\ensuremath{\multimap}}
\def\one{\ensuremath{\mathbf{1}}}
\def\nil{\ensuremath{\mathbf{0}}}
\def\lequiv{\ensuremath{\multimapboth}}

\DeclareGraphicsRule{.ai}{pdf}{.ai}{}
\newcommand{\emoji}[2][1em]{\ensuremath{\vcenter{\hbox{\includegraphics[width=#1]{twemoji/assets/#2.ai}}}}}
\newcommand{\mary}[1][1em]{\emoji[#1]{1f469}}
\newcommand{\john}[1][1em]{\emoji[#1]{1f466}}
\newcommand{\cake}[1][1em]{\emoji[#1]{1f382}}
\newcommand{\plato}[1][1em]{\emoji[#1]{1f381}}
\newcommand{\nocake}[1][1em]{\emoji[#1]{1f64c}}
\newcommand{\money}[1][1em]{\emoji[#1]{1f4b0}}
\newcommand{\dollar}[1][1em]{\emoji[#1]{1f4b5}}
\newcommand{\ptis}[1][1em]{\emoji[#1]{1f3ea}}
\newcommand{\good}[1][1em]{\emoji[#1]{2714}}
\newcommand{\bad}[1][1em]{\emoji[#1]{2716}}

\newcommand{\tm}[2][]{%
\toks0={#1}%
\edef\param{\the\toks0}%
\ifx\param\empty
  \ensuremath{{\color{ty}#2}}%
\else
  \ensuremath{{\color{tm}#1}\colon{\color{ty}#2}}%
\fi}
\newcommand{\seq}[2][]{\ensuremath{{\color{tm}#1}\vdash{\color{ty}#2}}}
\newcommand{\subst}[3]{\ensuremath{#1 \{ #2 / #3 \}}}
\newcommand{\nod}[2][]{\ensuremath{{\star}_{#1}{#2}}}
\newcommand{\give}[2][]{\ensuremath{{ ? }_{#1}{#2}}}
\newcommand{\take}[2][]{\ensuremath{{ ! }_{#1}{#2}}}

\usepackage{bussproofs}
\def\fCenter{\ensuremath{\;\vdash\;}}
\newcommand{\NOM}[1]{\RightLabel{\textsc{#1}}}
\newcommand{\SYM}[1]{\RightLabel{\ensuremath{#1}}}
\EnableBpAbbreviations
\newenvironment{proofbox}[1][0.9]%
  {\gdef\scalefactor{#1} \leavevmode\hbox\bgroup}
  {\scalebox{\scalefactor}{\DisplayProof} \egroup}
\newenvironment{proofblock}[1][0.9]%
  {\gdef\scalefactor{#1}\begin{center}\proofSkipAmount \leavevmode}%
  {\scalebox{\scalefactor}{\DisplayProof}\proofSkipAmount \end{center} }

%%% Local Variables:
%%% TeX-master: "main"
%%% End:
\title{Races in Classical Linear Logic}
\author{%
  Wen Kokke\\
  Supervised by Philip Wadler and J.\ Garrett Morris}
\date{}
\addbibresource{main.bib}
\begin{document}
\maketitle

%% Introduction
\chapter{Introduction}\label{sec:introduction}
%% - Motivating examples
Consider the following scenario:
\begin{quote}
  John and Mary are working from home one morning when they get a craving for a
  slice of cake. Being denizens of the web, they quickly find the nearest store
  which does home deliveries.
  Unfortunately for them, they both order their cake at the \emph{same} store,
  which has only one slice left. After that, all it can deliver is
  disappointment.
\end{quote}
This is an example of a race condition. We can model this scenario in the
\textpi-calculus, assuming \john, \mary and \store are three processes
modeling John, Mary and the store, and \sliceofcake and \nope are two channels
giving access to a slice of cake and disappointment, respectively.
As expected, this process has two possible outcomes: either John gets the cake,
and Mary gets disappointment, or vice versa.
\[
  \begin{array}{c}
    \tm{(\piPar{%
    \piSend{x}{\sliceofcake}{\piSend{x}{\nope}{\store}}
    }{%
    \piPar{\piRecv{x}{y}{\john}}{\piRecv{x}{z}{\mary}}
    })}
    \\[1ex]
    \rotatebox[origin=c]{270}{$\Longrightarrow^{\star}$}
    \\[1ex]
    \tm{(\piPar{\store}{\piPar{\piSub{\sliceofcake}{y}{\john}}{\piSub{\nope}{z}{\mary}}})}
    \quad
    \text{or}
    \quad
    \tm{(\piPar{\store}{\piPar{\piSub{\nope}{y}{\john}}{\piSub{\sliceofcake}{z}{\mary}}})}
  \end{array}
\]
While John or Mary may not like all of the outcomes, it is the store which is
responsible for implementing the online delivery service, and the store is happy
with either outcome. Thus, the above is program we would like to be able to
write.
\\[\baselineskip]\noindent
Now consider another scenario, which takes place \emph{after} John has already
bought the cake:
\begin{quote}
  Mary is \emph{really} disappointed when she finds out the cake has sold out.
  John, always looking to make some money, offers to sell the slice to her for a
  profit. Mary agrees to engage in a little bit of back-alley cake resale, but
  sadly there is no trust between the two.
  John demands payment first.
  Mary would rather get her slice of cake before she gives John the money.
\end{quote}
This is an example of a deadlock. We can also model this scenario in the
\textpi-calculus, assuming that \bill\ is a channel giving access to some
adequate amount of money.
\[
  \begin{array}{c}
    \tm{(\piPar{%
    \piRecv{x}{z}{\piSend{y}{\sliceofcake}{\john}}
    }{%
    \piRecv{y}{w}{\piSend{x}{\bill}{\mary}}
    })}
    \quad
    \centernot\Longrightarrow^{\star}
  \end{array}  
\]
The above process does not reduce. As both John and Mary would prefer the
exchange to be made, this program is desired by \emph{neither}. Thus, the above
is a program we would \emph{somehow} like to exclude.

%% - Overview:
Session types~\cite{honda1993} can provide a static guarantee that concurrent
programs, such as those above, respect communication protocols.
Session-typed calculi with logical foundations, such as
\piDILL~\cite{caires2010} and CP~\cite{wadler2012}, obtain deadlock freedom as a
result of a close correspondence with logic.
The same correspondence, however, also rules out non-determinism and race
conditions.

We present \nodcap (nodcap), an extension of CP~\cite{wadler2012} with
a novel account of non-determinism and races.  Inspired by bounded linear
logic~\cite{girard1992}, we introduce a form of shared channels, in which the
type of a shared channel tracks how many times it is reused.  As in the untyped
$\pi$-calculus, sharing introduces the potential of non-determinism.  We show
that our approach is sufficient to capture practical examples of races, such as
the web store, as well as other formal characterizations of non-determinism,
such as non-deterministic choice.  However, \nodcap does not lose the
metatheoretical benefits of CP: we show that it enjoys termination and
deadlock-freedom.

%% Background
\chapter{Background}\label{sec:background}
%% * Introduction
\section{Related work}
%% * - Work on πDILL and CP
\section{Classical Processes}
%% * Terms and types
\begin{definition}[Terms]\label{def:cp-terms}
  \[\!
    \begin{aligned}
      \tm{P}, \tm{Q}, \tm{R}
           :=& \; \tm{\cpLink{x}{y}}       &&\text{link}
      \\ \mid& \; \tm{\cpCut{x}{P}{Q}}     &&\text{parallel composition}
      \\ \mid& \; \tm{\cpSend{x}{y}{P}{Q}} &&\text{``output''}
      \\ \mid& \; \tm{\cpRecv{x}{y}{P}}    &&\text{``input''}
      \\ \mid& \; \tm{\cpHalt{x}}          &&\text{halt}
      \\ \mid& \; \tm{\cpWait{x}{P}}       &&\text{wait}
      \\ \mid& \; \tm{\cpInl{x}{P}}        &&\text{select left choice}
      \\ \mid& \; \tm{\cpInr{x}{P}}        &&\text{select right choice}
      \\ \mid& \; \tm{\cpCase{x}{P}{Q}}    &&\text{offer binary choice}
      \\ \mid& \; \tm{\cpAbsurd{x}}        &&\text{offer nullary choice}
    \end{aligned}
  \]  
\end{definition}
%%% Local Variables:
%%% TeX-master: "main"
%%% End:

\begin{definition}[Types]\label{def:cp-types}
  \[\!
    \begin{aligned}
      \ty{A}, \ty{B}, \ty{C}
           :=& \; \ty{A \tens B} &&\text{pair of independent processes}
      \\ \mid& \; \ty{A \parr B} &&\text{pair of interdependent processes}
      \\ \mid& \; \ty{\one}      &&\text{unit for} \; {\tens}
      \\ \mid& \; \ty{\bot}      &&\text{unit for} \; {\parr}
      \\ \mid& \; \ty{A \plus B} &&\text{internal choice}
      \\ \mid& \; \ty{A \with B} &&\text{external choice}
      \\ \mid& \; \ty{\nil}      &&\text{unit for} \; {\plus}
      \\ \mid& \; \ty{\top}      &&\text{unit for} \; {\with}
    \end{aligned}
  \]  
\end{definition}
%%% Local Variables:
%%% TeX-master: "main"
%%% End:

\begin{definition}[Duality]\label{def:cp-negation}
  \[\!
    \begin{array}{lclclcl}
              \ty{(A \tens B)^\bot} &=& \ty{A^\bot \parr B^\bot}
      &\quad& \ty{\one^\bot}        &=& \ty{\bot}
      \\      \ty{(A \parr B)^\bot} &=& \ty{A^\bot \tens B^\bot}
      &\quad& \ty{\bot^\bot}        &=& \ty{\one}
      \\      \ty{(A \plus B)^\bot} &=& \ty{A^\bot \with B^\bot}
      &\quad& \ty{\nil^\bot}        &=& \ty{\top}
      \\      \ty{(A \with B)^\bot} &=& \ty{A^\bot \plus B^\bot}
      &\quad& \ty{\top^\bot}        &=& \ty{\nil}
    \end{array}
  \]
\end{definition}
%%% Local Variables:
%%% TeX-master: "main"
%%% End:

\begin{lemma}[Involutive]\label{thm:cp-negation-involutive}
  We have $\ty{A^{\bot\bot}} = \ty{A}$.
\end{lemma}
  \begin{proof}
    By induction on the structure of the type $\ty{A}$.
  \end{proof}
%%% Local Variables:
%%% TeX-master: "main"
%%% End:

%% * Reduction rules
\begin{definition}[Structural congruence]\label{def:cp-equiv}
  We define the structural congruence $\equiv$ as reflexivity, transitivity, and
  congruence over terms, plus the following two axioms:
  \[
    \begin{array}{llll}
      \cpEquivCutComm
      & \tm{\cpCut{x}{P}{Q}}
      & \equiv \;
      & \tm{\cpCut{x}{Q}{P}}
      \\
      \cpEquivCutAss1
      & \tm{\cpCut{x}{P}{\cpCut{y}{Q}{R}}}
      & \equiv \;
      & \tm{\cpCut{y}{\cpCut{x}{P}{Q}}{R}}
        \quad \text{if} \; \notFreeIn{x}{R} \; \text{and} \; \notFreeIn{y}{P}
    \end{array}
  \]
\end{definition}
%%% Local Variables:
%%% TeX-master: "main"
%%% End:

\begin{definition}[Term reduction]\label{def:cp-reduction}
  \[
    \begin{array}{llll}
      \cpRedAxCut1
      & \tm{\cpCut{x}{\cpLink{w}{x}}{P}}
      & \Longrightarrow \;
      & \tm{\cpSub{w}{x}{P}} 
      \\
      \cpRedAxCut2
      & \tm{\cpCut{x}{\cpLink{x}{w}}{P}}
      & \Longrightarrow \;
      & \tm{\cpSub{w}{x}{P}} 
      \\
      \\
      \cpRedBetaTensParr
      & \tm{\cpCut{x}{\cpSend{x}{y}{P}{Q}}{\cpRecv{x}{z}{R}}}
      & \Longrightarrow \;
      & \tm{\cpCut{y}{P}{\cpCut{x}{Q}{\cpSub{y}{z}{R}}}}
      \\
      \cpRedBetaOneBot
      & \tm{\cpCut{x}{\cpHalt{x}}{\cpWait{x}{P}}}
      & \Longrightarrow \;
      & \tm{P}
      \\
      \cpRedBetaPlusWith1
      & \tm{\cpCut{x}{\cpInl{x}{P}}{\cpCase{x}{Q}{R}}}
      & \Longrightarrow \;
      & \tm{\cpCut{x}{P}{Q}}
      \\
      \cpRedBetaPlusWith2
      & \tm{\cpCut{x}{\cpInr{x}{P}}{\cpCase{x}{Q}{R}}}
      & \Longrightarrow \;
      & \tm{\cpCut{x}{P}{R}}
      \\
      \\
      \cpRedKappaTens1
      & \tm{\cpCut{x}{\cpSend{y}{z}{P}{Q}}{R}}
      & \Longrightarrow \;
      & \tm{\cpSend{y}{z}{\cpCut{x}{P}{R}}{Q}} \quad \text{if} \; \notFreeIn{x}{Q}
      \\
      \cpRedKappaTens2
      & \tm{\cpCut{x}{\cpSend{y}{z}{P}{Q}}{R}}
      & \Longrightarrow \;
      & \tm{\cpSend{y}{z}{P}{\cpCut{x}{Q}{R}}} \quad \text{if} \; \notFreeIn{x}{P}
      \\
      \cpRedKappaParr
      & \tm{\cpCut{x}{\cpRecv{y}{z}{P}}{R}}
      & \Longrightarrow \;
      & \tm{\cpRecv{y}{z}{\cpCut{x}{P}{R}}}
      \\
      \cpRedKappaBot
      & \tm{\cpCut{x}{\cpWait{y}{P}}{R}}
      & \Longrightarrow \;
      & \tm{\cpWait{y}{\cpCut{x}{P}{R}}}
      \\
      \cpRedKappaPlus1
      & \tm{\cpCut{x}{\cpInl{y}{P}}{R}}
      & \Longrightarrow \;
      & \tm{\cpInl{y}{\cpCut{x}{P}{R}}}
      \\
      \cpRedKappaPlus2
      &\tm{\cpCut{x}{\cpInr{y}{P}}{R}}
      & \Longrightarrow \;
      & \tm{\cpInr{y}{\cpCut{x}{P}{R}}}
      \\
      \cpRedKappaWith
      & \tm{\cpCut{x}{\cpCase{y}{P}{Q}}{R}}
      & \Longrightarrow \;
      & \tm{\cpCase{y}{\cpCut{x}{P}{R}}{\cpCut{x}{Q}{R}}}
      \\
      \cpRedKappaTop
      & \tm{\cpCut{x}{\cpAbsurd{y}}{R}}
      & \Longrightarrow \;
      & \tm{\cpAbsurd{y}}
    \end{array}
  \]

  \begin{center}
    \begin{prooftree*}
      \AXC{\reducesto{P}{P^\prime}}
      \SYM{\cpRedGammaCut}
      \UIC{\reducesto{\cpCut{x}{P}{Q}}{\cpCut{x}{P^\prime}{Q}}}
    \end{prooftree*}
    \begin{prooftree*}
      \AXC{$\tm{P}\equiv\tm{Q}$}
      \AXC{\reducesto{Q}{Q^\prime}}
      \AXC{$\tm{Q^\prime}\equiv\tm{P^\prime}$}
      \SYM{\cpRedGammaEquiv}
      \TIC{\reducesto{P}{P^\prime}}
    \end{prooftree*}
  \end{center}
\end{definition}
%%% Local Variables:
%%% TeX-master: "main"
%%% End:

\begin{lemmaB}[\cpEquivCutAssNoParen2]\label{thm:cp-cut-assoc2}
  If $\tm{x}\not\in\tm{R}$ and $\tm{y}\not\in\tm{P}$, then 
  \(
    \tm{\cpCut{y}{\cpCut{x}{P}{Q}}{R}} \equiv
    \tm{\cpCut{x}{P}{\cpCut{y}{Q}{R}}}
  \).
\end{lemmaB}
  \begin{proof}
    \begin{align*}
      \tm{\cpCut{y}{\cpCut{x}{P}{Q}}{R}} &\equiv \qquad \text{by \cpEquivCutComm} \\
      \tm{\cpCut{y}{\cpCut{x}{Q}{P}}{R}} &\equiv \qquad \text{by \cpEquivCutComm} \\
      \tm{\cpCut{y}{R}{\cpCut{x}{Q}{P}}} &\equiv \qquad \text{by \cpEquivCutAss1} \\
      \tm{\cpCut{x}{\cpCut{y}{R}{Q}}{P}} &\equiv \qquad \text{by \cpEquivCutComm} \\
      \tm{\cpCut{x}{P}{\cpCut{y}{R}{Q}}} &\equiv \qquad \text{by \cpEquivCutComm} \\
      \tm{\cpCut{x}{P}{\cpCut{y}{Q}{R}}}
    \end{align*}
    The side conditions for \cpEquivCutAss1 are given.
  \end{proof}
%%% Local Variables:
%%% TeX-master: "main"
%%% End:

%% TODO:
%%     insert a note stating that we will leave uses of congruence or gamma
%%     rules implicit 
\begin{theorem}[Symmetry]\label{thm:cp-symmetry}
  If $\tm{P} \equiv \tm{Q}$, then $\tm{Q} \equiv \tm{P}$.
\end{theorem}
  \begin{proof}
    By induction on the structure of the equivalence proof.
    The only interesting case is \cpEquivCutAss1, which follows from
    \cref{thm:cp-cut-assoc2}.
  \end{proof}
%%% Local Variables:
%%% TeX-master: "main"
%%% End:

%% * Typing judgement
\begin{definition}[Environments]\label{def:cp-environments}
  We define environments as follows:
  \[
    \ty{\Gamma}, \ty{\Delta}, \ty{\Theta}
    ::= \tmty{x_1}{A_1}\dots\tmty{x_n}{A_n}
  \] 
  Names in environments must be unique, and environments \ty{\Gamma} and
  \ty{\Delta} can only be combined as $\ty{\Gamma}, \ty{\Delta}$ if
  $\text{fv}(\ty{\Gamma}) \cap \text{fv}(\ty{\Delta}) = \varnothing$. 
\end{definition}
%%% Local Variables:
%%% TeX-master: "main"
%%% End:

\begin{figure*}[b]
  \begin{center}
    \cpInfAx
    \cpInfCut
  \end{center}
  \begin{center}
    \cpInfTens
    \cpInfParr
  \end{center}
  \begin{center}
    \cpInfOne
    \cpInfBot
  \end{center}
  \begin{center}
    \cpInfPlus1
    \cpInfPlus2
  \end{center}
  \begin{center}
    \cpInfWith
  \end{center}
  \begin{center}
    \cpInfNil
    \cpInfTop
  \end{center}
  \caption{Typing judgement for the multiplicative applicative subset of \rcp.}
  \label{fig:cp-typing-judgement}
\end{figure*}
%%% Local Variables:
%%% TeX-master: "main"
%%% End:

%% * Preservation 
\begin{lemma}[Preservation for $\equiv$]\label{thm:cp-preservation-equiv}
  If $\tm{P} \equiv \tm{Q}$ and $\seq[{ P }]{ \Gamma }$, then $\seq[{ Q }]{ \Gamma }$.
\end{lemma}
\begin{proof}
  By induction on the structure of the equivalence. The cases for reflexivity,
  transitivity and congruence are trivial. The two interesting cases, for
  \cpEquivCutComm and \cpEquivCutAss1 are given in \cref{fig:cp-preservation-equiv}
\end{proof}
%%% Local Variables:
%%% TeX-master: "main"
%%% End:

\begin{figure*}[htb]
  \centering
  \begin{tabular}{ll}
    \cpEquivLinkComm
    &
      \begin{prooftree*}
        \AXC{}
        \NOM{Ax}
        \UIC{$\seq[{ \cpLink{x}{y} }]{ \tmty{x}{A}, \tmty{y}{A^\bot} }$}
      \end{prooftree*}
    \\[20pt]
    $\equiv$
    &
      \begin{prooftree*}
        \AXC{}
        \NOM{Ax}
        \UIC{$\seq[{ \cpLink{y}{x} }]{ \tmty{y}{A^\bot}, \tmty{x}{A^{\bot\bot}} }$}
        \RightLabel{$(\cdot)^\bot$-Involutive}
        \UIC{$\seq[{ \cpLink{y}{x} }]{ \tmty{y}{A^\bot}, \tmty{x}{A} }$}
      \end{prooftree*}
    \\[25pt]
    \cpEquivCutComm
    &
      \begin{prooftree*}
        \AXC{$\seq[{ P }]{ \Gamma, \tmty{x}{A} }$}
        \AXC{$\seq[{ Q }]{ \Delta, \tmty{x}{A^\bot} }$}
        \NOM{Cut}
        \BIC{$\seq[{ \cpCut{x}{P}{Q} }]{ \Gamma, \Delta }$}
      \end{prooftree*}
    \\[20pt]
    $\equiv$
    &
      \begin{prooftree*}
        \AXC{$\seq[{ Q }]{ \Delta, \tmty{x}{A^\bot} }$}
        \AXC{$\seq[{ P }]{ \Gamma, \tmty{x}{A} }$}
        \RightLabel{$(\cdot)^\bot$-Involutive}
        \UIC{$\seq[{ P }]{ \Gamma, \tmty{x}{A^{\bot\bot}} }$}
        \NOM{Cut}
        \BIC{$\seq[{ \cpCut{x}{Q}{P} }]{ \Gamma, \Delta }$}
      \end{prooftree*}
    \\[25pt]
    \cpEquivCutAss1
    &
      \begin{prooftree*}
        \AXC{$\seq[{ P }]{ \Gamma, \tmty{x}{A} }$}
        \AXC{$\seq[{ Q }]{ \Delta, \tmty{x}{A^\bot}, \tmty{y}{B} }$}
        \AXC{$\seq[{ R }]{ \Theta, \tmty{y}{B^\bot} }$}
        \NOM{Cut}
        \BIC{$\seq[{ \cpCut{y}{Q}{R} }]{ \Delta, \Theta, \tmty{x}{A^\bot} }$}
        \NOM{Cut}
        \BIC{$\seq[{ \cpCut{x}{P}{\cpCut{y}{Q}{R}} }]{ \Gamma, \Delta, \Theta }$}
      \end{prooftree*}
    \\[20pt]
    $\equiv$
    &
      \begin{prooftree*}
        \AXC{$\seq[{ P }]{ \Gamma, \tmty{x}{A} }$}
        \AXC{$\seq[{ Q }]{ \Delta, \tmty{x}{A^\bot}, \tmty{y}{B} }$}
        \NOM{Cut}
        \BIC{$\seq[{ \cpCut{x}{P}{Q} }]{ \Gamma, \Delta, \tmty{y}{B} }$}
        \AXC{$\seq[{ R }]{ \Theta, \tmty{y}{B^\bot} }$}
        \NOM{Cut}
        \BIC{$\seq[{ \cpCut{y}{\cpCut{x}{P}{Q}}{R} }]{ \Gamma, \Delta, \Theta }$}
      \end{prooftree*}
  \end{tabular}
  \caption{Type preservation for the structural congruence of \cp}
  \label{fig:cp-preservation-equiv}
\end{figure*}
%%% Local Variables:
%%% TeX-master: "main"
%%% End:

\begin{theorem}[Preservation]\label{thm:cp-preservation}
  If \reducesto{P}{Q} and $\seq[{ P }]{ \Gamma }$, then $\seq[{ Q }]{ \Gamma }$.
\end{theorem}
\begin{proof}
  By induction on the structure of the reduction. See
  \cref{fig:cp-preservation-1} for the \cpRedAxCut{i} and \textbeta-reduction
  rules, and \cref{fig:cp-preservation-2a,fig:cp-preservation-2b} for the
  commutative conversions.
  %
  The case for \cpRedGammaCut is trivial by the induction hypothesis, and the
  case for \cpRedGammaEquiv is trivial by the induction hypothesis and
  \cref{thm:cp-preservation-equiv}. 
\end{proof}
%%% Local Variables:
%%% TeX-master: "main"
%%% End:

\input{fig-cp-preservation-1}
\begin{figure*}[ht]
  \centering
  \begin{tabular}{ll}
    \cpRedKappaTens1
    &
      \begin{prooftree*}
        \AXC{$\seq[{ P }]{ \Gamma, \tmty{x}{A}, \tmty{z}{B} }$}
        \AXC{$\seq[{ Q }]{ \Delta, \tmty{y}{C} }$}
        \SYM{\tens}
        \BIC{$\seq[{ \cpSend{y}{z}{P}{Q} }]{ \Gamma, \Delta, \tmty{y}{B \tens C} }$}
        \AXC{$\seq[{ R }]{ \Theta, \tmty{x}{A^\bot} }$}
        \NOM{Cut}
        \BIC{$\seq[{ \cpCut{x}{\cpSend{y}{z}{P}{Q}}{R} }]{ \Gamma, \Delta, \Theta, \tmty{y}{B \tens C} }$}
      \end{prooftree*}
    \\[30pt]
    $\Longrightarrow$
    &
       \begin{prooftree*}
         \AXC{$\seq[{ P }]{ \Gamma, \tmty{x}{A}, \tmty{z}{B} }$}
         \AXC{$\seq[{ R }]{ \Theta, \tmty{x}{A^\bot} }$}
         \NOM{Cut}
         \BIC{$\seq[{ \cpCut{x}{P}{Q} }]{ \Gamma, \Theta, \tmty{z}{B} }$}
         \AXC{$\seq[{ Q }]{ \Delta, \tmty{y}{C} }$}
         \SYM{\tens}
         \BIC{$\seq[{ \cpSend{y}{z}{\cpCut{x}{P}{Q}}{R} }]{ \Gamma, \Delta, \Theta, \tmty{y}{B \tens C} }$}
       \end{prooftree*}
    \\[30pt]
    \cpRedKappaTens2
    &
      (as above)
    \\[20pt]
    \cpRedKappaParr
    &
      \begin{prooftree*}
        \AXC{$\seq[{ P }]{ \Gamma, \tmty{x}{A}, \tmty{z}{B}, \tmty{y}{C} }$}
        \SYM{\parr}
        \UIC{$\seq[{ \cpRecv{y}{z}{P} }]{ \Gamma, \tmty{x}{A}, \tmty{y}{B \parr C} }$}
        \AXC{$\seq[{ R }]{ \Theta, \tmty{x}{A^\bot} }$}
        \NOM{Cut} 
        \BIC{$\seq[{ \cpCut{x}{\cpRecv{y}{z}{P}}{R} }]{ \Gamma, \Theta, \tmty{y}{B \parr C} }$}
      \end{prooftree*}
    \\[30pt]
    $\Longrightarrow$
    &
      \begin{prooftree*}
        \AXC{$\seq[{ P }]{ \Gamma, \tmty{x}{A}, \tmty{z}{B}, \tmty{y}{C} }$}
        \AXC{$\seq[{ R }]{ \Theta, \tmty{x}{A^\bot} }$}
        \NOM{Cut}
        \BIC{$\seq[{ \cpCut{x}{P}{R} }]{ \Gamma, \Theta, \tmty{z}{B}, \tmty{y}{C} }$}
        \SYM{\parr}
        \UIC{$\seq[{ \cpRecv{y}{z}{\cpCut{x}{P}{R}} }]{ \Gamma, \Theta, \tmty{y}{B \parr C} }$}
      \end{prooftree*}
    \\[40pt]
    \cpRedKappaBot
    &
      \begin{prooftree*}
        \AXC{$\seq[{ P }]{ \Gamma, \tmty{x}{A} }$}
        \SYM{\bot}
        \UIC{$\seq[{ \cpWait{y}{P} }]{ \Gamma, \tmty{x}{A}, \tmty{y}{\bot} }$}
        \AXC{$\seq[{ R }]{ \Theta, \tmty{x}{A^\bot} }$}
        \NOM{Cut} 
        \BIC{$\seq[{ \cpCut{x}{\cpWait{y}{P}}{R} }]{ \Gamma, \Theta, \tmty{y}{\bot} }$}
      \end{prooftree*}
    \\[30pt]
    $\Longrightarrow$
    &
      \begin{prooftree*}
        \AXC{$\seq[{ P }]{ \Gamma, \tmty{x}{A} }$}
        \AXC{$\seq[{ R }]{ \Theta, \tmty{x}{A^\bot} }$}
        \NOM{Cut} 
        \BIC{$\seq[{ \cpCut{x}{P}{R} }]{ \Gamma, \Theta }$}
        \SYM{\bot}
        \UIC{$\seq[{ \cpWait{y}{\cpCut{x}{P}{R}} }]{ \Gamma, \Theta, \tmty{y}{\bot} }$}
      \end{prooftree*}
  \end{tabular}

  \caption{Type preservation for the commuting conversions of \cp}
  \label{fig:cp-preservation-2a}
\end{figure*}
\begin{figure*}[ht]
  \makebox[\textwidth][c]{
    \begin{tabular}{ll}
      \cpRedKappaPlus1
      &
        \begin{prooftree*}
          \AXC{$\seq[{ P }]{ \Gamma, \tmty{x}{A}, \tmty{y}{B} }$}
          \SYM{\plus_1}
          \UIC{$\seq[{ \cpInl{y}{P} }]{ \Gamma, \tmty{x}{A}, \tmty{y}{B \plus C} }$}
          \AXC{$\seq[{ R }]{ \Theta, \tmty{x}{A^\bot} }$}
          \NOM{Cut}
          \BIC{$\seq[{ \cpCut{x}{\cpInl{y}{P}}{R} }]{ \Gamma, \Theta, \tmty{y}{B \plus C} }$}
        \end{prooftree*}
      \\[30pt]
      $\Longrightarrow$
      &
        \begin{prooftree*}
          \AXC{$\seq[{ P }]{ \Gamma, \tmty{x}{A}, \tmty{y}{B} }$}
          \AXC{$\seq[{ R }]{ \Theta, \tmty{x}{A^\bot} }$}
          \NOM{Cut}
          \BIC{$\seq[{ \cpCut{x}{P}{R} }]{ \Gamma, \Theta, \tmty{y}{B} }$}
          \SYM{\plus_1}
          \UIC{$\seq[{ \cpInl{y}{\cpCut{x}{P}{R}} }]{ \Gamma, \Theta, \tmty{y}{B \plus C} }$}
        \end{prooftree*}
      \\[30pt] 
      \cpRedKappaPlus2
      &
        (as above)
      \\[20pt]
      \cpRedKappaWith
      &
        \begin{prooftree*}
          \AXC{$\seq[{ P }]{ \Gamma, \tmty{x}{A}, \tmty{y}{B} }$}
          \AXC{$\seq[{ Q }]{ \Gamma, \tmty{x}{A}, \tmty{y}{C} }$}
          \SYM{\with}
          \BIC{$\seq[{ \cpCase{y}{P}{Q} }]{ \Gamma, \tmty{x}{A}, \tmty{y}{B \with C} }$}
          \AXC{$\seq[{ R }]{ \Theta, \tmty{x}{A^\bot} }$}
          \NOM{Cut}
          \BIC{$\seq[{ \cpCut{x}{\cpCase{y}{P}{Q}}{R} }]{ \Gamma, \Theta, \tmty{y}{B \with C} }$}
        \end{prooftree*}
      \\[30pt]
      $\Longrightarrow$
      &
        \begin{prooftree*}
          \AXC{$\seq[{ P }]{ \Gamma, \tmty{x}{A}, \tmty{y}{B} }$}
          \AXC{$\seq[{ R }]{ \Theta, \tmty{x}{A^\bot} }$}
          \NOM{Cut}
          \BIC{$\seq[{ \cpCut{x}{P}{R} }]{ \Gamma, \Theta, \tmty{y}{B} }$}
          \AXC{$\seq[{ Q }]{ \Gamma, \tmty{x}{A}, \tmty{y}{C} }$}
          \AXC{$\seq[{ R }]{ \Theta, \tmty{x}{A^\bot} }$}
          \NOM{Cut}
          \BIC{$\seq[{ \cpCut{x}{Q}{R} }]{ \Gamma, \Theta, \tmty{y}{C} }$}
          \SYM{\with}
          \BIC{$\seq[{ \cpCase{y}{\cpCut{x}{P}{R}}{\cpCut{x}{Q}{R}} }]{ \Gamma, \Theta, \tmty{y}{B \with C} }$}
        \end{prooftree*}
      \\[40pt] 
      \cpRedKappaTop
      &
        \begin{prooftree*}
          \AXC{}
          \SYM{\top}
          \UIC{$\seq[{ \cpAbsurd{y} }]{ \Gamma, \tmty{x}{A}, \tmty{y}{\top} }$}
          \AXC{$\seq[{ R }]{ \Theta, \tmty{x}{A^\bot} }$}
          \NOM{Cut}
          \BIC{$\seq[{ \cpCut{x}{\cpAbsurd{y}}{R} }]{ \Gamma, \Theta, \tmty{y}{\top} }$}
        \end{prooftree*}
      \\[30pt]
      $\Longrightarrow$
      &
        \begin{prooftree*}
          \AXC{}
          \SYM{\top}
          \UIC{$\seq[{ \cpAbsurd{y} }]{ \Gamma, \Theta, \tmty{y}{\top} }$}
        \end{prooftree*}
    \end{tabular}
  }
  \caption{Type preservation for the commuting conversions of \cp (cont'd)}
  \label{fig:cp-preservation-2b}
\end{figure*}
%%% Local Variables:
%%% TeX-master: "main"
%%% End:
%% * Progress
\begin{definition}[Action]\label{def:cp-action}
  A process $\tm{P}$ \emph{acts on} a channel $\tm{x}$ if it is of the form
  $\tm{\cpSend{x}{y}{P'}{Q'}}$, $\tm{\cpRecv{x}{y}{P'}}$, $\tm{\cpHalt{x}}$,
  $\tm{\cpWait{x}{P'}}$, $\tm{\cpInl{x}{P'}}$, $\tm{\cpInr{x}{P'}}$,
  $\tm{\cpCase{x}{P'}{Q'}}$ or $\tm{\cpAbsurd{x}}$. 
\end{definition}
%%% Local Variables:
%%% TeX-master: "main"
%%% End:

\begin{definition}[Canonical forms]\label{def:cp-canonical-forms}
  A process \tm{P} is in canonical form if it is in one of the following forms:
  \begin{multicols}{3}
    \begin{itemize}[noitemsep,topsep=0pt,parsep=0pt,partopsep=0pt]
    \item \tm{\cpLink{x}{y}}
    \item \tm{\cpSend{x}{y}{P'}{Q'}}
    \item \tm{\cpRecv{x}{y}{P'}}
    \item \tm{\cpHalt{x}}
    \item \tm{\cpWait{x}{P'}}
    \item \tm{\cpInl{x}{P'}}
    \item \tm{\cpInr{x}{P'}}
    \item \tm{\cpCase{x}{P'}{Q'}}
    \item \tm{\cpAbsurd{x}}
    \end{itemize}
  \end{multicols}
\end{definition}
%%% Local Variables:
%%% TeX-master: "main"
%%% End:

\begin{definition}[Evaluation contexts]\label{def:cp-evaluation-contexts}
  We define evaluation contexts as:
  \begin{align*}
    \tm{G}, \tm{H} := \tm{\Box}
    \mid \tm{\cpCut{x}{G}{P}}
    \mid \tm{\cpCut{x}{P}{G}}
  \end{align*}
\end{definition}
\begin{definition}[Plugging]\label{def:cp-evaluation-context-plugging}
  We define plugging for evaluation contexts as:
  \begin{gather*}
    \begin{array}{ll}
      \tm{\cpPlug{\Box}{R}}            
      & := \; \tm{R}
      \\
      \tm{\cpPlug{\cpCut{x}{G}{P}}{R}}
      & := \; \tm{\cpCut{x}{\cpPlug{G}{R}}{P}}
      \\
      \tm{\cpPlug{\cpCut{x}{P}{G}}{R}}
      & := \; \tm{\cpCut{x}{P}{\cpPlug{G}{R}}}
    \end{array}
  \end{gather*}
\end{definition}
%%% Local Variables:
%%% TeX-master: "main"
%%% End:

\begin{definition}[Plugging]\label{def:cp-evaluation-contexts-plugging}
  We define plugging for evaluation contexts as:
  \begin{gather*}
    \begin{array}{ll}
      \tm{\cpPlug{\Box}{R}}            
      & := \; \tm{R}
      \\
      \tm{\cpPlug{\cpCut{x}{G}{P}}{R}}
      & := \; \tm{\cpCut{x}{\cpPlug{G}{R}}{P}}
      \\
      \tm{\cpPlug{\cpCut{x}{P}{G}}{R}}
      & := \; \tm{\cpCut{x}{P}{\cpPlug{G}{R}}}
    \end{array}
  \end{gather*}
\end{definition}
%%% Local Variables:
%%% TeX-master: "main"
%%% End:

\input{thm-cp-display}
\begin{theorem}[Progress]\label{thm:cp-progress}
  If $\seq[{ P }]{ \Gamma }$, then $\tm{P}$ is in canonical form, or there
  exists a $\tm{P'}$ s.t.\ $\reducesto{P}{P'}$. 
\end{theorem}
\begin{proof}
  By induction on the structure of derivation for $\seq[{ P }]{ \Gamma }$.
  The only interesting case is when the last rule of the derivation is
  \textsc{Cut}. In every other case, the typing rule constructs a term in which
  is in canonical form. 
  \\
  If the last rule in the derivation is \textsc{Cut}, we consider the prefix of
  the derivation for $\seq[{ P }]{ \Gamma}$ which consists of all top-level
  cuts. A prefix of $n$ cuts introduces $n$ variables, but composes $n+1$
  actions. Therefore, one of the following must be true: 
  \begin{itemize}
  \item
    One of these actions is a link \tm{\cpLink{x}{y}}. At least one of \tm{x}
    and \tm{y} is a bound name. Let us assume \tm{x} is bound. We have:
    \begin{gather*}
      \begin{array}{ll}
        \tm{P}
        & = \quad \text{see above}
        \\
        \tm{\cpPlug{G}{\cpCut{x}{\cpPlug{H}{\cpLink{x}{y}}}{Q}}}
        & \equiv \quad \text{by \cref{thm:cp-display}}
          \hphantom{\; \text{and} \; \cref{thm:cp-preservation-equiv}}
        \\
        \tm{\cpPlug{G}{\cpPlug{H}{\cpCut{x}{\cpLink{x}{y}}{Q}}}}
      \end{array}
    \end{gather*}
    Similarly if \tm{y} is bound. We then apply one of \cpRedAxCut1 or
    \cpRedAxCut2.
  \item
    Two of these actions, on different sides of a \textsc{Cut}, act on the same
    channel. 
    \\
    Let us name these processes \tm{P_i} and \tm{P_j}, and their shared channel
    \tm{y}. We have:
    \begin{gather*}
      \begin{array}{ll}
        \tm{P}
        & = \quad \text{see above}
        \\
        \tm{\cpPlug{G}{\cpCut{y}{\cpPlug{H_i}{P_i}}{\cpPlug{H_j}{P_j}}}}
        & \equiv \quad \text{by} \; \cref{thm:cp-display}
        \\
        \tm{\cpPlug{G}{\cpPlug{H_i}{\cpCut{y}{P_i}{\cpPlug{H_j}{P_j}}}}}
        & \equiv \quad \text{by} \; \cref{thm:cp-display} \; \text{and} \;
          \cref{thm:cp-preservation-equiv}
        \\
        \tm{\cpPlug{G}{\cpPlug{H_i}{\cpPlug{H_j}{\cpCut{y}{P_i}{P_j}}}}} 
      \end{array}
    \end{gather*}
    We then apply one of the \textbeta-reduction rules.
  \item
    Otherwise (at least) one of the actions acts on a free variable.
    \\
    We then apply one of the commutative conversions.
  \end{itemize}
\end{proof}
%%% Local Variables:
%%% TeX-master: "main"
%%% End:

\begin{theorem}[Termination]\label{thm:cp-termination}
  If $\seq[{ P }]{ \Gamma }$, then there are no infinite $\Longrightarrow$
  reduction sequences.
\end{theorem}
  \begin{proof}
    Every reduction reduces a single cut to zero, one or two cuts.
    However, each of these cuts is \emph{smaller}, in the sense that the type of
    the channel on which the communication takes place is smaller, as each
    reduction eliminates a connective---see
    \cref{fig:cp-preservation-1,fig:cp-preservation-2a,fig:cp-preservation-2b}.
    Therefore, there cannot be an infinite reduction sequence.
  \end{proof}
%%% Local Variables:
%%% TeX-master: "main"
%%% End:


\chapter{Non-deterministic Classical Processes}\label{sec:main}
%% Main contribution
%% * Terms and types
\begin{definition}[Terms]\label{def:nc-terms}
  \[\!
    \begin{aligned}
      \tm{P}, \tm{Q}, \tm{R}
           :=& \; \dots
      \\ \mid& \; \tm{\ncCnt{x}{y}{P}} &&\text{create client}
      \\ \mid& \; \tm{\ncSrv{x}{y}{P}} &&\text{create server interaction}
      \\ \mid& \; \tm{\ncPool{P}{Q}}   &&\text{parallel composition of clients}
    \end{aligned}
  \]
\end{definition}
%%% Local Variables:
%%% TeX-master: "main"
%%% End:

\begin{definition}[Types]\label{def:nc-types}
  We extend \cref{def:cp-types} with the following types:
  \[\!
    \begin{aligned}
      \ty{A}, \ty{B}, \ty{C}
           :=& \; \dots
      \\ \mid& \; \ty{\take[n]{A}} &&\text{pool of} \; n \; \text{clients}
      \\ \mid& \; \ty{\give[n]{A}} &&n \; \text{server interactions}
    \end{aligned}
  \]  
\end{definition}
%%% Local Variables:
%%% TeX-master: "main"
%%% End:

\begin{definition}[Negation]\label{def:nc-negation}
  We extend \cref{def:cp-negation} with the following cases:
  \[\!
    \begin{array}{lclclcl}
              \ty{(\take[n]{A})^\bot} &=& \ty{\give[n]{A^\bot}}
      &\quad& \ty{(\give[n]{A})^\bot} &=& \ty{\take[n]{A^\bot}}
    \end{array}
  \]
\end{definition}
%%% Local Variables:
%%% TeX-master: "main"
%%% End:

\begin{lemma}[Involutive]\label{thm:nc-negation-involutive}
  We have $\ty{A^{\bot\bot}} = \ty{A}$.
\end{lemma}
\begin{proof}
  By induction on the structure of the type $\ty{A}$.
\end{proof}
%%% Local Variables:
%%% TeX-master: "main"
%%% End:

%% * Reduction rules
\begin{definition}[Structural congruence]\label{def:nc-equiv}
  We extend \cref{def:cp-equiv} with the following equivalences:
  \[
    \begin{array}{llll}
%     \ncEquivPoolId
%     & \tm{\ncPool{P}{\ncHalt}}
%     & \equiv \;
%     & \tm{P}
%     \\
      \ncEquivPoolComm
      & \tm{\ncPool{P}{Q}}
      & \equiv \;
      & \tm{\ncPool{Q}{P}}
      \\
      \ncEquivPoolAss1
      & \tm{\ncPool{P}{\ncPool{Q}{R}}}
      & \equiv \;
      & \tm{\ncPool{\ncPool{P}{Q}}{R}}
    \end{array}
  \]
\end{definition} 
%%% Local Variables:
%%% TeX-master: "main"
%%% End:

\begin{definition}[Term reduction]\label{def:nc-reduction}
  We extend \cref{def:cp-reduction} with the following reductions:
  \[
    \begin{array}{llll}
      \ncRedBetaStar{1}
      & \tm{\cpCut{x}{\ncCnt{x}{y}{P}}{\ncSrv{x}{z}{R}}}
      & \Longrightarrow \;
      & \tm{\cpCut{y}{P}{\cpSub{y}{z}{R}}}
      \\
      \ncRedBetaStar{n+1}
      & \tm{\cpCut{x}{\ncPool{\ncCnt{x}{y}{P}}{Q}}{\ncSrv{x}{z}{R}}}
      & \Longrightarrow \;
      & \tm{\cpCut{x}{Q}{\cpCut{y}{P}{\cpSub{y}{z}{R}}}}
      \\
      \\
      \ncRedKappaTake
      & \tm{\cpCut{x}{\ncCnt{y}{z}{P}}{R}}
      & \Longrightarrow \;
      & \tm{\ncCnt{y}{z}{\cpCut{x}{P}{R}}}
      \\
      \ncRedKappaGive
      & \tm{\cpCut{x}{\ncSrv{y}{z}{P}}{R}}
      & \Longrightarrow \;
      & \tm{\ncSrv{y}{z}{\cpCut{x}{P}{R}}}
    \end{array}
  \]
  \begin{prooftree}
    \AXC{\reducesto{P}{P'}}
    \SYM{\ncRedGammaPool}
    \UIC{\reducesto{\ncPool{P}{Q}}{\ncPool{P'}{Q}}}
  \end{prooftree}
\end{definition}
%%% Local Variables:
%%% TeX-master: "main"
%%% End:

\begin{lemma}[\ncEquivPoolAssNoParen2]\label{thm:nc-pool-assoc2}
  We have
  \[
    \tm{\ncPool{\ncPool{P}{Q}}{R}} \equiv
    \tm{\ncPool{P}{\ncPool{Q}{R}}}
  \]
\end{lemma}
\begin{proof}
  \begin{align*}
    \tm{\ncPool{\ncPool{P}{Q}}{R}} &\equiv \qquad \text{by \ncEquivPoolComm} \\
    \tm{\ncPool{\ncPool{Q}{P}}{R}} &\equiv \qquad \text{by \ncEquivPoolComm} \\
    \tm{\ncPool{R}{\ncPool{Q}{P}}} &\equiv \qquad \text{by \ncEquivPoolAss1} \\
    \tm{\ncPool{\ncPool{R}{Q}}{P}} &\equiv \qquad \text{by \ncEquivPoolComm} \\
    \tm{\ncPool{P}{\ncPool{R}{Q}}} &\equiv \qquad \text{by \ncEquivPoolComm} \\
    \tm{\ncPool{P}{\ncPool{Q}{R}}}
  \end{align*}
\end{proof}
%%% Local Variables:
%%% TeX-master: "main"
%%% End:

\begin{theorem}[Symmetry]\label{thm:nc-symmetry}
  If $\tm{P} \equiv \tm{Q}$, then $\tm{Q} \equiv \tm{P}$.
\end{theorem}
\begin{proof}
  By induction on the structure of the equivalence proof.
\end{proof}
%%% Local Variables:
%%% TeX-master: "main"
%%% End:

%% * Typing judgement
\input{def-nc-environments}
\begin{figure*}[b]
  \centering
  \begin{center}
    \ncInfTake1
    \ncInfGive1
  \end{center}
  \begin{center}
    \ncInfPool
  \end{center}
  \begin{center}
    \ncInfCont
  \end{center}
  \caption{Typing judgement for \nodcap extending that of Figure~\ref{fig:cp-typing-judgement}}
  \label{fig:nc-typing-judgement}
\end{figure*}
%%% Local Variables:
%%% TeX-master: "main"
%%% End:

%% * Cut elimination
\begin{theorem}[Preservation for $\equiv$]\label{thm:nc-preservation-equiv}
  If $\seq[{ P }]{ \Gamma }$ and $\tm{P} \equiv \tm{Q}$,
  then $\seq[{ Q }]{ \Gamma }$.
\end{theorem}
\begin{proof}
  By induction on the structure of the equivalence. The cases for reflexivity,
  transitivity and congruence are trivial. The cases for \cpEquivCutComm and
  \cpEquivCutAss1 are given in \cref{fig:cp-preservation-equiv}.
  The cases for \ncEquivPoolComm and \ncEquivPoolAss1 are given in
  \cref{fig:nc-preservation-equiv}.
\end{proof}
%%% Local Variables:
%%% TeX-master: "main"
%%% End:

\begin{figure*}[ht]
  \centering
  \begin{tabular}{ll}
    \ncEquivPoolComm
    &
      \begin{prooftree*}
        \AXC{$\seq[{ P }]{ \Gamma, \tmty{x}{\take[m]{A}} }$}
        \AXC{$\seq[{ Q }]{ \Delta, \tmty{x}{\take[n]{A}} }$}
        \NOM{Pool}
        \BIC{$\seq[{ \ncPool{P}{Q} }]{ \Gamma, \Delta, \take[m+n]{A} }$}
      \end{prooftree*}
    \\[30pt]
    $\equiv$
    &
      \begin{prooftree*}
        \AXC{$\seq[{ Q }]{ \Delta, \tmty{x}{\take[n]{A}} }$}
        \AXC{$\seq[{ P }]{ \Gamma, \tmty{x}{\take[m]{A}} }$}
        \NOM{Pool}
        \BIC{$\seq[{ \ncPool{Q}{P} }]{ \Gamma, \Delta, \tmty{x}{\take[m+n]{A}} }$}
      \end{prooftree*}
    \\[40pt]
    \ncEquivPoolAss1
    &
      \begin{prooftree*}
        \AXC{$\seq[{ P }]{ \Gamma, \tmty{x}{\take[l]{A}} }$}
        \AXC{$\seq[{ Q }]{ \Delta, \tmty{x}{\take[m]{A}} }$}
        \AXC{$\seq[{ R }]{ \Theta, \tmty{x}{\take[n]{A}} }$}
        \NOM{Pool}
        \BIC{$\seq[{ \ncPool{Q}{R} }]{ \Delta, \Theta, \tmty{x}{\take[m+n]{A}} }$}
        \NOM{Pool}
        \BIC{$\seq[{ \ncPool{P}{\ncPool{Q}{R}} }]{ \Gamma, \Delta, \Theta, \tmty{x}{\take[l+m+n]{A}} }$}
      \end{prooftree*}
    \\[30pt]
    $\equiv$
    &
      \begin{prooftree*}
        \AXC{$\seq[{ P }]{ \Gamma, \tmty{x}{\take[l]{A}} }$}
        \AXC{$\seq[{ Q }]{ \Delta, \tmty{x}{\take[m]{A}} }$}
        \NOM{Pool}
        \BIC{$\seq[{ \ncPool{P}{Q} }]{ \Gamma, \Delta, \tmty{x}{\take[l+m]{A}} }$}
        \AXC{$\seq[{ R }]{ \Theta, \tmty{x}{\take[n]{A}} }$}
        \NOM{Pool}
        \BIC{$\seq[{ \ncPool{\ncPool{P}{Q}}{R} }]{ \Gamma, \Delta, \Theta, \tmty{x}{\take[l+m+n]{A}} }$}
      \end{prooftree*}
    \\[20pt]
    or
    \\[30pt]
    \ncEquivPoolAss1
    &
      \begin{prooftree*}
        \AXC{$\seq[{ P }]{ \Gamma, \tmty{x}{\take[k]{A}} }$}
        \AXC{$\seq[{ Q }]{ \Delta, \tmty{x}{\take[l]{A}}, \tmty{y}{\take[m]{B}} }$}
        \AXC{$\seq[{ R }]{ \Theta, \tmty{x}{\take[n]{B}} }$}
        \NOM{Pool}
        \BIC{$\seq[{ \ncPool{Q}{R} }]{ \Delta, \Theta, \tmty{x}{\take[l]{A}}, \tmty{y}{\take[m+n]{B}} }$}
        \NOM{Pool}
        \BIC{$\seq[{ \ncPool{P}{\ncPool{Q}{R}} }]{ \Gamma, \Delta, \Theta, \tmty{x}{\take[k+l]{A}}, \tmty{y}{\take[m+n]{B}} }$}
      \end{prooftree*}
    \\[30pt]
    $\equiv$
    &
      \begin{prooftree*}
        \AXC{$\seq[{ P }]{ \Gamma, \tmty{x}{\take[k]{A}} }$}
        \AXC{$\seq[{ Q }]{ \Delta, \tmty{x}{\take[l]{A}}, \tmty{y}{\take[m]{B}} }$}
        \NOM{Pool}
        \BIC{$\seq[{ \ncPool{Q}{R} }]{ \Gamma, \Delta, \tmty{x}{\take[k+l]{A}}, \tmty{y}{\take[m]{B}} }$}
        \AXC{$\seq[{ R }]{ \Theta, \tmty{x}{\take[o]{B}} }$}
        \NOM{Pool}
        \BIC{$\seq[{ \ncPool{P}{\ncPool{Q}{R}} }]{ \Gamma, \Delta, \Theta, \tmty{x}{\take[k+l]{A}}, \tmty{y}{\take[m+n]{B}} }$}
      \end{prooftree*}
  \end{tabular}
  \caption{Type preservation for the structural congruence of \nc}
  \label{fig:nc-preservation-equiv}
\end{figure*}
%%% Local Variables:
%%% TeX-master: "main"
%%% End:

\begin{theorem}[Preservation]\label{thm:nc-preservation}
  If \reducesto{P}{Q} and $\seq[{ P }]{ \Gamma }$, then $\seq[{ Q }]{ \Gamma }$.
\end{theorem}
\begin{proof}
  By induction on the structure of the reduction. See
  \cref{fig:cp-preservation-1} for \cpRedAxCut{i} and the \textbeta-reduction
  rules from \cp, and \cref{fig:cp-preservation-2a,fig:cp-preservation-2b} for the
  commutative conversions from \cp.
  See \cref{fig:nc-preservation-1,fig:nc-preservation-2a,fig:nc-preservation-2b}
  for the \textbeta-reduction rules and commutative conversions from \nodcap.
  The cases for \cpRedGammaCut and \ncRedGammaPool are trivial by call to the
  induction hypothesis, and the case for \cpRedGammaEquiv is trivial by call to
  the induction hypothesis and \cref{thm:nc-preservation-equiv}.
\end{proof}
%%% Local Variables:
%%% TeX-master: "main"
%%% End:

\begin{figure*}[ht]
  \makebox[\textwidth][c]{
    \begin{tabular}{ll}
      \ncRedBetaStar1
      &
        \begin{prooftree*}
          \AXC{$\seq[{ P }]{ \Gamma, \tmty{y}{A} }$}
          \SYM{\take[1]{}}
          \UIC{$\seq[{ \ncCnt{x}{y}{P} }]{ \Gamma, \tmty{x}{\take[1]{A}} }$}
          \AXC{$\seq[{ Q }]{ \Delta, \tmty{z}{A^\bot} }$}
          \SYM{\give[1]{}}
          \UIC{$\seq[{ \ncSrv{x}{z}{Q} }]{ \Delta, \tmty{x}{\give[1]{A}} }$}
          \NOM{Cut}
          \BIC{$\seq[{ \cpCut{x}{\ncCnt{x}{y}{P}}{\ncSrv{x}{z}{Q}} }]{ \Gamma, \Delta }$}
        \end{prooftree*}
      \\[30pt]
      $\Longrightarrow$
      &
        \begin{prooftree*}
          \AXC{$\seq[{ P }]{ \Gamma, \tmty{y}{A} }$}
          \AXC{$\seq[{ Q }]{ \Delta, \tmty{z}{A^\bot} }$}
          \NOM{Cut}
          \BIC{$\seq[{ \cpCut{y}{P}{\cpSub{y}{z}{Q}} }]{ \Gamma, \Delta }$}
        \end{prooftree*}
      \\[40pt]
      \ncRedBetaStar{n+1}
      &
        \begin{prooftree*}
          \AXC{$\seq[{ P }]{ \Gamma, \tmty{y}{A} }$}
          \SYM{\take[1]{}}
          \UIC{$\seq[{ \ncCnt{x}{y}{P} }]{ \Gamma, \tmty{x}{\take[1]{A}} }$}
          \AXC{$\seq[{ Q }]{ \Delta, \tmty{x}{\take[n]{A}} }$}
          \NOM{Pool}
          \BIC{$\seq[{ \ncPool{\ncCnt{x}{y}{P}}{Q} }]{ \Gamma, \Delta, \tmty{x}{\take[n+1]{A}} }$}
          \AXC{$\seq[{ R }]{ \Theta, \tmty{z}{A^\bot}, \tmty{x}{\give[n]{A}} }$}
          \SYM{\give[1]{}}
          \UIC{$\seq[{ \ncSrv{x'}{z}{Q} }]{ \Theta, \tmty{x'}{\give[1]{A}}, \tmty{x}{\give[n]{A}} }$}
          \NOM{Cont}
          \UIC{$\seq[{ \ncSrv{x}{z}{Q} }]{ \Theta, \tmty{x}{\give[n+1]{A}} }$}
          \NOM{Cut}
          \BIC{$\seq[{ \cpCut{x}{\ncPool{\ncCnt{x}{y}{P}}{Q}}{\ncSrv{x}{z}{Q}} }]{ \Gamma, \Delta, \Theta }$}
        \end{prooftree*}
      \\[40pt]
      $\Longrightarrow$
      &
        \begin{prooftree*}
          \AXC{$\seq[{ Q }]{ \Delta, \tmty{x}{\take[n]{A}} }$}
          \AXC{$\seq[{ P }]{ \Gamma, \tmty{y}{A} }$}
          \AXC{$\seq[{ R }]{ \Theta, \tmty{z}{A^\bot}, \tmty{x}{\give[n]{A}} }$}
          \NOM{Cut}
          \BIC{$\seq[{ \cpCut{y}{P}{\cpSub{y}{z}{R}} }]{ \Gamma, \Theta, \tmty{x}{\give[n]{A}} }$}
          \NOM{Cut}
          \BIC{$\seq[{ \cpCut{x}{Q}{\cpCut{y}{P}{\cpSub{y}{z}{R}}} }]{ \Gamma, \Delta, \Theta }$}
        \end{prooftree*}
      \\[40pt]
      \ncRedKappaTake
      &
        \begin{prooftree*}
          \AXC{$\seq[{ P }]{ \Gamma, \tmty{x}{A}, \tmty{z}{B} }$}
          \SYM{\take[1]{}}
          \UIC{$\seq[{ \ncCnt{y}{z}{P} }]{ \Gamma, \tmty{x}{A}, \tmty{y}{\take[1]{B}} }$}
          \AXC{$\seq[{ R }]{ \Delta, \tmty{x}{A^\bot} }$}
          \NOM{Cut}
          \BIC{$\seq[{ \cpCut{x}{\ncCnt{y}{z}{P}}{R} }]{ \Gamma, \Delta, \tmty{y}{\take[1]{B}} }$}
        \end{prooftree*}
      \\[30pt]
      $\Longrightarrow$
      &
        \begin{prooftree*}
          \AXC{$\seq[{ P }]{ \Gamma, \tmty{x}{A}, \tmty{z}{B} }$}
          \AXC{$\seq[{ R }]{ \Delta, \tmty{x}{A^\bot} }$}
          \NOM{Cut}
          \BIC{$\seq[{ \cpCut{x}{P}{R} }]{ \Gamma, \Delta, \tmty{z}{B} }$}
          \SYM{\take[1]{}}
          \UIC{$\seq[{ \ncCnt{y}{z}{\cpCut{x}{P}{R}} }]{ \Gamma, \Delta, \tmty{z}{\take[1]{B}} }$}
        \end{prooftree*}
      \\[30pt]
      \ncRedKappaGive
      &
        (as above)
    \end{tabular}
  }   

  \caption{Type preservation for the \textbeta-reduction rules and commutative conversions of \nodcap}
  \label{fig:nc-preservation-1}
\end{figure*}
%%% Local Variables:
%%% TeX-master: "main"
%%% End: 
\begin{figure*}[ht]
  \begin{tabular}{ll}
    \ncRedKappaTake
    &
      \begin{prooftree*}
        \AXC{$\seq[{ P }]{ \Gamma, \tmty{x}{A}, \tmty{z}{B} }$}
        \SYM{\take[1]{}}
        \UIC{$\seq[{ \ncCnt{y}{z}{P} }]{ \Gamma, \tmty{x}{A}, \tmty{y}{\take[1]{B}} }$}
        \AXC{$\seq[{ R }]{ \Delta, \tmty{x}{A^\bot} }$}
        \NOM{Cut}
        \BIC{$\seq[{ \cpCut{x}{\ncCnt{y}{z}{P}}{R} }]{ \Gamma, \Delta, \tmty{y}{\take[1]{B}} }$}
      \end{prooftree*}
    \\[30pt]
    $\Longrightarrow$
    &
      \begin{prooftree*}
        \AXC{$\seq[{ P }]{ \Gamma, \tmty{x}{A}, \tmty{z}{B} }$}
        \AXC{$\seq[{ R }]{ \Delta, \tmty{x}{A^\bot} }$}
        \NOM{Cut}
        \BIC{$\seq[{ \cpCut{x}{P}{R} }]{ \Gamma, \Delta, \tmty{z}{B} }$}
        \SYM{\take[1]{}}
        \UIC{$\seq[{ \ncCnt{y}{z}{\cpCut{x}{P}{R}} }]{ \Gamma, \Delta, \tmty{z}{\take[1]{B}} }$}
      \end{prooftree*}
    \\[30pt]
    \ncRedKappaGive
    &
      (as above)
    \ncRedKappaPoolTens1
    &
      \begin{prooftree*}
        \AXC{$\seq[{ P }]{\Gamma, \tmty{x}{\take[m]{A}}, \tmty{z}{B}}$}
        \AXC{$\seq[{ Q }]{\Delta, \tmty{y}{C}}$}
        \SYM{\tens}
        \BIC{$\seq[{ \cpSend{y}{z}{P}{Q} }]%
          {\Gamma, \Delta, \tmty{x}{A}, \tmty{y}{B \tens C}}$}
        \AXC{$\seq[{ R }]{\Theta, \tmty{x}{\take[n]{A}}}$}
        \NOM{Pool}
        \BIC{$\seq[{ \ncPool{\cpSend{y}{z}{P}{Q}}{R} }]%
          {\Gamma, \Delta, \Theta, \tmty{x}{\take[m+n]{A}}, \tmty{y}{B \tens C}}$}
      \end{prooftree*}
    \\[30pt]
    $\Longrightarrow$
    &
      \begin{prooftree*}
        \AXC{$\seq[{ P }]{\Gamma, \tmty{x}{\take[m]{A}}, \tmty{z}{B}}$}
        \AXC{$\seq[{ R }]{\Theta, \tmty{x}{\take[n]{A}}}$}
        \NOM{Pool}
        \BIC{$\seq[{ \ncPool{P}{R} }]%
          {\Gamma, \Theta, \tmty{x}{\take[m+n]{A}}, \tmty{z}{B}}$}
        \AXC{$\seq[{ Q }]{\Delta, \tmty{y}{C}}$}
        \SYM{\tens}
        \BIC{$\seq[{ \cpSend{y}{z}{\ncPool{P}{R}}{R} }]%
          {\Gamma, \Delta, \Theta, \tmty{x}{\take[m+n]{A}}, \tmty{y}{B \tens C}}$}
      \end{prooftree*}
    \\[30pt]
    \ncRedKappaPoolTens2
    &
      (as above)
    \\[20pt]
    \ncRedKappaPoolParr
    &
      \begin{prooftree*}
        \AXC{$\seq[{ P }]{\Gamma, \tmty{x}{\take[m]{A}}, \tmty{z}{B}, \tmty{y}{C}}$}
        \SYM{\parr}
        \UIC{$\seq[{ \cpRecv{y}{z}{P} }]%
          {\Gamma, \tmty{x}{\take[m]{A}}, \tmty{z}{B \parr C}}$}
        \AXC{$\seq[{ R }]{\Theta, \tmty{x}{\take[n]{A}}}$}
        \NOM{Pool}
        \BIC{$\seq[{ \ncPool{\cpRecv{y}{z}{P}}{R} }]%
          {\Gamma, \Theta, \tmty{x}{\take[m+n]{A}}, \tmty{y}{B \parr C}}$}
      \end{prooftree*}
    \\[30pt]
    $\Longrightarrow$
    &
      \begin{prooftree*}
        \AXC{$\seq[{ P }]{\Gamma, \tmty{x}{\take[m]{A}}, \tmty{z}{B}, \tmty{y}{C}}$}
        \AXC{$\seq[{ R }]{\Theta, \tmty{x}{\take[n]{A}}}$}
        \NOM{Pool}
        \BIC{$\seq[{ \ncPool{P}{R} }]%
          {\Gamma, \Theta, \tmty{x}{\take[m+n]{A}}, \tmty{z}{B}, \tmty{y}{C}}$}
        \SYM{\parr}
        \UIC{$\seq[{ \cpRecv{y}{z}{\ncPool{P}{R}} }]%
          {\Gamma, \Theta, \tmty{x}{\take[m+n]{A}}, \tmty{z}{B \parr C}}$}
      \end{prooftree*}
    \\[40pt]
    \ncRedKappaPoolBot
    &
      \begin{prooftree*}
        \AXC{$\seq[{ P }]{\Gamma, \tmty{x}{\take[m]{A}}}$}
        \SYM{\bot}
        \UIC{$\seq[{ \cpWait{y}{P} }]{\Gamma, \tmty{x}{\take[m]{A}, \tmty{y}{\bot}}}$}
        \AXC{$\seq[{ R }]{\Theta, \tmty{x}{\take[n]{A}}}$} 
        \NOM{Pool}
        \BIC{$\seq[{ \ncPool{\cpWait{y}{P}}{R} }]{\Gamma, \Theta, \tmty{x}{\take[m+n]{A}}, \tmty{y}{\bot}}$}
      \end{prooftree*}
    \\[30pt]
    $\Longrightarrow$
    &
      \begin{prooftree*}
        \AXC{$\seq[{ P }]{\Gamma, \tmty{x}{\take[m]{A}}}$}
        \AXC{$\seq[{ R }]{\Theta, \tmty{x}{\take[n]{A}}}$} 
        \NOM{Pool}
        \BIC{$\seq[{ \ncPool{P}{R} }]{\Gamma, \Theta, \tmty{x}{\take[m+n]{A}}}$}
        \SYM{\bot}
        \UIC{$\seq[{ \cpWait{y}{\ncPool{P}{R}} }]{\Gamma, \Theta, \tmty{x}{\take[m+n]{A}}, \tmty{y}{\bot}}$}
      \end{prooftree*}
  \end{tabular}

  \caption{Type preservation for the commuting conversions with the pooling
    rules of \nodcap}
  \label{fig:nc-preservation-2a}
\end{figure*}
%
\begin{figure*}[ht]
  \makebox[\textwidth][c]{
    \begin{tabular}{ll}
      \ncRedKappaPoolPlus1
      &
        \begin{prooftree*}
          \AXC{$\seq[{ P }]{\Gamma, \tmty{x}{\take[m]{A}}, \tmty{y}{B}}$}
          \SYM{\plus_1}
          \UIC{$\seq[{ \cpInl{y}{P} }]{\Gamma, \tmty{x}{\take[m]{A}}, \tmty{y}{B \plus C}}$}
          \AXC{$\seq[{ R }]{\Theta, \tmty{x}{\take[n]{A}}}$}
          \NOM{Pool}
          \BIC{$\seq[{ \ncPool{\cpInl{x}{P}}{R} }]%
            {\Gamma, \Theta, \tmty{x}{\take[m+n]{A}, \tmty{y}{B \plus C}}}$}
        \end{prooftree*}
      \\[30pt]
      $\Longrightarrow$
      &
        \begin{prooftree*}
          \AXC{$\seq[{ P }]{\Gamma, \tmty{x}{\take[m]{A}}, \tmty{y}{B}}$}
          \AXC{$\seq[{ R }]{\Theta, \tmty{x}{\take[n]{A}}}$}
          \NOM{Pool}
          \BIC{$\seq[{ \ncPool{P}{R} }]{\Gamma, \Theta, \tmty{x}{\take[m+n]{A}}, \tmty{y}{B}}$}
          \SYM{\plus_1}
          \UIC{$\seq[{ \cpInl{x}{\ncPool{P}{R}} }]%
            {\Gamma, \Theta, \tmty{x}{\take[m+n]{A}, \tmty{y}{B \plus C}}}$}
        \end{prooftree*}
      \\[30pt]
      \ncRedKappaPoolPlus2
      &
        (as above)
      \\[20pt]
      \ncRedKappaPoolWith
      &
        \begin{prooftree*}
          \AXC{$\seq[{ P }]{\Gamma, \tmty{x}{\take[m]{A}}, \tmty{y}{B}}$}
          \AXC{$\seq[{ Q }]{\Gamma, \tmty{x}{\take[m]{A}}, \tmty{y}{C}}$}
          \SYM{\with}
          \BIC{$\seq[{ \cpCase{y}{P}{Q} }]{\Gamma, \tmty{x}{\take[m]{A}}, \tmty{y}{B \with C}}$}
          \AXC{$\seq[{ R }]{\Theta, \tmty{x}{\take[n]{A}}}$}
          \NOM{Pool}
          \BIC{$\seq[{ \ncPool{\cpCase{y}{P}{Q}}{R} }]{\Gamma, \Theta, \tmty{x}{\take[m+n]{A}}, \tmty{y}{B \with C}}$}
        \end{prooftree*}
      \\[30pt]
      $\Longrightarrow$
      &
        \begin{prooftree*}
          \AXC{$\seq[{ P }]{\Gamma, \tmty{x}{\take[m]{A}}, \tmty{y}{B}}$}
          \AXC{$\seq[{ R }]{\Theta, \tmty{x}{\take[n]{A}}}$}
          %\NOM{Pool}
          \BIC{$\seq[{ \ncPool{P}{R} }]%
            {\Gamma, \Theta, \tmty{x}{\take[m+n]{A}}, \tmty{y}{B}}$}
          \AXC{$\seq[{ Q }]{\Gamma, \tmty{x}{\take[m]{A}}, \tmty{y}{C}}$}
          \AXC{$\seq[{ R }]{\Theta, \tmty{x}{\take[n]{A}}}$}
          \NOM{Pool}
          \BIC{$\seq[{ \ncPool{Q}{R} }]%
            {\Gamma, \Theta, \tmty{x}{\take[m+n]{A}}, \tmty{y}{C}}$}
          \SYM{\with}
          \BIC{$\seq[{ \cpCase{y}{\ncPool{P}{R}}{\ncPool{Q}{R}} }]%
            {\Gamma, \Theta, \tmty{x}{\take[m+n]{A}}, \tmty{y}{B \with C}}$}
        \end{prooftree*}
      \\[40pt]
      \ncRedKappaPoolTop
      &
        \begin{prooftree*}
          \AXC{}
          \SYM{\top}
          \UIC{$\seq[{ \cpAbsurd{y} }]%
            {\Gamma, \tmty{x}{\take[m]{A}}, \tmty{y}{\top}}$}
          \AXC{$\seq[{ R }]{\Theta, \tmty{x}{\take[n]{A}}}$}
          \NOM{Pool}
          \BIC{$\seq[{ \ncPool{\cpAbsurd{y}}{R} }]%
            {\Gamma, \Theta, \tmty{x}{\take[n]{A}}}$}
        \end{prooftree*}
      \\[30pt]
      $\Longrightarrow$
      &
        \begin{prooftree*}
          \AXC{}
          \SYM{\top}
          \UIC{$\seq[{ \cpAbsurd{y} }]%
            {\Gamma, \Theta, \tmty{x}{\take[m+n]{A}}, \tmty{y}{\top}}$}
        \end{prooftree*}
      \\[40pt]
      \ncRedKappaPoolTake
      &
        \begin{prooftree*}
          \AXC{$\seq[{ P }]{\Gamma, \tmty{x}{\take[m]{A}}, \tmty{z}{B}}$}
          \SYM{\take[1]{}}
          \UIC{$\seq[{ \ncCnt{y}{z}{P} }]%
            {\Gamma, \tmty{x}{\take[m]{A}}, \tmty{y}{\take[1]{B}}}$}
          \AXC{$\seq[{ R }]{\Theta, \tmty{x}{\take[n]{A}}}$}
          \NOM{Pool}
          \BIC{$\seq[{ \ncPool{\ncCnt{y}{z}{P}}{R} }]%
            {\Gamma, \Theta, \tmty{x}{\take[m+n]{A}}, \tmty{y}{\take[1]{B}}}$}
        \end{prooftree*}
      \\[30pt]
      $\Longrightarrow$
      &
        \begin{prooftree*}
          \AXC{$\seq[{ P }]{\Gamma, \tmty{x}{\take[m]{A}}, \tmty{z}{B}}$}
          \AXC{$\seq[{ R }]{\Theta, \tmty{x}{\take[n]{A}}}$}
          \NOM{Pool}
          \BIC{$\seq[{ \ncPool{P}{R} }]%
            {\Gamma, \Theta, \tmty{x}{\take[m+n]{A}}, \tmty{z}{B}}$}
          \SYM{\take[1]{}}
          \UIC{$\seq[{ \ncCnt{y}{z}{\ncPool{P}{R}} }]%
            {\Gamma, \Theta, \tmty{x}{\take[m+n]{A}}, \tmty{y}{B}}$}
        \end{prooftree*}
      \\[30pt]
      \ncRedKappaPoolGive
      &
        (as above)
    \end{tabular}
  }
  
  \caption{Type preservation for the commuting conversions with the pooling
    rules of \nodcap (cont'd)}
  \label{fig:nc-preservation-2b}
\end{figure*}
%%% Local Variables:
%%% TeX-master: "main"
%%% End: 

%% * Progress
\begin{definition}[Action]\label{def:nc-action}
  We extend \cref{def:cp-action} with the following cases:
  \begin{itemize}[noitemsep,topsep=0pt,parsep=0pt,partopsep=0pt]
  \item \tm{\ncCnt{x}{y}{P'}}
  \item \tm{\ncSrv{x}{y}{P'}}
  \end{itemize}
\end{definition}
%%% Local Variables:
%%% TeX-master: "main"
%%% End:

\begin{definition}[Canonical forms]\label{def:nc-canonical-forms}
  We extend \cref{def:cp-canonical-forms} with the following canonical forms:
  \begin{center}
    \begin{prooftree*}
      \AXC{\vphantom{\canonical{P}\canonical{Q}}}
      \UIC{\canonical{\ncCnt{x}{y}{P}}}
    \end{prooftree*}
    \begin{prooftree*}
      \AXC{\vphantom{\canonical{P}\canonical{Q}}}
      \UIC{\canonical{\ncSrv{x}{y}{P}}}
    \end{prooftree*}
    \begin{prooftree*}
      \AXC{\canonical{P}}
      \AXC{\canonical{Q}}
      \BIC{\canonical{\ncPool{P}{Q}}}
    \end{prooftree*}
  \end{center}
\end{definition}
%%% Local Variables:
%%% TeX-master: "main"
%%% End:

\begin{definition}[Evaluation contexts]\label{def:nc-evaluation-contexts}
  We extend \cref{def:cp-evaluation-contexts} with the following constructs:
  \begin{align*}
    \tm{G}, \tm{H} := \dots \mid \tm{\ncPool{G}{P}} \mid \tm{\ncPool{P}{G}}
  \end{align*}
\end{definition}
%%% Local Variables:
%%% TeX-master: "main"
%%% End:

\begin{definition}[Plugging]\label{def:nc-evaluation-contexts-plugging}
  We extend \cref{def:cp-evaluation-contexts-plugging} with the following cases:
  \begin{gather*}
    \begin{array}{ll}
      \tm{\cpPlug{\ncPool{G}{P}}{R}}
      & := \; \tm{\ncPool{\cpPlug{G}{R}}{P}}
      \\
      \tm{\cpPlug{\ncPool{P}{G}}{R}}
      & := \; \tm{\ncPool{P}{\cpPlug{G}{R}}}
    \end{array}
  \end{gather*}
\end{definition}
%%% Local Variables:
%%% TeX-master: "main"
%%% End:

\begin{lemma}\label{thm:nc-display-1}
  If $\seq[{ \tm{\cpCut{x}{\cpPlug{G}{P}}{Q}} }]{ \Gamma }$ and
  $\notFreeIn{x}{G}$, then $\tm{\cpCut{x}{\cpPlug{G}{P}}{Q}} \equiv
  \tm{\cpPlug{G}{\cpCut{x}{P}{Q}}}$.
\end{lemma}
\begin{proof}
  By induction on the structure of the evaluation context \tm{G}.
  \begin{itemize}
  \item
    Case $\tm{\Box}$, $\tm{\cpCut{y}{H}{R}}$, and $\tm{\cpCut{y}{R}{H}}$. See \cref{thm:cp-display}.
  \item
    Case $\tm{\ncPool{G}{R}}$.
    \[\!
      \begin{array}{ll}
        \tm{\cpCut{x}{\ncPool{\cpPlug{G}{P}}{R}}{Q}} & \equiv \quad \text{by} \; \ncEquivPoolComm \\
        \tm{\cpCut{x}{\ncPool{R}{\cpPlug{G}{P}}}{Q}} & \equiv \quad \text{by} \; \ncRedKappaPool1 \\
        \tm{\ncPool{R}{\cpCut{x}{\cpPlug{G}{P}}{Q}}} & \equiv \quad \text{by} \; \ncEquivPoolComm \\
        \tm{\ncPool{\cpCut{x}{\cpPlug{G}{P}}{Q}}{R}} & \equiv \quad \text{by the induction hypothesis}\\
        \tm{\ncPool{\cpPlug{G}{\cpCut{x}{P}{Q}}}{R}} &
      \end{array}
    \]
  \item
    Case $\tm{\ncPool{R}{G}}$.
    \[\!
      \begin{array}{ll}
        \tm{\cpCut{x}{\ncPool{R}{\cpPlug{G}{P}}}{Q}} & \equiv \quad \text{by} \; \ncRedKappaPool1 \\
        \tm{\ncPool{R}{\cpCut{x}{\cpPlug{G}{P}}{Q}}} & \equiv \quad \text{by the induction hypothesis}\\
        \tm{\ncPool{R}{\cpPlug{G}{\cpCut{x}{P}{Q}}}} &
      \end{array}
    \]
  \end{itemize}
  In each case, the side condition for \ncRedKappaPool1, $\notFreeIn{x}{R}$, can
  be inferred from $\notFreeIn{x}{G}$, and the side conditions for the induction
  hypothesis can be inferred from \cref{thm:nc-preservation-equiv} and
  $\notFreeIn{x}{G}$.
\end{proof}
%%% Local Variables:
%%% TeX-master: "main"
%%% End:
\input{thm-nc-display-2}
\input{thm-nc-display-3}
\begin{theorem}[Progress]\label{thm:nc-progress}
  If $\seq[{ P }]{ \Gamma }$, then $\tm{P}$ is in canonical form, or there
  exists a $\tm{P'}$ s.t.\ $\reducesto{P}{P'}$.
\end{theorem}
\begin{proof}
  By induction on the structure of derivation for $\seq[{ P }]{ \Gamma }$.
  There only interesting cases are when the last rule of the derivation is
  \textsc{Cut} or \textsc{Pool}. In every other case, the typing rule constructs
  a term in which is in canonical form. 
  \\
  If the last rule in the derivation is \textsc{Cut} or \textsc{Pool}, we
  consider the prefix of the derivation for $\seq[{ P }]{ \Gamma}$ which
  consists of all top-level cuts and pooling rules. A prefix of $n$ cuts and $m$
  pooling rules introduces $n$ variables, but composes $n+m+1$ actions, at most
  $m+1$ of which are on the same side of all cut rules.
  Therefore, one of the following must be true:
  \begin{itemize}
  \item
    One of these actions was introduced by an application of \textsc{Ax}.
    \\
    We proceed as in \cref{thm:cp-progress}. 
  \item
    Two of these actions, on different sides of a \textsc{Cut}, act on the same
    channel. Let us name these processes \tm{P_i} and \tm{P_j}, and their shared
    channel \tm{y}. We have
    $\tm{P} = \tm{\cpPlug{G}{\cpCut{y}{\cpPlug{H_i}{P_i}}{\cpPlug{H_j}{P_j}}}}$.
    We distinguish the following cases:
    \begin{itemize}
    \item
      We have either
      $\seq[{ \cpPlug{H_i}{P_i}}]{ \Delta, \tmty{y}{\take[n]{A}} }$ or
      $\seq[{ \cpPlug{H_j}{P_j} }]{\Delta, \tmty{y}{\take[n]{A}} }$.
      \\
      We rewrite by \cref{thm:nc-display-3}, then apply one of \ncRedBetaStar{1}
      and \ncRedBetaStar{n+1}. 
    \item
      Otherwise, we can infer $\notFreeIn{y}{H_i}$ and $\notFreeIn{y}{H_j}$.
      \\
      We proceed as in \cref{thm:cp-progress}. 
    \end{itemize}
  \item
    The process is in canonical form \tm{
      \ncPool{\ncCnt{x_1}{y_1}{P_1}}{
        \ncPool{\dots}{\ncCnt{x_n}{y_n}{P_n}}\dots}}. 
  \item 
    Otherwise (at least) one of the actions acts on a free variable.
    \\
    We apply one of the commutative conversions.
  \end{itemize}
\end{proof}
%%% Local Variables:
%%% TeX-master: "main"
%%% End:

\begin{theorem}[Termination]\label{thm:nc-termination}
  If $\seq[{ P }]{ \Gamma }$, then there are no infinite $\Longrightarrow$
  reduction sequences.
\end{theorem}
\begin{proof}
  Every reduction reduces a single cut to zero, one or two cuts.
  However, each of these cuts is \emph{smaller}, in the sense that the type of
  the channel on which the communication takes place is smaller.
  Each reduction either eliminates a connective, or decreases a resource index
  on the type of a shared channel.
  See
  \cref{fig:cp-preservation-1,fig:cp-preservation-2a,fig:cp-preservation-2b,fig:nc-preservation-1,fig:nc-preservation-2a,fig:nc-preservation-2b}.
  Therefore, there cannot be an infinite $\Longrightarrow$ reduction sequence.
\end{proof}
%%% Local Variables:
%%% TeX-master: "main"
%%% End:
%% * Equivalence with non-deterministic local choice

\chapter{Discussion}\label{sec:discussion}
%% Discussion
%% - Interaction with recursion
%% - Interaction with affine and relevant sessions

%% Bibliography
\printbibliography
\end{document}

%%% Local Variables:
%%% TeX-master: "main"
%%% End:
