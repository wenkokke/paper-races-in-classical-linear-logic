\documentclass[12pt,a4paper,UKenglish,mscres,logo,twoside,notimes,parskip,lfcs]{infthesis}
\shieldtype{3}
\usepackage{microtype}
\usepackage{alphabeta}
\usepackage[greek,english]{babel}
\languageattribute{greek}{polutoniko}
\usepackage[dvipsnames,table]{xcolor}
\colorlet{tm}{Black}
\colorlet{ty}{Black}

\usepackage{fullpage}
\usepackage{amsmath, amscd, amsthm, amssymb, mathrsfs,amsfonts}
\usepackage{textcomp}
\usepackage{txfonts}
\usepackage{graphicx}

%%% TERM CALCULUS
\newcommand{\link}[2]{#1{\leftrightarrow}#2}
\newcommand{\wait}[1]{#1().}
\newcommand{\halt}[1]{#1[].0}
\newcommand{\send}[2]{#1[#2].}
\newcommand{\recv}[2]{#1(#2).}
\newcommand{\case}[3]{%
\toks0={#2#3}%
\edef\param{\the\toks0}%
\ifx\param\empty
  \ensuremath{\text{case} \; #1 \; \{\}}
\else
  \ensuremath{\text{case} \; #1 \; \{ #2 ; #3 \}}
\fi}
\newcommand{\inl}[1]{#1[\text{inl}].}
\newcommand{\inr}[1]{#1[\text{inr}].}
\newcommand{\expn}[2]{#1 \uparrow #2.}
\newcommand{\intl}[2]{#1 \downarrow #2.}
\newcommand{\cut}[1]{\nu #1.}

\def\parr{\ensuremath{\mathbin{\rotatebox[origin=c]{180}{\&}}}}
\def\with{\ensuremath{\mathbin{\text{\&}}}}
\def\plus{\ensuremath{\oplus}}
\def\tens{\ensuremath{\otimes}}
\def\limp{\ensuremath{\multimap}}
\def\one{\ensuremath{\mathbf{1}}}
\def\nil{\ensuremath{\mathbf{0}}}
\def\lequiv{\ensuremath{\multimapboth}}

\DeclareGraphicsRule{.ai}{pdf}{.ai}{}
\newcommand{\emoji}[2][1em]{\ensuremath{\vcenter{\hbox{\includegraphics[width=#1]{twemoji/assets/#2.ai}}}}}
\newcommand{\mary}[1][1em]{\emoji[#1]{1f469}}
\newcommand{\john}[1][1em]{\emoji[#1]{1f466}}
\newcommand{\cake}[1][1em]{\emoji[#1]{1f382}}
\newcommand{\plato}[1][1em]{\emoji[#1]{1f381}}
\newcommand{\nocake}[1][1em]{\emoji[#1]{1f64c}}
\newcommand{\money}[1][1em]{\emoji[#1]{1f4b0}}
\newcommand{\dollar}[1][1em]{\emoji[#1]{1f4b5}}
\newcommand{\ptis}[1][1em]{\emoji[#1]{1f3ea}}
\newcommand{\good}[1][1em]{\emoji[#1]{2714}}
\newcommand{\bad}[1][1em]{\emoji[#1]{2716}}

\newcommand{\tm}[2][]{%
\toks0={#1}%
\edef\param{\the\toks0}%
\ifx\param\empty
  \ensuremath{{\color{ty}#2}}%
\else
  \ensuremath{{\color{tm}#1}\colon{\color{ty}#2}}%
\fi}
\newcommand{\seq}[2][]{\ensuremath{{\color{tm}#1}\vdash{\color{ty}#2}}}
\newcommand{\subst}[3]{\ensuremath{#1 \{ #2 / #3 \}}}
\newcommand{\nod}[2][]{\ensuremath{{\star}_{#1}{#2}}}
\newcommand{\give}[2][]{\ensuremath{{ ? }_{#1}{#2}}}
\newcommand{\take}[2][]{\ensuremath{{ ! }_{#1}{#2}}}

\usepackage{bussproofs}
\def\fCenter{\ensuremath{\;\vdash\;}}
\newcommand{\NOM}[1]{\RightLabel{\textsc{#1}}}
\newcommand{\SYM}[1]{\RightLabel{\ensuremath{#1}}}
\EnableBpAbbreviations
\newenvironment{proofbox}[1][0.9]%
  {\gdef\scalefactor{#1} \leavevmode\hbox\bgroup}
  {\scalebox{\scalefactor}{\DisplayProof} \egroup}
\newenvironment{proofblock}[1][0.9]%
  {\gdef\scalefactor{#1}\begin{center}\proofSkipAmount \leavevmode}%
  {\scalebox{\scalefactor}{\DisplayProof}\proofSkipAmount \end{center} }

%%% Local Variables:
%%% TeX-master: "main"
%%% End:
\title{Races in Classical Linear Logic}
\author{Wen Kokke}
\addbibresource{main.bib}
\begin{document}
\begin{preliminary}
  \abstract{%
    Process calculi based in logic, such as CP, provide a foundation for
    deadlock-free concurrent programming, but at the cost of excluding
    non-determinism and races.  We introduce \nodcap (nodcap), which extends CP
    with a novel account of non-determinism.  Our approach draws on bounded linear
    logic to provide a strongly-typed account of standard process calculus
    expressions of non-determinism.  We show that our extension is expressive
    enough to capture many uses of non-determinism in untyped calculi, such as
    non-deterministic choice, while preserving CP's meta-theoretic properties,
    including deadlock freedom.  We have formalized our calculus and its
    properties using Agda.
  }
  \maketitle
  \begin{acknowledgements}
  \end{acknowledgements}
  \standarddeclaration
  \tableofcontents
\end{preliminary}

\wen{Replace references to \cpRedAxCut1 and \cpRedAxCut2 with \emph{just} a
  reference to the first, and insert a note explaining how we deviate from \cp
  in this particular instance.}
%% Introduction
\chapter{Introduction}\label{sec:introduction}
%% - Motivating examples
Consider the following scenario:
\begin{quote}
  John and Mary are working from home one morning when they get a craving for a
  slice of cake. Being denizens of the web, they quickly find the nearest store
  which does home deliveries.
  Unfortunately for them, they both order their cake at the \emph{same} store,
  which has only one slice left. After that, all it can deliver is
  disappointment.
\end{quote}
This is an example of a race condition. We can model this scenario in the
\textpi-calculus, assuming \john, \mary and \store are three processes
modeling John, Mary and the store, and \sliceofcake and \nope are two channels
giving access to a slice of cake and disappointment, respectively.
As expected, this process has two possible outcomes: either John gets the cake,
and Mary gets disappointment, or vice versa.
\[
  \begin{array}{c}
    \tm{(\piPar{%
    \piSend{x}{\sliceofcake}{\piSend{x}{\nope}{\store}}
    }{%
    \piPar{\piRecv{x}{y}{\john}}{\piRecv{x}{z}{\mary}}
    })}
    \\[1ex]
    \rotatebox[origin=c]{270}{$\Longrightarrow^{\star}$}
    \\[1ex]
    \tm{(\piPar{\store}{\piPar{\piSub{\sliceofcake}{y}{\john}}{\piSub{\nope}{z}{\mary}}})}
    \quad
    \text{or}
    \quad
    \tm{(\piPar{\store}{\piPar{\piSub{\nope}{y}{\john}}{\piSub{\sliceofcake}{z}{\mary}}})}
  \end{array}
\]
While John or Mary may not like all of the outcomes, it is the store which is
responsible for implementing the online delivery service, and the store is happy
with either outcome. Thus, the above is program we would like to be able to
write.
\\[\baselineskip]\noindent
Now consider another scenario, which takes place \emph{after} John has already
bought the cake:
\begin{quote}
  Mary is \emph{really} disappointed when she finds out the cake has sold out.
  John, always looking to make some money, offers to sell the slice to her for a
  profit. Mary agrees to engage in a little bit of back-alley cake resale, but
  sadly there is no trust between the two.
  John demands payment first.
  Mary would rather get her slice of cake before she gives John the money.
\end{quote}
This is an example of a deadlock. We can also model this scenario in the
\textpi-calculus, assuming that \bill\ is a channel giving access to some
adequate amount of money.
\[
  \begin{array}{c}
    \tm{(\piPar{%
    \piRecv{x}{z}{\piSend{y}{\sliceofcake}{\john}}
    }{%
    \piRecv{y}{w}{\piSend{x}{\bill}{\mary}}
    })}
    \quad
    \centernot\Longrightarrow^{\star}
  \end{array}  
\]
The above process does not reduce. As both John and Mary would prefer the
exchange to be made, this program is desired by \emph{neither}. Thus, the above
is a program we would \emph{somehow} like to exclude.

%% - Overview:
Session types~\cite{honda1993} can provide a static guarantee that concurrent
programs, such as those above, respect communication protocols.
Session-typed calculi with logical foundations, such as
\piDILL~\cite{caires2010} and CP~\cite{wadler2012}, obtain deadlock freedom as a
result of a close correspondence with logic.
The same correspondence, however, also rules out non-determinism and race
conditions.

We present \nodcap (nodcap), an extension of CP~\cite{wadler2012} with
a novel account of non-determinism and races.  Inspired by bounded linear
logic~\cite{girard1992}, we introduce a form of shared channels, in which the
type of a shared channel tracks how many times it is reused.  As in the untyped
$\pi$-calculus, sharing introduces the potential of non-determinism.  We show
that our approach is sufficient to capture practical examples of races, such as
the web store, as well as other formal characterizations of non-determinism,
such as non-deterministic choice.  However, \nodcap does not lose the
metatheoretical benefits of CP: we show that it enjoys termination and
deadlock-freedom.
%%% Local Variables:
%%% TeX-master: "main"
%%% End:

%% Background
\chapter{Background}\label{sec:background}
\section{Classical Processes}\label{sec:cp}
In this section, we will discuss a rudimentary subset of the typed process
calculus \cp~\cite{wadler2012}, which we will refer to as \rcp.
We have chosen to discuss only a subset in order to keep our later discussion of
our extension to \cp in \cref{sec:main} as simple as possible.
However, we foresee no problems in extending the proofs from \cref{sec:main} to
cover the full version of \cp.
\rcp corresponds to rudimentary linear logic~\cite[RLL]{girard1992}, which is
also known as multiplicative-applicative linear logic.

This section will proceed as follows. First, we will discuss the terms, the
structural congruence, and the types of \rcp.
Then we will discuss the terms and their corresponding types, in small groups,
together with their typing and reduction rules.
Finally, we will prove preservation, progress and termination for \rcp.

\subsection{Terms and types}\label{sec:cp:terms-and-types}
The term language for \rcp is a variant of the
\textpi-calculus~\cite{milner1992b}.
Its terms are defined by the following grammar:
\begin{definition}[Terms]\label{def:cp-terms}
  \[\!
    \begin{aligned}
      \tm{P}, \tm{Q}, \tm{R}
           :=& \; \tm{\cpLink{x}{y}}       &&\text{link}
      \\ \mid& \; \tm{\cpCut{x}{P}{Q}}     &&\text{parallel composition}
      \\ \mid& \; \tm{\cpSend{x}{y}{P}{Q}} &&\text{``output''}
      \\ \mid& \; \tm{\cpRecv{x}{y}{P}}    &&\text{``input''}
      \\ \mid& \; \tm{\cpHalt{x}}          &&\text{halt}
      \\ \mid& \; \tm{\cpWait{x}{P}}       &&\text{wait}
      \\ \mid& \; \tm{\cpInl{x}{P}}        &&\text{select left choice}
      \\ \mid& \; \tm{\cpInr{x}{P}}        &&\text{select right choice}
      \\ \mid& \; \tm{\cpCase{x}{P}{Q}}    &&\text{offer binary choice}
      \\ \mid& \; \tm{\cpAbsurd{x}}        &&\text{offer nullary choice}
    \end{aligned}
  \]  
\end{definition}
%%% Local Variables:
%%% TeX-master: "main"
%%% End:

The construct \tm{\cpLink{x}{y}} links two
channels~\cite{sangiorgi1996,boreale1998}, forwarding messages received on
\tm{x} to \tm{y} and vice versa. 
The construct \tm{\cpCut{x}{P}{Q}} creates a new channel \tm{x}, and composes
two processes, \tm{P} and \tm{Q}, which communicate on \tm{x}, in parallel.
Therefore, in \tm{\cpCut{x}{P}{Q}} the name \tm{x} is bound in both \tm{P} and
\tm{Q}. 
In \tm{\cpRecv{x}{y}{P}} and \tm{\cpSend{x}{y}{P}{Q}}, round brackets are used
for input, square brackets for output. 
We use bound output~\cite{sangiorgi1996}.
This means that unlike in the \textpi-calculus, both input and output bind a new
name. 
In \tm{\cpRecv{x}{y}{P}} the new name \tm{y} is bound in \tm{P}. 
In \tm{\cpSend{x}{y}{P}{Q}}, the new name \tm{y} is only bound in \tm{P}, while
\tm{x} is only bound in \tm{Q}.

Terms in \rcp are identified up to structural congruence, which states that
parallel compositions \tm{\cpCut{x}{P}{Q}} are associative and commutative.
It is defined as follows:
\begin{definition}[Structural congruence]\label{def:cp-equiv}
  We define the structural congruence $\equiv$ as reflexivity, transitivity, and
  congruence over terms, plus the following two axioms:
  \[
    \begin{array}{llll}
      \cpEquivCutComm
      & \tm{\cpCut{x}{P}{Q}}
      & \equiv \;
      & \tm{\cpCut{x}{Q}{P}}
      \\
      \cpEquivCutAss1
      & \tm{\cpCut{x}{P}{\cpCut{y}{Q}{R}}}
      & \equiv \;
      & \tm{\cpCut{y}{\cpCut{x}{P}{Q}}{R}}
        \quad \text{if} \; \notFreeIn{x}{R} \; \text{and} \; \notFreeIn{y}{P}
    \end{array}
  \]
\end{definition}
%%% Local Variables:
%%% TeX-master: "main"
%%% End:

We do not add an axiom for \cpEquivCutAss2, as it follows from
\cref{def:cp-equiv}, see~\cref{thm:cp-cut-assoc2}.
Note that throughout this thesis, we will leave uses of the transitivity and
congruence rules implicit.
\begin{lemmaB}[\cpEquivCutAssNoParen2]\label{thm:cp-cut-assoc2}
  If $\tm{x}\not\in\tm{R}$ and $\tm{y}\not\in\tm{P}$, then 
  \(
    \tm{\cpCut{y}{\cpCut{x}{P}{Q}}{R}} \equiv
    \tm{\cpCut{x}{P}{\cpCut{y}{Q}{R}}}
  \).
\end{lemmaB}
  \begin{proof}
    \begin{align*}
      \tm{\cpCut{y}{\cpCut{x}{P}{Q}}{R}} &\equiv \qquad \text{by \cpEquivCutComm} \\
      \tm{\cpCut{y}{\cpCut{x}{Q}{P}}{R}} &\equiv \qquad \text{by \cpEquivCutComm} \\
      \tm{\cpCut{y}{R}{\cpCut{x}{Q}{P}}} &\equiv \qquad \text{by \cpEquivCutAss1} \\
      \tm{\cpCut{x}{\cpCut{y}{R}{Q}}{P}} &\equiv \qquad \text{by \cpEquivCutComm} \\
      \tm{\cpCut{x}{P}{\cpCut{y}{R}{Q}}} &\equiv \qquad \text{by \cpEquivCutComm} \\
      \tm{\cpCut{x}{P}{\cpCut{y}{Q}{R}}}
    \end{align*}
    The side conditions for \cpEquivCutAss1 are given.
  \end{proof}
%%% Local Variables:
%%% TeX-master: "main"
%%% End:

Furthermore, structural congruence is a symmetric relation.
\begin{theorem}[Symmetry]\label{thm:cp-symmetry}
  If $\tm{P} \equiv \tm{Q}$, then $\tm{Q} \equiv \tm{P}$.
\end{theorem}
  \begin{proof}
    By induction on the structure of the equivalence proof.
    The only interesting case is \cpEquivCutAss1, which follows from
    \cref{thm:cp-cut-assoc2}.
  \end{proof}
%%% Local Variables:
%%% TeX-master: "main"
%%% End:

%
Channels in \rcp are typed using a session type system which corresponds to RLL,
the multiplicative, additive fragment of linear logic.
These are defined using the following grammar:
\begin{definition}[Types]\label{def:cp-types}
  \[\!
    \begin{aligned}
      \ty{A}, \ty{B}, \ty{C}
           :=& \; \ty{A \tens B} &&\text{pair of independent processes}
      \\ \mid& \; \ty{A \parr B} &&\text{pair of interdependent processes}
      \\ \mid& \; \ty{\one}      &&\text{unit for} \; {\tens}
      \\ \mid& \; \ty{\bot}      &&\text{unit for} \; {\parr}
      \\ \mid& \; \ty{A \plus B} &&\text{internal choice}
      \\ \mid& \; \ty{A \with B} &&\text{external choice}
      \\ \mid& \; \ty{\nil}      &&\text{unit for} \; {\plus}
      \\ \mid& \; \ty{\top}      &&\text{unit for} \; {\with}
    \end{aligned}
  \]  
\end{definition}
%%% Local Variables:
%%% TeX-master: "main"
%%% End:

Duality plays a crucial role in both linear logic and session types.
In \cp, the two endpoints of a channel are assigned dual types.
This ensures that, for instance, whenever a process \emph{sends} across a
channel, the process on the other end of that channel is waiting to
\emph{receive}.
Each type \ty{A} has a dual, written \ty{A^\bot}, which is defined as follows:
\begin{definition}[Duality]\label{def:cp-negation}
  \[\!
    \begin{array}{lclclcl}
              \ty{(A \tens B)^\bot} &=& \ty{A^\bot \parr B^\bot}
      &\quad& \ty{\one^\bot}        &=& \ty{\bot}
      \\      \ty{(A \parr B)^\bot} &=& \ty{A^\bot \tens B^\bot}
      &\quad& \ty{\bot^\bot}        &=& \ty{\one}
      \\      \ty{(A \plus B)^\bot} &=& \ty{A^\bot \with B^\bot}
      &\quad& \ty{\nil^\bot}        &=& \ty{\top}
      \\      \ty{(A \with B)^\bot} &=& \ty{A^\bot \plus B^\bot}
      &\quad& \ty{\top^\bot}        &=& \ty{\nil}
    \end{array}
  \]
\end{definition}
%%% Local Variables:
%%% TeX-master: "main"
%%% End:

Duality is an involutive function.
\begin{lemma}[Involutive]\label{thm:cp-negation-involutive}
  We have $\ty{A^{\bot\bot}} = \ty{A}$.
\end{lemma}
  \begin{proof}
    By induction on the structure of the type $\ty{A}$.
  \end{proof}
%%% Local Variables:
%%% TeX-master: "main"
%%% End:

%
Environments associate channels with types. They are defined as follows:
\begin{definition}[Environments]\label{def:cp-environments}
  We define environments as follows:
  \[
    \ty{\Gamma}, \ty{\Delta}, \ty{\Theta}
    ::= \tmty{x_1}{A_1}\dots\tmty{x_n}{A_n}
  \] 
  Names in environments must be unique, and environments \ty{\Gamma} and
  \ty{\Delta} can only be combined as $\ty{\Gamma}, \ty{\Delta}$ if
  $\text{fv}(\ty{\Gamma}) \cap \text{fv}(\ty{\Delta}) = \varnothing$. 
\end{definition}
%%% Local Variables:
%%% TeX-master: "main"
%%% End:

Typing judgements associative processes with their collection of channels, and
enforce the communication protocols specified by the types of those channels.
They are defined as follows:
\input{def-cp-typing-judgement}
\begin{figure*}[b]
  \begin{center}
    \cpInfAx
    \cpInfCut
  \end{center}
  \begin{center}
    \cpInfTens
    \cpInfParr
  \end{center}
  \begin{center}
    \cpInfOne
    \cpInfBot
  \end{center}
  \begin{center}
    \cpInfPlus1
    \cpInfPlus2
  \end{center}
  \begin{center}
    \cpInfWith
  \end{center}
  \begin{center}
    \cpInfNil
    \cpInfTop
  \end{center}
  \caption{Typing judgement for the multiplicative applicative subset of \rcp.}
  \label{fig:cp-typing-judgement}
\end{figure*}
%%% Local Variables:
%%% TeX-master: "main"
%%% End:

Reductions relate processes with their reduced forms.
They are defined as follows:
\begin{definition}[Term reduction]\label{def:cp-reduction}
  \[
    \begin{array}{llll}
      \cpRedAxCut1
      & \tm{\cpCut{x}{\cpLink{w}{x}}{P}}
      & \Longrightarrow \;
      & \tm{\cpSub{w}{x}{P}} 
      \\
      \cpRedAxCut2
      & \tm{\cpCut{x}{\cpLink{x}{w}}{P}}
      & \Longrightarrow \;
      & \tm{\cpSub{w}{x}{P}} 
      \\
      \\
      \cpRedBetaTensParr
      & \tm{\cpCut{x}{\cpSend{x}{y}{P}{Q}}{\cpRecv{x}{z}{R}}}
      & \Longrightarrow \;
      & \tm{\cpCut{y}{P}{\cpCut{x}{Q}{\cpSub{y}{z}{R}}}}
      \\
      \cpRedBetaOneBot
      & \tm{\cpCut{x}{\cpHalt{x}}{\cpWait{x}{P}}}
      & \Longrightarrow \;
      & \tm{P}
      \\
      \cpRedBetaPlusWith1
      & \tm{\cpCut{x}{\cpInl{x}{P}}{\cpCase{x}{Q}{R}}}
      & \Longrightarrow \;
      & \tm{\cpCut{x}{P}{Q}}
      \\
      \cpRedBetaPlusWith2
      & \tm{\cpCut{x}{\cpInr{x}{P}}{\cpCase{x}{Q}{R}}}
      & \Longrightarrow \;
      & \tm{\cpCut{x}{P}{R}}
      \\
      \\
      \cpRedKappaTens1
      & \tm{\cpCut{x}{\cpSend{y}{z}{P}{Q}}{R}}
      & \Longrightarrow \;
      & \tm{\cpSend{y}{z}{\cpCut{x}{P}{R}}{Q}} \quad \text{if} \; \notFreeIn{x}{Q}
      \\
      \cpRedKappaTens2
      & \tm{\cpCut{x}{\cpSend{y}{z}{P}{Q}}{R}}
      & \Longrightarrow \;
      & \tm{\cpSend{y}{z}{P}{\cpCut{x}{Q}{R}}} \quad \text{if} \; \notFreeIn{x}{P}
      \\
      \cpRedKappaParr
      & \tm{\cpCut{x}{\cpRecv{y}{z}{P}}{R}}
      & \Longrightarrow \;
      & \tm{\cpRecv{y}{z}{\cpCut{x}{P}{R}}}
      \\
      \cpRedKappaBot
      & \tm{\cpCut{x}{\cpWait{y}{P}}{R}}
      & \Longrightarrow \;
      & \tm{\cpWait{y}{\cpCut{x}{P}{R}}}
      \\
      \cpRedKappaPlus1
      & \tm{\cpCut{x}{\cpInl{y}{P}}{R}}
      & \Longrightarrow \;
      & \tm{\cpInl{y}{\cpCut{x}{P}{R}}}
      \\
      \cpRedKappaPlus2
      &\tm{\cpCut{x}{\cpInr{y}{P}}{R}}
      & \Longrightarrow \;
      & \tm{\cpInr{y}{\cpCut{x}{P}{R}}}
      \\
      \cpRedKappaWith
      & \tm{\cpCut{x}{\cpCase{y}{P}{Q}}{R}}
      & \Longrightarrow \;
      & \tm{\cpCase{y}{\cpCut{x}{P}{R}}{\cpCut{x}{Q}{R}}}
      \\
      \cpRedKappaTop
      & \tm{\cpCut{x}{\cpAbsurd{y}}{R}}
      & \Longrightarrow \;
      & \tm{\cpAbsurd{y}}
    \end{array}
  \]

  \begin{center}
    \begin{prooftree*}
      \AXC{\reducesto{P}{P^\prime}}
      \SYM{\cpRedGammaCut}
      \UIC{\reducesto{\cpCut{x}{P}{Q}}{\cpCut{x}{P^\prime}{Q}}}
    \end{prooftree*}
    \begin{prooftree*}
      \AXC{$\tm{P}\equiv\tm{Q}$}
      \AXC{\reducesto{Q}{Q^\prime}}
      \AXC{$\tm{Q^\prime}\equiv\tm{P^\prime}$}
      \SYM{\cpRedGammaEquiv}
      \TIC{\reducesto{P}{P^\prime}}
    \end{prooftree*}
  \end{center}
\end{definition}
%%% Local Variables:
%%% TeX-master: "main"
%%% End:

\begin{figure}
  \subsection*{Structural congruence}
  \(
  \begin{array}{llll}
    & \tm{\cpCut{x}{P}{Q}}
    & \equiv \;
    & \tm{\cpCut{x}{Q}{P}}
    \\
    & \tm{\cpCut{x}{P}{\cpCut{y}{Q}{R}}}
    & \equiv \;
    & \tm{\cpCut{y}{\cpCut{x}{P}{Q}}{R}}
  \end{array}
  \)
  \\[1ex]
  (plus reflexivity, symmetry, and transitivity)
  
  \subsection*{Reduction rules}
  \(
  \begin{array}{llll}
    & \tm{\cpCut{x}{\cpLink{w}{x}}{P}}
    & \Longrightarrow \;
    & \tm{\cpSub{w}{x}{P}} 
    \\
    & \tm{\cpCut{x}{\cpSend{x}{y}{P}{Q}}{\cpRecv{x}{z}{R}}}
    & \Longrightarrow \;
    & \tm{\cpCut{y}{P}{\cpCut{x}{Q}{\cpSub{y}{z}{R}}}}
    \\
    & \tm{\cpCut{x}{\cpHalt{x}}{\cpWait{x}{P}}}
    & \Longrightarrow \;
    & \tm{P}
    \\
    & \tm{\cpCut{x}{\cpInl{x}{P}}{\cpCase{x}{Q}{R}}}
    & \Longrightarrow \;
    & \tm{\cpCut{x}{P}{Q}}
    \\
    & \tm{\cpCut{x}{\cpInr{x}{P}}{\cpCase{x}{Q}{R}}}
    & \Longrightarrow \;
    & \tm{\cpCut{x}{P}{R}}
    \\
    \\
    & \tm{\cpCut{x}{\cpSend{y}{z}{P}{Q}}{R}}
    & \Longrightarrow \;
    & \tm{\cpSend{y}{z}{\cpCut{x}{P}{R}}{Q}} \qquad \text{if} \; x \in P
    \\
    & \tm{\cpCut{x}{\cpSend{y}{z}{P}{Q}}{R}}
    & \Longrightarrow \;
    & \tm{\cpSend{y}{z}{P}{\cpCut{x}{Q}{R}}} \qquad \text{if} \; x \in Q
    \\
    & \tm{\cpCut{x}{\cpRecv{y}{z}{P}}{R}}
    & \Longrightarrow \;
    & \tm{\cpRecv{y}{z}{\cpCut{x}{P}{R}}}
    \\
    & \tm{\cpCut{x}{\cpWait{y}{P}}{R}}
    & \Longrightarrow \;
    & \tm{\cpWait{y}{\cpCut{x}{P}{R}}}
    \\
    & \tm{\cpCut{x}{\cpInl{y}{P}}{R}}
    & \Longrightarrow \;
    & \tm{\cpInl{y}{\cpCut{x}{P}{R}}}
    \\
    &\tm{\cpCut{x}{\cpInr{y}{P}}{R}}
    & \Longrightarrow \;
    & \tm{\cpInr{y}{\cpCut{x}{P}{R}}}
    \\
    & \tm{\cpCut{x}{\cpCase{y}{P}{Q}}{R}}
    & \Longrightarrow \;
    & \tm{\cpCase{y}{\cpCut{x}{P}{R}}{\cpCut{x}{Q}{R}}}
    \\
    & \tm{\cpCut{x}{\cpAbsurd{y}}{R}}
    & \Longrightarrow \;
    & \tm{\cpAbsurd{y}}
  \end{array}
  \)
  
  \begin{prooftree}
    \AXC{\reducesto{P}{P^\prime}}
    \UIC{\reducesto{\cpCut{x}{P}{Q}}{\cpCut{x}{P^\prime}{Q}}}
  \end{prooftree}

  \begin{prooftree}
    \AXC{$\tm{P}\equiv\tm{Q}$}
    \AXC{\reducesto{Q}{Q^\prime}}
    \AXC{$\tm{Q^\prime}\equiv\tm{P^\prime}$}
    \TIC{\reducesto{P}{P^\prime}}
  \end{prooftree}

  \caption{Reduction rules for \cp}\label{fig:cp-term-reduction}
\end{figure}
%%% Local Variables:
%%% TeX-master: "main"
%%% End:

We will discuss the interpretations of each connective, together with their
typing and reduction rules, in
\cref{sec:cp-dependence,sec:cp-choice,sec:cp-duality}.

\subsection{Multiplicatives and in- and interdependence}
\label{sec:cp-dependence}
The multiplicatives ($\ty{\tens}, \ty{\parr}$) deal with independence and
interdependence:
\begin{itemize}
\item
  A channel of type \ty{A \tens B} represents a pair of channels, which
  communicate with two \emph{independent} processes---that is to say, two
  processes who share no channels.
  A process acting on a channel of type \ty{A \tens B} will send one endpoint of
  a fresh channel, and then split into a pair of independent processes.
  One of these processes will be responsible for an interaction of type \ty{A}
  over the fresh channel, while the other process continues to interact as
  \ty{B}.
\item
  A channel of type \ty{A \parr B} represents a pair of interdependent channels,
  which are used within a single process. 
  A process acting on a channel of type \ty{A \parr B} will receive a channel to
  act on, and communicate on its channels in whatever order it pleases.
  This means that the usage of one channel can depend on that of
  another---e.g.\ the interaction of type \ty{B} could depend on the result of
  the interaction of type \ty{A}, or vise versa, and if \ty{A} and \ty{B} are
  complex types, their interactions could likewise interweave in complex ways. 
\end{itemize}
While the rules for \ty{\tens} and \ty{\parr} introduce input and output
operations, these are inessential---the essential distinction lies two in the
fact that (\tens) composes two independent processes, and therefore \emph{must}
split the environment between them, whereas (\parr) uses a single process, which
then can---and must---use all the channels in the environment.
\begin{center}
  \cpInfTens
  \cpInfParr
\end{center}
The \textbeta-reduction rule for terms introduced by $(\tens)$ and $(\parr)$
implements the behaviour outlined above: 
\[
  \tm{\cpCut{x}{\cpSend{x}{y}{P}{Q}}{\cpRecv{x}{z}{R}}}
  \Longrightarrow
  \tm{\cpCut{y}{P}{\cpCut{x}{Q}{\cpSub{y}{z}{R}}}}
\]
%
The rules for the multiplicative units ($\ty{\one}, \ty{\bot}$) follow the same
pattern, except for the nullary instead of the binary case:
\begin{itemize}
\item
  A term constructed by $(\one)$ must composes \emph{zero} independent
  processes, and thus must halt. Furthermore, it must be able to split its
  environment between zero processes, and thus its environment must be empty.
\item
  A term constructed by $(\bot)$, on the other hand, uses a single process,
  which is not further restricted. 
\end{itemize}
Note that the rules for $\ty{\one}$ and $\ty{\bot}$ introduce a nullary send and
receive operation, such as those found in the polyadic \textpi-calculus.
\begin{center}
  \cpInfOne
  \cpInfBot
\end{center}
The \textbeta-reduction rule for terms introduced by $(\one)$ and $(\bot)$
implements the behaviour outlined above:
\[
  \tm{\cpCut{x}{\cpHalt{x}}{\cpWait{x}{P}}}
  \Longrightarrow
  \tm{P}
\]

\subsection{Additives and choice}\label{sec:cp-choice}
The additives ($\ty{\plus}, \ty{\with}$) deal with choice:
\begin{itemize}
\item
  A process acting on a channel of type \ty{A \plus B} either sends the value
  \tm{inl} to select an interaction of type \ty{A} or the value \tm{inr} to
  select one of type \ty{B}.
\item
  A process acting on a channel of type \ty{A \with B} receives such a value,
  and then offers an interaction of either type \ty{A} or \ty{B},
  correspondingly.
\end{itemize}
Note that, in essence, the additive operations implement the sending and
receiving of a single bit of information, \tm{inl} or \tm{inr}, and branching
based on the value of that bit.
The rule for constructing a process which sends \tm{inr}, $(\plus_2)$, has been
omitted, but can be found in~\cref{fig:cp-typing-judgement}.
\begin{center}
  \cpInfPlus1
  \cpInfWith
\end{center}
The \textbeta-reduction rules for terms introduced by $(\plus_1)$, $(\plus_2)$
and $(\with)$ implements the behaviour outlined above.
\[
  \tm{\cpCut{x}{\cpInl{x}{P}}{\cpCase{x}{Q}{R}}}
  \Longrightarrow
  \tm{\cpCut{x}{P}{Q}}
\]
\[
  \tm{\cpCut{x}{\cpInr{x}{P}}{\cpCase{x}{Q}{R}}}
  \Longrightarrow
  \tm{\cpCut{x}{P}{R}}
\]
%
The rules for the additive units ($\ty{\nil}, \ty{\top}$) follow the same
pattern, except for a nullary choice:
\begin{itemize}
\item
  There is \emph{no} rule for \ty{\nil}, as a process acting on a channel of
  that type would have to select one of \emph{zero} options, which is clearly
  impossible.
\item
  A process acting on a channel of type \ty{\top} will wait to receive a choice
  of out \emph{zero} options. Since this will clearly never arrive, we have two
  options: either we block, waiting forever, or we simply crash.
\end{itemize}
It may seem odd at first to include a type for the process which cannot possibly
exist, and for the process which waits forever, but these make sensible units
for choice.
When offered a choice of type \ty{A \plus \nil}, one can either choose to
interact as \ty{A}, or choose to commit to doing the impossible.
Similarly, when offering a choice of type \ty{A \with \top}, one can safely
implement the right branch with a process which waits forever, as no sound
process will ever be able to select that branch anyway.
\begin{center}
  \cpInfNil
  \cpInfTop
\end{center}
As there is no way to construct a process of type \ty{\nil}, there is no
reduction rule for the additive units.
However, it is worth discussing the commutative conversion for \ty{\top} here,
separately.
In \cp, commutative conversions push communications, \tm{\cpCut{x}{P}{Q}},
deeper into terms, under unrelated actions.
However, looking at the commutative conversion for \ty{\top} from different
perspective, we see that it states that if one of the two communication partners 
is blocked forever, we can consider their composition to be blocked forever
as well:
\[
  \tm{\cpCut{x}{\cpAbsurd{y}}{R}}
  \Longrightarrow
  \tm{\cpAbsurd{y}}
\]


\subsection{Structural rules and duality}\label{sec:cp-duality}

\subsection{Properties of \rcp}\label{sec:cp-properties}


%%% Local Variables:
%%% TeX-master: "main"
%%% End:

\chapter{\cp as a type system for the \textpi-calculus}\label{sec:cppi}
\cp has a tight correspondence with classical linear logic.
This has many advantages.
It is deadlock free and terminating, and as seen in \cref{sec:cp-properties},
the proofs of its meta-theoretical properties are brief and aesthetically
pleasing.

The price paid for this is a somewhat weaker correspondence with the
\textpi-calculus.
It would be useful to be able to think of \cp as a type system for the
\textpi-calculus.
However, as it stands there are many differences between these systems.
Most prominent of these are the commutative conversions. These reduction rules
are taken directly from the proof normalisation procedure of classical linear
logic, and do not correspond to any reductions in the \textpi-calculus.

\citenat{lindley2015semantics} observed that, using a different reduction
strategy, which more closely resembles that of the \textpi-calculus, we can
ensure that the commutative conversions are always applied \emph{last}.
That is to say, they define two separate reduction relations:
$\longrightarrow_{C}$ for \cpRedAxCut1, \cpRedAxCut2 and \textbeta-reductions,
and $\longrightarrow_{CC}$ for commutative conversions, and show that any
sequence of reductions has the following form:
\[
  P \longrightarrow_{C} \dots \longrightarrow_{C} Q \longrightarrow_{CC} \dots \longrightarrow_{CC} R
\]
In this dissertation, we use a reduction strategy which follows
\citenat{lindley2015semantics}, but opts to drop the suffix of commutative
conversions.
As a consequence of this, we can drop the commutative conversions from our
reduction system, which therefore more tightly corresponds to that of the
\textpi-calculus. 
A consequence of dropping the commutative conversions is that the reduction
system as a whole becomes deterministic.
The price we pay for this is a weaker correspondence to classical linear logic.
This shows in our notion of canonical form, which is weaker, and in our proof of
progress, which is much more involved.

For clarity's sake: whenever we refer to \cp or \rcp, for the remainder of this
dissertation, we refer to the variant \emph{without} commutative conversions,
i.e.\ which uses the following definition of term reduction.
\begin{definition}[Term reduction]\label{def:cp-term-reduction-2}
  A reduction $\reducesto{P}{P'}$ denotes that the process \tm{P} can reduce to
  the process \tm{P'} in a single step. Reductions can only be constructed using
  the rules in~\cref{fig:cp-term-reduction-1}.
  % 
  The relation $\Longrightarrow^\star$ is the reflexive, transitive closure of
  $\Longrightarrow$.
\end{definition}
%%% Local Variables:
%%% TeX-master: "main"
%%% End:


This chapter will proceed as follows.
In \cref{sec:cp-canonical-forms}, we will describe what it means for a term to
be in canonical form.
In \cref{sec:cp-evaluation-contexts}, we will define evaluation contexts.
In \cref{sec:cp-progress}, we will give a new proof of progress, which follows
\citenat{lindley2015semantics}.

\section{Canonical forms}\label{sec:cp-canonical-forms}
The reduction strategy described by \citeauthor{lindley2015semantics} applies
\cpRedAxCut1, \cpRedAxCut2, and \textbeta-reductions until the process blocks on
one or more external communications, and then applies the commuting conversions
to bubble one of those external communications to the front of the term.
This allows them to define canonical forms as any term which is not a cut.
For us, terms in canonical form will be those terms which are blocked on an
external communication, before any commutative conversions are applied.
In this section, we will describe the form of such terms.

We have informally used the phrase ``act on'' in previous sections. It is time
to formally define what it means when we say a process \emph{acts on} some
channel.
\begin{definition}[Action]\label{def:cp-action}
  A process $\tm{P}$ \emph{acts on} a channel $\tm{x}$ if it is of the form
  $\tm{\cpSend{x}{y}{P'}{Q'}}$, $\tm{\cpRecv{x}{y}{P'}}$, $\tm{\cpHalt{x}}$,
  $\tm{\cpWait{x}{P'}}$, $\tm{\cpInl{x}{P'}}$, $\tm{\cpInr{x}{P'}}$,
  $\tm{\cpCase{x}{P'}{Q'}}$ or $\tm{\cpAbsurd{x}}$. 
\end{definition}
%%% Local Variables:
%%% TeX-master: "main"
%%% End:

Furthermore, we will need the notion of an \emph{evaluation prefix}.
Intuitively, evaluation prefixes are multi-holed contexts consisting solely of
cuts. We will use evaluation prefixes in order to have a view of every
\emph{action} in a process at once.
\begin{definition}[Evaluation prefixes]\label{def:cp-evaluation-prefixes}
  We define evaluation prefixes as:
  \begin{align*}
    \tm{G}, \tm{H} := \tm{\Box} \mid \tm{\cpCut{x}{G}{H}}
  \end{align*}
\end{definition}
\begin{definition}[Plugging]\label{def:cp-evaluation-prefix-plugging}
  We define plugging for an evaluation prefix with $n$ holes as:
  \[
    \begin{array}{ll}
      \tm{\cpPlug{\Box}{R}} & := \; \tm{R} \\
      \tm{\cpPlug{\cpCut{x}{G}{H}}{R_1 \dots R_m, R_{m+1} \dots R_{n}}}
                            & := \; \tm{\cpCut{x}{\cpPlug{G}{R_1 \dots R_m}}{\cpPlug{H}{R_{m+1} \dots R_n}}}
    \end{array}
  \]
  Note that in the second case, \tm{G} is an evaluation prefix with $m$ holes,
  and \tm{H} is an evaluation prefix with $(n-m)$ holes.
\end{definition}
%%% Local Variables:
%%% TeX-master: "main"
%%% End:

Intuitively, we can say that every term of the form
\tm{\cpPlug{G}{P_1 \dots P_n}} is equivalent to some term of the form
\tm{\cpCut{x_1}{P_1}{\cpCut{x_2}{P_2}{\dots \cpCut{x_n}{P_{n-1}}{P_n} \dots}}} 
where $\tm{x_1} \dots \tm{x_{n-1}}$ are the channels bound in \tm{G}.
In fact, a similar equivalence was used by \citeauthor{lindley2015semantics}
\cite{lindley2015semantics} in their semantics for \cp. 
\begin{definition}[Maximum evaluation prefix]\label{def:cp-maximum-evaluation-prefix}
  We say that \tm{G} is the evaluation prefix of \tm{P} when there exist terms
  $\tm{P_1} \dots \tm{P_n}$ such that $\tm{P} = \tm{\cpPlug{G}{P_1 \dots P_n}}$.
  We say that \tm{G} is the maximum evaluation prefix if each \tm{P_i} is an
  action. 
\end{definition}
\begin{lemma}\label{thm:cp-maximum-evaluation-prefix}
  Every term \tm{P} has a maximum evaluation prefix.
\end{lemma}
\begin{proof}
  By induction on the structure of \tm{P}.
\end{proof}
%%% Local Variables:
%%% TeX-master: "main"
%%% End:

We can now define what it means for a term to be in canonical form. Intuitively,
a process is in canonical form either when there is no top-level cut, or when it
is blocked on an external communication. We state this formally as follows:
\begin{definition}[Canonical forms]\label{def:cp-canonical-forms}
  A process \tm{P} is in canonical form if it is in one of the following forms:
  \begin{multicols}{3}
    \begin{itemize}[noitemsep,topsep=0pt,parsep=0pt,partopsep=0pt]
    \item \tm{\cpLink{x}{y}}
    \item \tm{\cpSend{x}{y}{P'}{Q'}}
    \item \tm{\cpRecv{x}{y}{P'}}
    \item \tm{\cpHalt{x}}
    \item \tm{\cpWait{x}{P'}}
    \item \tm{\cpInl{x}{P'}}
    \item \tm{\cpInr{x}{P'}}
    \item \tm{\cpCase{x}{P'}{Q'}}
    \item \tm{\cpAbsurd{x}}
    \end{itemize}
  \end{multicols}
\end{definition}
%%% Local Variables:
%%% TeX-master: "main"
%%% End:

This definition is adequate, as it matches our intuition. We will see this in
our proof for \cref{thm:cp-progress-3}.

\section{Evaluation contexts}\label{sec:cp-evaluation-contexts}
Intuitively, evaluation contexts are one-holed term contexts under which
reduction can take place. For \rcp, these consist solely of cuts.
\begin{definition}[Evaluation contexts]\label{def:cp-evaluation-contexts}
  We define evaluation contexts as:
  \begin{align*}
    \tm{G}, \tm{H} := \tm{\Box}
    \mid \tm{\cpCut{x}{G}{P}}
    \mid \tm{\cpCut{x}{P}{G}}
  \end{align*}
\end{definition}
\begin{definition}[Plugging]\label{def:cp-evaluation-context-plugging}
  We define plugging for evaluation contexts as:
  \begin{gather*}
    \begin{array}{ll}
      \tm{\cpPlug{\Box}{R}}            
      & := \; \tm{R}
      \\
      \tm{\cpPlug{\cpCut{x}{G}{P}}{R}}
      & := \; \tm{\cpCut{x}{\cpPlug{G}{R}}{P}}
      \\
      \tm{\cpPlug{\cpCut{x}{P}{G}}{R}}
      & := \; \tm{\cpCut{x}{P}{\cpPlug{G}{R}}}
    \end{array}
  \end{gather*}
\end{definition}
%%% Local Variables:
%%% TeX-master: "main"
%%% End:

We can prove that we can push any cut downwards under an evaluation context, as
long as the channel it binds does not occur in the context itself.
\begin{lemmaB}\label{thm:cp-display-cut-1}
  If $\seq[{ \tm{\cpCut{x}{\cpPlug{G}{P}}{Q}} }]{ \Gamma }$ and
  $\notFreeIn{x}{G}$, then $\tm{\cpCut{x}{\cpPlug{G}{P}}{Q}} \equiv
  \tm{\cpPlug{G}{\cpCut{x}{P}{Q}}}$. 
\end{lemmaB}
  \begin{proof}
    By induction on the structure of the evaluation context \tm{G}.
    \begin{itemize}
    \item
      Case $\tm{\Box}$. By reflexivity.
    \item
      Case $\tm{\cpCut{y}{G}{R}}$.
      \[\!
        \begin{array}{ll}
          \tm{\cpCut{x}{\cpCut{y}{\cpPlug{G}{P}}{R}}{Q}} & \equiv \quad \text{by \cpEquivCutComm}\\
          \tm{\cpCut{x}{\cpCut{y}{R}{\cpPlug{G}{P}}}{Q}} & \equiv \quad \text{by \cpEquivCutAss2}\\
          \tm{\cpCut{y}{R}{\cpCut{x}{\cpPlug{G}{P}}{Q}}} & \equiv \quad \text{by \cpEquivCutComm}\\
          \tm{\cpCut{y}{\cpCut{x}{\cpPlug{G}{P}}{Q}}{R}} & \equiv \quad \text{by the induction hypothesis and \cref{thm:cp-preservation-equiv}}\\ 
          \tm{\cpCut{y}{\cpPlug{G}{\cpCut{x}{P}{Q}}}{R}} &
        \end{array}
      \]
    \item
      Case $\tm{\cpCut{y}{R}{G}}$.
      \[\!
        \begin{array}{ll}
          \tm{\cpCut{x}{\cpCut{y}{R}{\cpPlug{G}{P}}}{Q}} & \equiv \quad \text{by \cpEquivCutAss2}\\
          \tm{\cpCut{y}{R}{\cpCut{x}{\cpPlug{G}{P}}{Q}}} & \equiv \quad \text{by the induction hypothesis and \cref{thm:cp-preservation-equiv}}\\
          \tm{\cpCut{y}{R}{\cpPlug{G}{\cpCut{x}{P}{Q}}}}
        \end{array}
      \]
    \end{itemize}
    In each case, the side conditions for \cpEquivCutAss2, $\notFreeIn{x}{R}$ and
    $\notFreeIn{y}{Q}$, can be inferred from $\notFreeIn{x}{G}$ and the fact that
    $\tm{\cpCut{x}{\cpPlug{G}{P}}{Q}}$ is well-typed.
  \end{proof}
%%% Local Variables:
%%% TeX-master: "main"
%%% End:
And vice versa. However, we will not use the following lemma in this
dissertation, and leave its proof as an exercise to the reader.
\begin{lemmaB}\label{thm:cp-display-cut-2}
  If $\seq[{ \cpPlug{E}{\cpCut{x}{P}{Q}} }]{ \Gamma }$ and $\notFreeIn{x}{E}$,
  then 
  $\tm{\cpPlug{E}{\cpCut{x}{P}{Q}}} \equiv \tm{\cpCut{x}{\cpPlug{E}{P}}{Q}}$. 
\end{lemmaB}
%%% Local Variables:
%%% TeX-master: "main"
%%% End:

\section{Progress}\label{sec:cp-progress}
Progress is the fact that every term is either in some canonical form, or can be
reduced further.
%
There are two important lemmas which we will need in order to prove progress.
These relate evaluation prefixes to evaluation contexts.
%
Specifically, if a process under an evaluation prefix is a link, we can rewrite
the entire process in such a way as to reveal the cut which introduced one of
the channels acted upon by that link.
\begin{lemmaB}\label{thm:cp-progress-link}
  If $\seq[{ \cpPlug{G}{P_1 \dots P_n} }]{ \Gamma }$, and some \tm{P_i} is a
  link \tm{\cpLink{x}{y}}, then either \tm{x} and \tm{y} are not bound by
  \tm{G}, or there exist \tm{H}, \tm{H'} and \tm{Q} such that
  $\tm{\cpPlug{G}{P_1 \dots P_n}} \equiv
  \tm{\cpPlug{H}{\cpCut{x}{\cpPlug{H'}{\cpLink{x}{y}}}{Q}}}$. 
\end{lemmaB}
\begin{proof}
  By induction on the structure of \tm{G}.
  \begin{itemize}
  \item
    Case \tm{\Box}. Clearly \tm{x} and \tm{y} are not bound.
  \item
    Case \tm{\cpCut{z}{\cpPlug{G'}{P_1 \dots P_i \dots P_m}}{\cpPlug{G''}{P_{m+1} \dots P_n}}}.\\
    Case \tm{\cpCut{z}{\cpPlug{G'}{P_1 \dots P_m}}{\cpPlug{G''}{P_{m+1} \dots P_i \dots P_n}}}.\\
    We apply the induction hypothesis. There are two cases:
    \begin{itemize}
    \item
      If \tm{x} and \tm{y} were not bound, they remain unbound.
    \item
      If \tm{x} or \tm{y} is bound deeper in \tm{G}, we prepend one of
      \tm{\cpCut{z}{\Box}{\cpPlug{G''}{P_{m+1} \dots P_n}}},
      \tm{\cpCut{z}{\cpPlug{G'}{P_1 \dots P_m}}{\Box}},
      \tm{\ncPool{\Box}{\cpPlug{G''}{P_{m+1} \dots P_n}}}, or
      \tm{\ncPool{\cpPlug{G'}{P_1 \dots P_m}}{\Box}} to \tm{H}.
      \\
      The desired equality follows by congruence.
    \end{itemize}
  \item
    Case \tm{\cpCut{x}{\cpPlug{G'}{P_1 \dots P_i \dots P_m}}{\cpPlug{G''}{P_{m+1} \dots P_n}}}.\\
    Case \tm{\cpCut{y}{\cpPlug{G'}{P_1 \dots P_i \dots P_m}}{\cpPlug{G''}{P_{m+1} \dots P_n}}}.\\
    Case \tm{\cpCut{x}{\cpPlug{G'}{P_1 \dots P_m}}{\cpPlug{G''}{P_{m+1} \dots P_i \dots P_n}}}.\\
    Case \tm{\cpCut{y}{\cpPlug{G'}{P_1 \dots P_m}}{\cpPlug{G''}{P_{m+1} \dots P_i \dots P_n}}}.
    \\
    Let $\tm{H} := \tm{\Box}$ and 
    \(\arraycolsep=0pt\begin{array}[t]{ll}
      \text{let}
      & \; \tm{H_i} := \tm{\cpPlug{G'}{P_1 \dots P_{i-1}, \Box, P_{i+1}, \dots P_m}}
      \\
      \text{or}
      & \; \tm{H_i} := \tm{\cpPlug{G''}{P_{m+1} \dots P_{i-1}, \Box, P_{i+1}, \dots P_n}}
    \end{array}\)
    \\[1ex]
    By reflexivity and \cpEquivCutComm.
  \end{itemize}
\end{proof}
%%% Local Variables:
%%% TeX-master: "main"
%%% End:
And if two processes under an evaluation prefix act on the same channel, then we
can rewrite the entire process in such a way as to reveal the cut which
introduced that channel. 
\begin{lemmaB}\label{thm:cp-progress-beta}
  If $\seq[{ \cpPlug{G}{P_1 \dots P_n} }]{ \Gamma }$, and some \tm{P_i} and
  \tm{P_j}, on different sides of at least one cut, act on the same bound
  channel \tm{x}, then there exist \tm{E}, \tm{E_i} and \tm{E_j} such that 
  \(
  \tm{\cpPlug{G}{P_1 \dots P_n}} =
  \tm{\cpPlug{E}{\cpCut{x}{\cpPlug{E_i}{P_i}}{\cpPlug{E_j}{P_j}}}}
  \).
\end{lemmaB}
\begin{proof}
  By induction on the structure of \tm{G}.
  \begin{itemize}
  \item
    Case \tm{\cpCut{x}{\cpPlug{G'}{P_1\dots P_i\dots P_m}}{\cpPlug{G''}{P_{m+1}\dots P_j\dots P_n}}}. 
    \\
    \(\arraycolsep=0pt\begin{array}[t]{lll}
      \text{Let}&\ \tm{E}  &\ :=\ \tm{\Box}, \\
                &\ \tm{E_i}&\ :=\ \tm{\cpPlug{G'}{P_1\dots P_{i-1},\Box,P_{i+1}\dots P_m}}, \\
                &\ \tm{E_j}&\ :=\ \tm{\cpPlug{G''}{P_{m+1}\dots P_{j-1},\Box,P_{j+1}\dots P_n}}.
    \end{array}\)
    \\[1ex]
    By reflexivity.
  \item
    Case \tm{\cpCut{x}{\cpPlug{G'}{P_1\dots P_j\dots P_m}}{\cpPlug{G''}{P_{m+1}\dots P_i\dots P_n}}}.
    \\
    As above.
  \item
    Case \tm{\cpCut{y}{\cpPlug{G'}{P_1\dots P_i\dots P_j\dots P_m}}{\cpPlug{G''}{P_{m+1}\dots P_n}}}. \\
    Case \tm{\cpCut{y}{\cpPlug{G'}{P_1\dots P_m}}{\cpPlug{G''}{P_{m+1}\dots P_i\dots P_j\dots P_n}}}.
    \\
    We obtain \tm{E}, \tm{E_1} and \tm{E_2} from the induction hypothesis and
    \cref{thm:cp-preservation-equiv}, and then prepend either
    \tm{\cpCut{y}{\Box}{\cpPlug{G''}{P_{m+1}\dots P_n}}} or
    \tm{\cpCut{y}{\cpPlug{G'}{P_1\dots P_m}}{\Box}} to \tm{E}.
    The desired equality follows by congruence.
  \end{itemize}
  The case for \tm{\Box} is excluded because $n > 1$.
  The cases in which \tm{P_i} and \tm{P_j} are on the \emph{same} side of the
  cut, but the cut binds \tm{x}, and the cases in which \tm{P_i} and \tm{P_j}
  are on different sides of the cut, but the cut binds some other channel
  \tm{y}, are excluded by the type system.
\end{proof}
%%% Local Variables:
%%% TeX-master: "main"
%%% End:


Finally, we are ready to prove progress.
Note that in proofs throughout this dissertation, we will leave uses of
\cpRedGammaEquiv and the congruence rules for reductions implicit. 
\begin{theorem}[Progress]\label{thm:cp-progress-3}
  If $\seq[{ P }]{ \Gamma }$, then either $\tm{P}$ is in canonical form, or
  there exists a $\tm{P'}$ such that $\reducesto{P}{P'}$. 
\end{theorem}
\begin{proof}
  By induction on the structure of derivation for $\seq[{ P }]{ \Gamma }$.
  The only interesting case is when the last rule of the derivation is
  \textsc{Cut}---in every other case, the typing rule constructs a term in which 
  is in canonical form. 
  \\
  We consider the maximum evaluation prefix \tm{G} of \tm{P}, such that $\tm{P}
  = \tm{\cpPlug{G}{P_1 \dots P_{n+1}}}$ and each $P_i$ is an action.
  The prefix \tm{G} consists of $n$ cuts, and introduces $n$ channels, but
  composes $n+1$ actions. Therefore, one of the following must be true:
  \begin{itemize}
  \item
    One of the processes is a link \tm{\cpLink{x}{y}} acting on a bound channel.
    We have:
    \begin{gather*}
      \begin{array}{ll}
        \tm{\cpPlug{G}{P_1 \dots \cpLink{x}{y} \dots P_{n+1}}}
        & \equiv \quad \text{by \cref{thm:cp-progress-link}}
        \\
        \tm{\cpPlug{H}{\cpCut{z}{\cpPlug{H'}{\cpLink{x}{y}}}{Q}}}
        & \equiv \quad \text{by \cref{thm:cp-display-cut-1}}
        \\
        \tm{\cpPlug{H}{\cpPlug{H'}{\cpCut{z}{\cpLink{x}{y}}{Q}}}}
      \end{array}
    \end{gather*}
    Where $\tm{z} = \tm{x}$ or $\tm{z} = \tm{y}$.
    We then apply one of \cpRedAxCut1 or \cpRedAxCut2.
  \item
    Two of the processes, \tm{P_i} and \tm{P_j}, act on the same bound channel
    \tm{x}. We have:
    \begin{gather*}
      \begin{array}{ll}
        \tm{\cpPlug{G}{P_1 \dots P_i \dots P_j \dots P_{n+1}}}
        & = \quad \text{by \cref{thm:cp-progress-beta}}
        \\
        \tm{\cpPlug{G}{\cpCut{x}{\cpPlug{H_i}{P_i}}{\cpPlug{H_j}{P_j}}}}
        & \equiv \quad \text{by \cref{thm:cp-display-cut-1}} 
        \\
        \tm{\cpPlug{G}{\cpPlug{H_i}{\cpCut{x}{P_i}{\cpPlug{H_j}{P_j}}}}}
        & \equiv \quad \text{by \cref{thm:cp-display-cut-1} and
          \cref{thm:cp-preservation-equiv}} 
        \\
        \tm{\cpPlug{G}{\cpPlug{H_i}{\cpPlug{H_j}{\cpCut{x}{P_i}{P_j}}}}} 
      \end{array}
    \end{gather*}
    We then apply one of the \textbeta-reduction rules.
  \item
    Otherwise (at least) one of the processes acts on an external channel.
    \\
    No process \tm{P_i} is a link.
    No two processes \tm{P_i} and \tm{P_j} act on the same channel \tm{x}.
    Therefore, \tm{P} is canonical.
  \end{itemize}
\end{proof}
%%% Local Variables:
%%% TeX-master: "main"
%%% End:

The proof of progress described in this section is novel, though it takes
inspiration from the reduction system for \cp described by \citenat{lindley2015semantics}. 
The proof of itself is somewhat involved. The reason for
this is that we want our reduction strategy to match that of the
\textpi-calculus as closely as possible.
It is also for this reason that our proof of progress involves some
non-determinism. For instance, in the second case of the proof, we do not
specify how to select the two processes \tm{P_i} and \tm{P_j} if there are
multiple options available.

\section{Example}
Let's have a look at the differences between the reduction strategy we have
defined in this chapter, and the reduction strategy which \citenat{wadler2012}
defines. Let's imagine the following scenario:
\begin{quote}
  Alice, John, and Mary went on a lovely trip together.
  However, John and Mary can be a bit scatterbrained sometimes, and it just so
  happens that they both forgot to bring their wallets.
  They both owe Alice some money, which they're now trying to pay back at the
  same time.
\end{quote}
We can model this interaction in \cp as the following term, assuming \alice,
\john, and \mary are three processes representing Alice, John and Mary, and
\bankjohn and \bankmary are two processes representing John and Mary's
respective banks.
Note that, in an abuse of notation, we will adopt an ``anti-Barendregt''
convention for our examples.
This means that if two channel names will be identified after a reduction, we
will preemptively give them the same name, for instance, \tm{z} and \tm{w} in
the example below. 
\begin{align*}
  \tm{\cpCut{x}{\cpCut{y}{
  \cpRecv{x}{z}{\cpRecv{y}{w}{\alice}}}{
  \cpSend{y}{w}{\bankjohn}{\john}}}{
  \cpSend{x}{z}{\bankmary}{\mary}}}
\end{align*}
Note that in the above interaction, no \textbeta-reduction rule applies
\emph{immediately}. This means that using the reduction strategy described by
\citenat{wadler2012}, we will apply a commutative conversion. 
\begin{align*}
  \begin{array}{lll}
    \tm{\cpCut{x}{\cpCut{y}{
    \cpRecv{x}{z}{\cpRecv{y}{w}{\alice}}}{
    \cpSend{y}{w}{\bankjohn}{\john}}}{
    \cpSend{x}{z}{\bankmary}{\mary}}}
    & \Longrightarrow & \text{by \cpRedKappaParr}
    \\
    \tm{\cpCut{x}{\cpRecv{x}{z}{\cpCut{y}{
    \cpRecv{y}{w}{\alice}}{
    \cpSend{y}{w}{\bankjohn}{\john}}}}{
    \cpSend{x}{z}{\bankmary}{\mary}}}
    & \Longrightarrow & \text{by \cpRedBetaTensParr}
    \\
    \tm{\cpCut{z}{\cpCut{x}{\cpCut{y}{
    \cpRecv{y}{w}{\alice}}{
    \cpSend{y}{w}{\bankjohn}{\john}}}{
    \mary}}{
    \bankmary}}
    & \Longrightarrow & \text{by \cpRedBetaTensParr}
    \\
    \tm{\cpCut{z}{\cpCut{x}{\cpCut{w}{\cpCut{y}{
    \alice}{
    \john}}{
    \bankjohn}}{
    \mary}}{
    \bankmary}}
  \end{array}
\end{align*}
On the other hand, using the reduction strategy described in this chapter, we will
rewrite using structural congruence.
\begin{align*}
  \begin{array}{lll}
    \tm{\cpCut{x}{\cpCut{y}{
    \cpRecv{x}{z}{\cpRecv{y}{w}{\alice}}}{
    \cpSend{y}{w}{\bankjohn}{\john}}}{
    \cpSend{x}{z}{\bankmary}{\mary}}}
    & \equiv          & \text{by \cref{thm:cp-display-cut-1}}
    \\
    \tm{\cpCut{y}{\cpCut{x}{
    \cpRecv{x}{z}{\cpRecv{y}{w}{\alice}}}{
    \cpSend{x}{z}{\bankmary}{\mary}}}{
    \cpSend{y}{w}{\bankjohn}{\john}}}
    & \Longrightarrow & \text{by \cpRedBetaTensParr}
    \\
    \tm{\cpCut{y}{\cpCut{z}{\cpCut{x}{
    \cpRecv{y}{w}{\alice}}{
    \mary}}{
    \bankmary}}{
    \cpSend{y}{w}{\bankjohn}{\john}}} 
    & \equiv          & \text{by \cref{thm:cp-display-cut-1}}
    \\
    \tm{\cpCut{z}{\cpCut{x}{\cpCut{y}{
    \cpRecv{y}{w}{\alice}}{
    \cpSend{y}{w}{\bankjohn}{\john}}}{
    \mary}}{
    \bankmary}}
    & \Longrightarrow & \text{by \cpRedBetaTensParr}
    \\
    \tm{\cpCut{z}{\cpCut{x}{\cpCut{w}{\cpCut{y}{
    \alice}{
    \john}}{
    \bankjohn}}{
    \mary}}{
    \bankmary}}
  \end{array}
\end{align*}
Note that the second reduction sequence is a \emph{valid} reduction sequence in
the either reduction system---it is simply not the sequence \emph{chosen} by the
reduction strategy described by \citenat{wadler2012}.

%%% Local Variables:
%%% TeX-master: "main"
%%% End:

\chapter{Non-deterministic Classical Processes}\label{sec:main}
%% Main contribution
%% * Terms and types
\begin{definition}[Terms]\label{def:nc-terms}
  \[\!
    \begin{aligned}
      \tm{P}, \tm{Q}, \tm{R}
           :=& \; \dots
      \\ \mid& \; \tm{\ncCnt{x}{y}{P}} &&\text{create client}
      \\ \mid& \; \tm{\ncSrv{x}{y}{P}} &&\text{create server interaction}
      \\ \mid& \; \tm{\ncPool{P}{Q}}   &&\text{parallel composition of clients}
    \end{aligned}
  \]
\end{definition}
%%% Local Variables:
%%% TeX-master: "main"
%%% End:

\begin{definition}[Types]\label{def:nc-types}
  We extend \cref{def:cp-types} with the following types:
  \[\!
    \begin{aligned}
      \ty{A}, \ty{B}, \ty{C}
           :=& \; \dots
      \\ \mid& \; \ty{\take[n]{A}} &&\text{pool of} \; n \; \text{clients}
      \\ \mid& \; \ty{\give[n]{A}} &&n \; \text{server interactions}
    \end{aligned}
  \]  
\end{definition}
%%% Local Variables:
%%% TeX-master: "main"
%%% End:

\begin{definition}[Negation]\label{def:nc-negation}
  We extend \cref{def:cp-negation} with the following cases:
  \[\!
    \begin{array}{lclclcl}
              \ty{(\take[n]{A})^\bot} &=& \ty{\give[n]{A^\bot}}
      &\quad& \ty{(\give[n]{A})^\bot} &=& \ty{\take[n]{A^\bot}}
    \end{array}
  \]
\end{definition}
%%% Local Variables:
%%% TeX-master: "main"
%%% End:

\begin{lemma}[Involutive]\label{thm:nc-negation-involutive}
  We have $\ty{A^{\bot\bot}} = \ty{A}$.
\end{lemma}
\begin{proof}
  By induction on the structure of the type $\ty{A}$.
\end{proof}
%%% Local Variables:
%%% TeX-master: "main"
%%% End:

%% * Reduction rules
\begin{definition}[Structural congruence]\label{def:nc-equiv}
  We extend \cref{def:cp-equiv} with the following equivalences:
  \[
    \begin{array}{llll}
%     \ncEquivPoolId
%     & \tm{\ncPool{P}{\ncHalt}}
%     & \equiv \;
%     & \tm{P}
%     \\
      \ncEquivPoolComm
      & \tm{\ncPool{P}{Q}}
      & \equiv \;
      & \tm{\ncPool{Q}{P}}
      \\
      \ncEquivPoolAss1
      & \tm{\ncPool{P}{\ncPool{Q}{R}}}
      & \equiv \;
      & \tm{\ncPool{\ncPool{P}{Q}}{R}}
    \end{array}
  \]
\end{definition} 
%%% Local Variables:
%%% TeX-master: "main"
%%% End:

\begin{definition}[Term reduction]\label{def:nc-reduction}
  We extend \cref{def:cp-reduction} with the following reductions:
  \[
    \begin{array}{llll}
      \ncRedBetaStar{1}
      & \tm{\cpCut{x}{\ncCnt{x}{y}{P}}{\ncSrv{x}{z}{R}}}
      & \Longrightarrow \;
      & \tm{\cpCut{y}{P}{\cpSub{y}{z}{R}}}
      \\
      \ncRedBetaStar{n+1}
      & \tm{\cpCut{x}{\ncPool{\ncCnt{x}{y}{P}}{Q}}{\ncSrv{x}{z}{R}}}
      & \Longrightarrow \;
      & \tm{\cpCut{x}{Q}{\cpCut{y}{P}{\cpSub{y}{z}{R}}}}
      \\
      \\
      \ncRedKappaTake
      & \tm{\cpCut{x}{\ncCnt{y}{z}{P}}{R}}
      & \Longrightarrow \;
      & \tm{\ncCnt{y}{z}{\cpCut{x}{P}{R}}}
      \\
      \ncRedKappaGive
      & \tm{\cpCut{x}{\ncSrv{y}{z}{P}}{R}}
      & \Longrightarrow \;
      & \tm{\ncSrv{y}{z}{\cpCut{x}{P}{R}}}
    \end{array}
  \]
  \begin{prooftree}
    \AXC{\reducesto{P}{P'}}
    \SYM{\ncRedGammaPool}
    \UIC{\reducesto{\ncPool{P}{Q}}{\ncPool{P'}{Q}}}
  \end{prooftree}
\end{definition}
%%% Local Variables:
%%% TeX-master: "main"
%%% End:

\begin{lemma}[\ncEquivPoolAssNoParen2]\label{thm:nc-pool-assoc2}
  We have
  \[
    \tm{\ncPool{\ncPool{P}{Q}}{R}} \equiv
    \tm{\ncPool{P}{\ncPool{Q}{R}}}
  \]
\end{lemma}
\begin{proof}
  \begin{align*}
    \tm{\ncPool{\ncPool{P}{Q}}{R}} &\equiv \qquad \text{by \ncEquivPoolComm} \\
    \tm{\ncPool{\ncPool{Q}{P}}{R}} &\equiv \qquad \text{by \ncEquivPoolComm} \\
    \tm{\ncPool{R}{\ncPool{Q}{P}}} &\equiv \qquad \text{by \ncEquivPoolAss1} \\
    \tm{\ncPool{\ncPool{R}{Q}}{P}} &\equiv \qquad \text{by \ncEquivPoolComm} \\
    \tm{\ncPool{P}{\ncPool{R}{Q}}} &\equiv \qquad \text{by \ncEquivPoolComm} \\
    \tm{\ncPool{P}{\ncPool{Q}{R}}}
  \end{align*}
\end{proof}
%%% Local Variables:
%%% TeX-master: "main"
%%% End:

\begin{theorem}[Symmetry]\label{thm:nc-symmetry}
  If $\tm{P} \equiv \tm{Q}$, then $\tm{Q} \equiv \tm{P}$.
\end{theorem}
\begin{proof}
  By induction on the structure of the equivalence proof.
\end{proof}
%%% Local Variables:
%%% TeX-master: "main"
%%% End:

%% * Typing judgement
\input{def-nc-environments}
\begin{figure*}[b]
  \centering
  \begin{center}
    \ncInfTake1
    \ncInfGive1
  \end{center}
  \begin{center}
    \ncInfPool
  \end{center}
  \begin{center}
    \ncInfCont
  \end{center}
  \caption{Typing judgement for \nodcap extending that of Figure~\ref{fig:cp-typing-judgement}}
  \label{fig:nc-typing-judgement}
\end{figure*}
%%% Local Variables:
%%% TeX-master: "main"
%%% End:

%% * Cut elimination
\begin{theorem}[Preservation for $\equiv$]\label{thm:nc-preservation-equiv}
  If $\seq[{ P }]{ \Gamma }$ and $\tm{P} \equiv \tm{Q}$,
  then $\seq[{ Q }]{ \Gamma }$.
\end{theorem}
\begin{proof}
  By induction on the structure of the equivalence. The cases for reflexivity,
  transitivity and congruence are trivial. The cases for \cpEquivCutComm and
  \cpEquivCutAss1 are given in \cref{fig:cp-preservation-equiv}.
  The cases for \ncEquivPoolComm and \ncEquivPoolAss1 are given in
  \cref{fig:nc-preservation-equiv}.
\end{proof}
%%% Local Variables:
%%% TeX-master: "main"
%%% End:

\begin{figure*}[ht]
  \centering
  \begin{tabular}{ll}
    \ncEquivPoolComm
    &
      \begin{prooftree*}
        \AXC{$\seq[{ P }]{ \Gamma, \tmty{x}{\take[m]{A}} }$}
        \AXC{$\seq[{ Q }]{ \Delta, \tmty{x}{\take[n]{A}} }$}
        \NOM{Pool}
        \BIC{$\seq[{ \ncPool{P}{Q} }]{ \Gamma, \Delta, \take[m+n]{A} }$}
      \end{prooftree*}
    \\[30pt]
    $\equiv$
    &
      \begin{prooftree*}
        \AXC{$\seq[{ Q }]{ \Delta, \tmty{x}{\take[n]{A}} }$}
        \AXC{$\seq[{ P }]{ \Gamma, \tmty{x}{\take[m]{A}} }$}
        \NOM{Pool}
        \BIC{$\seq[{ \ncPool{Q}{P} }]{ \Gamma, \Delta, \tmty{x}{\take[m+n]{A}} }$}
      \end{prooftree*}
    \\[40pt]
    \ncEquivPoolAss1
    &
      \begin{prooftree*}
        \AXC{$\seq[{ P }]{ \Gamma, \tmty{x}{\take[l]{A}} }$}
        \AXC{$\seq[{ Q }]{ \Delta, \tmty{x}{\take[m]{A}} }$}
        \AXC{$\seq[{ R }]{ \Theta, \tmty{x}{\take[n]{A}} }$}
        \NOM{Pool}
        \BIC{$\seq[{ \ncPool{Q}{R} }]{ \Delta, \Theta, \tmty{x}{\take[m+n]{A}} }$}
        \NOM{Pool}
        \BIC{$\seq[{ \ncPool{P}{\ncPool{Q}{R}} }]{ \Gamma, \Delta, \Theta, \tmty{x}{\take[l+m+n]{A}} }$}
      \end{prooftree*}
    \\[30pt]
    $\equiv$
    &
      \begin{prooftree*}
        \AXC{$\seq[{ P }]{ \Gamma, \tmty{x}{\take[l]{A}} }$}
        \AXC{$\seq[{ Q }]{ \Delta, \tmty{x}{\take[m]{A}} }$}
        \NOM{Pool}
        \BIC{$\seq[{ \ncPool{P}{Q} }]{ \Gamma, \Delta, \tmty{x}{\take[l+m]{A}} }$}
        \AXC{$\seq[{ R }]{ \Theta, \tmty{x}{\take[n]{A}} }$}
        \NOM{Pool}
        \BIC{$\seq[{ \ncPool{\ncPool{P}{Q}}{R} }]{ \Gamma, \Delta, \Theta, \tmty{x}{\take[l+m+n]{A}} }$}
      \end{prooftree*}
    \\[20pt]
    or
    \\[30pt]
    \ncEquivPoolAss1
    &
      \begin{prooftree*}
        \AXC{$\seq[{ P }]{ \Gamma, \tmty{x}{\take[k]{A}} }$}
        \AXC{$\seq[{ Q }]{ \Delta, \tmty{x}{\take[l]{A}}, \tmty{y}{\take[m]{B}} }$}
        \AXC{$\seq[{ R }]{ \Theta, \tmty{x}{\take[n]{B}} }$}
        \NOM{Pool}
        \BIC{$\seq[{ \ncPool{Q}{R} }]{ \Delta, \Theta, \tmty{x}{\take[l]{A}}, \tmty{y}{\take[m+n]{B}} }$}
        \NOM{Pool}
        \BIC{$\seq[{ \ncPool{P}{\ncPool{Q}{R}} }]{ \Gamma, \Delta, \Theta, \tmty{x}{\take[k+l]{A}}, \tmty{y}{\take[m+n]{B}} }$}
      \end{prooftree*}
    \\[30pt]
    $\equiv$
    &
      \begin{prooftree*}
        \AXC{$\seq[{ P }]{ \Gamma, \tmty{x}{\take[k]{A}} }$}
        \AXC{$\seq[{ Q }]{ \Delta, \tmty{x}{\take[l]{A}}, \tmty{y}{\take[m]{B}} }$}
        \NOM{Pool}
        \BIC{$\seq[{ \ncPool{Q}{R} }]{ \Gamma, \Delta, \tmty{x}{\take[k+l]{A}}, \tmty{y}{\take[m]{B}} }$}
        \AXC{$\seq[{ R }]{ \Theta, \tmty{x}{\take[o]{B}} }$}
        \NOM{Pool}
        \BIC{$\seq[{ \ncPool{P}{\ncPool{Q}{R}} }]{ \Gamma, \Delta, \Theta, \tmty{x}{\take[k+l]{A}}, \tmty{y}{\take[m+n]{B}} }$}
      \end{prooftree*}
  \end{tabular}
  \caption{Type preservation for the structural congruence of \nc}
  \label{fig:nc-preservation-equiv}
\end{figure*}
%%% Local Variables:
%%% TeX-master: "main"
%%% End:

\begin{theorem}[Preservation]\label{thm:nc-preservation}
  If \reducesto{P}{Q} and $\seq[{ P }]{ \Gamma }$, then $\seq[{ Q }]{ \Gamma }$.
\end{theorem}
\begin{proof}
  By induction on the structure of the reduction. See
  \cref{fig:cp-preservation-1} for \cpRedAxCut{i} and the \textbeta-reduction
  rules from \cp, and \cref{fig:cp-preservation-2a,fig:cp-preservation-2b} for the
  commutative conversions from \cp.
  See \cref{fig:nc-preservation-1,fig:nc-preservation-2a,fig:nc-preservation-2b}
  for the \textbeta-reduction rules and commutative conversions from \nodcap.
  The cases for \cpRedGammaCut and \ncRedGammaPool are trivial by call to the
  induction hypothesis, and the case for \cpRedGammaEquiv is trivial by call to
  the induction hypothesis and \cref{thm:nc-preservation-equiv}.
\end{proof}
%%% Local Variables:
%%% TeX-master: "main"
%%% End:

\begin{figure*}[ht]
  \makebox[\textwidth][c]{
    \begin{tabular}{ll}
      \ncRedBetaStar1
      &
        \begin{prooftree*}
          \AXC{$\seq[{ P }]{ \Gamma, \tmty{y}{A} }$}
          \SYM{\take[1]{}}
          \UIC{$\seq[{ \ncCnt{x}{y}{P} }]{ \Gamma, \tmty{x}{\take[1]{A}} }$}
          \AXC{$\seq[{ Q }]{ \Delta, \tmty{z}{A^\bot} }$}
          \SYM{\give[1]{}}
          \UIC{$\seq[{ \ncSrv{x}{z}{Q} }]{ \Delta, \tmty{x}{\give[1]{A}} }$}
          \NOM{Cut}
          \BIC{$\seq[{ \cpCut{x}{\ncCnt{x}{y}{P}}{\ncSrv{x}{z}{Q}} }]{ \Gamma, \Delta }$}
        \end{prooftree*}
      \\[30pt]
      $\Longrightarrow$
      &
        \begin{prooftree*}
          \AXC{$\seq[{ P }]{ \Gamma, \tmty{y}{A} }$}
          \AXC{$\seq[{ Q }]{ \Delta, \tmty{z}{A^\bot} }$}
          \NOM{Cut}
          \BIC{$\seq[{ \cpCut{y}{P}{\cpSub{y}{z}{Q}} }]{ \Gamma, \Delta }$}
        \end{prooftree*}
      \\[40pt]
      \ncRedBetaStar{n+1}
      &
        \begin{prooftree*}
          \AXC{$\seq[{ P }]{ \Gamma, \tmty{y}{A} }$}
          \SYM{\take[1]{}}
          \UIC{$\seq[{ \ncCnt{x}{y}{P} }]{ \Gamma, \tmty{x}{\take[1]{A}} }$}
          \AXC{$\seq[{ Q }]{ \Delta, \tmty{x}{\take[n]{A}} }$}
          \NOM{Pool}
          \BIC{$\seq[{ \ncPool{\ncCnt{x}{y}{P}}{Q} }]{ \Gamma, \Delta, \tmty{x}{\take[n+1]{A}} }$}
          \AXC{$\seq[{ R }]{ \Theta, \tmty{z}{A^\bot}, \tmty{x}{\give[n]{A}} }$}
          \SYM{\give[1]{}}
          \UIC{$\seq[{ \ncSrv{x'}{z}{Q} }]{ \Theta, \tmty{x'}{\give[1]{A}}, \tmty{x}{\give[n]{A}} }$}
          \NOM{Cont}
          \UIC{$\seq[{ \ncSrv{x}{z}{Q} }]{ \Theta, \tmty{x}{\give[n+1]{A}} }$}
          \NOM{Cut}
          \BIC{$\seq[{ \cpCut{x}{\ncPool{\ncCnt{x}{y}{P}}{Q}}{\ncSrv{x}{z}{Q}} }]{ \Gamma, \Delta, \Theta }$}
        \end{prooftree*}
      \\[40pt]
      $\Longrightarrow$
      &
        \begin{prooftree*}
          \AXC{$\seq[{ Q }]{ \Delta, \tmty{x}{\take[n]{A}} }$}
          \AXC{$\seq[{ P }]{ \Gamma, \tmty{y}{A} }$}
          \AXC{$\seq[{ R }]{ \Theta, \tmty{z}{A^\bot}, \tmty{x}{\give[n]{A}} }$}
          \NOM{Cut}
          \BIC{$\seq[{ \cpCut{y}{P}{\cpSub{y}{z}{R}} }]{ \Gamma, \Theta, \tmty{x}{\give[n]{A}} }$}
          \NOM{Cut}
          \BIC{$\seq[{ \cpCut{x}{Q}{\cpCut{y}{P}{\cpSub{y}{z}{R}}} }]{ \Gamma, \Delta, \Theta }$}
        \end{prooftree*}
      \\[40pt]
      \ncRedKappaTake
      &
        \begin{prooftree*}
          \AXC{$\seq[{ P }]{ \Gamma, \tmty{x}{A}, \tmty{z}{B} }$}
          \SYM{\take[1]{}}
          \UIC{$\seq[{ \ncCnt{y}{z}{P} }]{ \Gamma, \tmty{x}{A}, \tmty{y}{\take[1]{B}} }$}
          \AXC{$\seq[{ R }]{ \Delta, \tmty{x}{A^\bot} }$}
          \NOM{Cut}
          \BIC{$\seq[{ \cpCut{x}{\ncCnt{y}{z}{P}}{R} }]{ \Gamma, \Delta, \tmty{y}{\take[1]{B}} }$}
        \end{prooftree*}
      \\[30pt]
      $\Longrightarrow$
      &
        \begin{prooftree*}
          \AXC{$\seq[{ P }]{ \Gamma, \tmty{x}{A}, \tmty{z}{B} }$}
          \AXC{$\seq[{ R }]{ \Delta, \tmty{x}{A^\bot} }$}
          \NOM{Cut}
          \BIC{$\seq[{ \cpCut{x}{P}{R} }]{ \Gamma, \Delta, \tmty{z}{B} }$}
          \SYM{\take[1]{}}
          \UIC{$\seq[{ \ncCnt{y}{z}{\cpCut{x}{P}{R}} }]{ \Gamma, \Delta, \tmty{z}{\take[1]{B}} }$}
        \end{prooftree*}
      \\[30pt]
      \ncRedKappaGive
      &
        (as above)
    \end{tabular}
  }   

  \caption{Type preservation for the \textbeta-reduction rules and commutative conversions of \nodcap}
  \label{fig:nc-preservation-1}
\end{figure*}
%%% Local Variables:
%%% TeX-master: "main"
%%% End: 
\begin{figure*}[ht]
  \begin{tabular}{ll}
    \ncRedKappaTake
    &
      \begin{prooftree*}
        \AXC{$\seq[{ P }]{ \Gamma, \tmty{x}{A}, \tmty{z}{B} }$}
        \SYM{\take[1]{}}
        \UIC{$\seq[{ \ncCnt{y}{z}{P} }]{ \Gamma, \tmty{x}{A}, \tmty{y}{\take[1]{B}} }$}
        \AXC{$\seq[{ R }]{ \Delta, \tmty{x}{A^\bot} }$}
        \NOM{Cut}
        \BIC{$\seq[{ \cpCut{x}{\ncCnt{y}{z}{P}}{R} }]{ \Gamma, \Delta, \tmty{y}{\take[1]{B}} }$}
      \end{prooftree*}
    \\[30pt]
    $\Longrightarrow$
    &
      \begin{prooftree*}
        \AXC{$\seq[{ P }]{ \Gamma, \tmty{x}{A}, \tmty{z}{B} }$}
        \AXC{$\seq[{ R }]{ \Delta, \tmty{x}{A^\bot} }$}
        \NOM{Cut}
        \BIC{$\seq[{ \cpCut{x}{P}{R} }]{ \Gamma, \Delta, \tmty{z}{B} }$}
        \SYM{\take[1]{}}
        \UIC{$\seq[{ \ncCnt{y}{z}{\cpCut{x}{P}{R}} }]{ \Gamma, \Delta, \tmty{z}{\take[1]{B}} }$}
      \end{prooftree*}
    \\[30pt]
    \ncRedKappaGive
    &
      (as above)
    \ncRedKappaPoolTens1
    &
      \begin{prooftree*}
        \AXC{$\seq[{ P }]{\Gamma, \tmty{x}{\take[m]{A}}, \tmty{z}{B}}$}
        \AXC{$\seq[{ Q }]{\Delta, \tmty{y}{C}}$}
        \SYM{\tens}
        \BIC{$\seq[{ \cpSend{y}{z}{P}{Q} }]%
          {\Gamma, \Delta, \tmty{x}{A}, \tmty{y}{B \tens C}}$}
        \AXC{$\seq[{ R }]{\Theta, \tmty{x}{\take[n]{A}}}$}
        \NOM{Pool}
        \BIC{$\seq[{ \ncPool{\cpSend{y}{z}{P}{Q}}{R} }]%
          {\Gamma, \Delta, \Theta, \tmty{x}{\take[m+n]{A}}, \tmty{y}{B \tens C}}$}
      \end{prooftree*}
    \\[30pt]
    $\Longrightarrow$
    &
      \begin{prooftree*}
        \AXC{$\seq[{ P }]{\Gamma, \tmty{x}{\take[m]{A}}, \tmty{z}{B}}$}
        \AXC{$\seq[{ R }]{\Theta, \tmty{x}{\take[n]{A}}}$}
        \NOM{Pool}
        \BIC{$\seq[{ \ncPool{P}{R} }]%
          {\Gamma, \Theta, \tmty{x}{\take[m+n]{A}}, \tmty{z}{B}}$}
        \AXC{$\seq[{ Q }]{\Delta, \tmty{y}{C}}$}
        \SYM{\tens}
        \BIC{$\seq[{ \cpSend{y}{z}{\ncPool{P}{R}}{R} }]%
          {\Gamma, \Delta, \Theta, \tmty{x}{\take[m+n]{A}}, \tmty{y}{B \tens C}}$}
      \end{prooftree*}
    \\[30pt]
    \ncRedKappaPoolTens2
    &
      (as above)
    \\[20pt]
    \ncRedKappaPoolParr
    &
      \begin{prooftree*}
        \AXC{$\seq[{ P }]{\Gamma, \tmty{x}{\take[m]{A}}, \tmty{z}{B}, \tmty{y}{C}}$}
        \SYM{\parr}
        \UIC{$\seq[{ \cpRecv{y}{z}{P} }]%
          {\Gamma, \tmty{x}{\take[m]{A}}, \tmty{z}{B \parr C}}$}
        \AXC{$\seq[{ R }]{\Theta, \tmty{x}{\take[n]{A}}}$}
        \NOM{Pool}
        \BIC{$\seq[{ \ncPool{\cpRecv{y}{z}{P}}{R} }]%
          {\Gamma, \Theta, \tmty{x}{\take[m+n]{A}}, \tmty{y}{B \parr C}}$}
      \end{prooftree*}
    \\[30pt]
    $\Longrightarrow$
    &
      \begin{prooftree*}
        \AXC{$\seq[{ P }]{\Gamma, \tmty{x}{\take[m]{A}}, \tmty{z}{B}, \tmty{y}{C}}$}
        \AXC{$\seq[{ R }]{\Theta, \tmty{x}{\take[n]{A}}}$}
        \NOM{Pool}
        \BIC{$\seq[{ \ncPool{P}{R} }]%
          {\Gamma, \Theta, \tmty{x}{\take[m+n]{A}}, \tmty{z}{B}, \tmty{y}{C}}$}
        \SYM{\parr}
        \UIC{$\seq[{ \cpRecv{y}{z}{\ncPool{P}{R}} }]%
          {\Gamma, \Theta, \tmty{x}{\take[m+n]{A}}, \tmty{z}{B \parr C}}$}
      \end{prooftree*}
    \\[40pt]
    \ncRedKappaPoolBot
    &
      \begin{prooftree*}
        \AXC{$\seq[{ P }]{\Gamma, \tmty{x}{\take[m]{A}}}$}
        \SYM{\bot}
        \UIC{$\seq[{ \cpWait{y}{P} }]{\Gamma, \tmty{x}{\take[m]{A}, \tmty{y}{\bot}}}$}
        \AXC{$\seq[{ R }]{\Theta, \tmty{x}{\take[n]{A}}}$} 
        \NOM{Pool}
        \BIC{$\seq[{ \ncPool{\cpWait{y}{P}}{R} }]{\Gamma, \Theta, \tmty{x}{\take[m+n]{A}}, \tmty{y}{\bot}}$}
      \end{prooftree*}
    \\[30pt]
    $\Longrightarrow$
    &
      \begin{prooftree*}
        \AXC{$\seq[{ P }]{\Gamma, \tmty{x}{\take[m]{A}}}$}
        \AXC{$\seq[{ R }]{\Theta, \tmty{x}{\take[n]{A}}}$} 
        \NOM{Pool}
        \BIC{$\seq[{ \ncPool{P}{R} }]{\Gamma, \Theta, \tmty{x}{\take[m+n]{A}}}$}
        \SYM{\bot}
        \UIC{$\seq[{ \cpWait{y}{\ncPool{P}{R}} }]{\Gamma, \Theta, \tmty{x}{\take[m+n]{A}}, \tmty{y}{\bot}}$}
      \end{prooftree*}
  \end{tabular}

  \caption{Type preservation for the commuting conversions with the pooling
    rules of \nodcap}
  \label{fig:nc-preservation-2a}
\end{figure*}
%
\begin{figure*}[ht]
  \makebox[\textwidth][c]{
    \begin{tabular}{ll}
      \ncRedKappaPoolPlus1
      &
        \begin{prooftree*}
          \AXC{$\seq[{ P }]{\Gamma, \tmty{x}{\take[m]{A}}, \tmty{y}{B}}$}
          \SYM{\plus_1}
          \UIC{$\seq[{ \cpInl{y}{P} }]{\Gamma, \tmty{x}{\take[m]{A}}, \tmty{y}{B \plus C}}$}
          \AXC{$\seq[{ R }]{\Theta, \tmty{x}{\take[n]{A}}}$}
          \NOM{Pool}
          \BIC{$\seq[{ \ncPool{\cpInl{x}{P}}{R} }]%
            {\Gamma, \Theta, \tmty{x}{\take[m+n]{A}, \tmty{y}{B \plus C}}}$}
        \end{prooftree*}
      \\[30pt]
      $\Longrightarrow$
      &
        \begin{prooftree*}
          \AXC{$\seq[{ P }]{\Gamma, \tmty{x}{\take[m]{A}}, \tmty{y}{B}}$}
          \AXC{$\seq[{ R }]{\Theta, \tmty{x}{\take[n]{A}}}$}
          \NOM{Pool}
          \BIC{$\seq[{ \ncPool{P}{R} }]{\Gamma, \Theta, \tmty{x}{\take[m+n]{A}}, \tmty{y}{B}}$}
          \SYM{\plus_1}
          \UIC{$\seq[{ \cpInl{x}{\ncPool{P}{R}} }]%
            {\Gamma, \Theta, \tmty{x}{\take[m+n]{A}, \tmty{y}{B \plus C}}}$}
        \end{prooftree*}
      \\[30pt]
      \ncRedKappaPoolPlus2
      &
        (as above)
      \\[20pt]
      \ncRedKappaPoolWith
      &
        \begin{prooftree*}
          \AXC{$\seq[{ P }]{\Gamma, \tmty{x}{\take[m]{A}}, \tmty{y}{B}}$}
          \AXC{$\seq[{ Q }]{\Gamma, \tmty{x}{\take[m]{A}}, \tmty{y}{C}}$}
          \SYM{\with}
          \BIC{$\seq[{ \cpCase{y}{P}{Q} }]{\Gamma, \tmty{x}{\take[m]{A}}, \tmty{y}{B \with C}}$}
          \AXC{$\seq[{ R }]{\Theta, \tmty{x}{\take[n]{A}}}$}
          \NOM{Pool}
          \BIC{$\seq[{ \ncPool{\cpCase{y}{P}{Q}}{R} }]{\Gamma, \Theta, \tmty{x}{\take[m+n]{A}}, \tmty{y}{B \with C}}$}
        \end{prooftree*}
      \\[30pt]
      $\Longrightarrow$
      &
        \begin{prooftree*}
          \AXC{$\seq[{ P }]{\Gamma, \tmty{x}{\take[m]{A}}, \tmty{y}{B}}$}
          \AXC{$\seq[{ R }]{\Theta, \tmty{x}{\take[n]{A}}}$}
          %\NOM{Pool}
          \BIC{$\seq[{ \ncPool{P}{R} }]%
            {\Gamma, \Theta, \tmty{x}{\take[m+n]{A}}, \tmty{y}{B}}$}
          \AXC{$\seq[{ Q }]{\Gamma, \tmty{x}{\take[m]{A}}, \tmty{y}{C}}$}
          \AXC{$\seq[{ R }]{\Theta, \tmty{x}{\take[n]{A}}}$}
          \NOM{Pool}
          \BIC{$\seq[{ \ncPool{Q}{R} }]%
            {\Gamma, \Theta, \tmty{x}{\take[m+n]{A}}, \tmty{y}{C}}$}
          \SYM{\with}
          \BIC{$\seq[{ \cpCase{y}{\ncPool{P}{R}}{\ncPool{Q}{R}} }]%
            {\Gamma, \Theta, \tmty{x}{\take[m+n]{A}}, \tmty{y}{B \with C}}$}
        \end{prooftree*}
      \\[40pt]
      \ncRedKappaPoolTop
      &
        \begin{prooftree*}
          \AXC{}
          \SYM{\top}
          \UIC{$\seq[{ \cpAbsurd{y} }]%
            {\Gamma, \tmty{x}{\take[m]{A}}, \tmty{y}{\top}}$}
          \AXC{$\seq[{ R }]{\Theta, \tmty{x}{\take[n]{A}}}$}
          \NOM{Pool}
          \BIC{$\seq[{ \ncPool{\cpAbsurd{y}}{R} }]%
            {\Gamma, \Theta, \tmty{x}{\take[n]{A}}}$}
        \end{prooftree*}
      \\[30pt]
      $\Longrightarrow$
      &
        \begin{prooftree*}
          \AXC{}
          \SYM{\top}
          \UIC{$\seq[{ \cpAbsurd{y} }]%
            {\Gamma, \Theta, \tmty{x}{\take[m+n]{A}}, \tmty{y}{\top}}$}
        \end{prooftree*}
      \\[40pt]
      \ncRedKappaPoolTake
      &
        \begin{prooftree*}
          \AXC{$\seq[{ P }]{\Gamma, \tmty{x}{\take[m]{A}}, \tmty{z}{B}}$}
          \SYM{\take[1]{}}
          \UIC{$\seq[{ \ncCnt{y}{z}{P} }]%
            {\Gamma, \tmty{x}{\take[m]{A}}, \tmty{y}{\take[1]{B}}}$}
          \AXC{$\seq[{ R }]{\Theta, \tmty{x}{\take[n]{A}}}$}
          \NOM{Pool}
          \BIC{$\seq[{ \ncPool{\ncCnt{y}{z}{P}}{R} }]%
            {\Gamma, \Theta, \tmty{x}{\take[m+n]{A}}, \tmty{y}{\take[1]{B}}}$}
        \end{prooftree*}
      \\[30pt]
      $\Longrightarrow$
      &
        \begin{prooftree*}
          \AXC{$\seq[{ P }]{\Gamma, \tmty{x}{\take[m]{A}}, \tmty{z}{B}}$}
          \AXC{$\seq[{ R }]{\Theta, \tmty{x}{\take[n]{A}}}$}
          \NOM{Pool}
          \BIC{$\seq[{ \ncPool{P}{R} }]%
            {\Gamma, \Theta, \tmty{x}{\take[m+n]{A}}, \tmty{z}{B}}$}
          \SYM{\take[1]{}}
          \UIC{$\seq[{ \ncCnt{y}{z}{\ncPool{P}{R}} }]%
            {\Gamma, \Theta, \tmty{x}{\take[m+n]{A}}, \tmty{y}{B}}$}
        \end{prooftree*}
      \\[30pt]
      \ncRedKappaPoolGive
      &
        (as above)
    \end{tabular}
  }
  
  \caption{Type preservation for the commuting conversions with the pooling
    rules of \nodcap (cont'd)}
  \label{fig:nc-preservation-2b}
\end{figure*}
%%% Local Variables:
%%% TeX-master: "main"
%%% End: 

%% * Progress
\begin{definition}[Action]\label{def:nc-action}
  We extend \cref{def:cp-action} with the following cases:
  \begin{itemize}[noitemsep,topsep=0pt,parsep=0pt,partopsep=0pt]
  \item \tm{\ncCnt{x}{y}{P'}}
  \item \tm{\ncSrv{x}{y}{P'}}
  \end{itemize}
\end{definition}
%%% Local Variables:
%%% TeX-master: "main"
%%% End:

\begin{definition}[Canonical forms]\label{def:nc-canonical-forms}
  We extend \cref{def:cp-canonical-forms} with the following canonical forms:
  \begin{center}
    \begin{prooftree*}
      \AXC{\vphantom{\canonical{P}\canonical{Q}}}
      \UIC{\canonical{\ncCnt{x}{y}{P}}}
    \end{prooftree*}
    \begin{prooftree*}
      \AXC{\vphantom{\canonical{P}\canonical{Q}}}
      \UIC{\canonical{\ncSrv{x}{y}{P}}}
    \end{prooftree*}
    \begin{prooftree*}
      \AXC{\canonical{P}}
      \AXC{\canonical{Q}}
      \BIC{\canonical{\ncPool{P}{Q}}}
    \end{prooftree*}
  \end{center}
\end{definition}
%%% Local Variables:
%%% TeX-master: "main"
%%% End:

\begin{definition}[Evaluation contexts]\label{def:nc-evaluation-contexts}
  We extend \cref{def:cp-evaluation-contexts} with the following constructs:
  \begin{align*}
    \tm{G}, \tm{H} := \dots \mid \tm{\ncPool{G}{P}} \mid \tm{\ncPool{P}{G}}
  \end{align*}
\end{definition}
%%% Local Variables:
%%% TeX-master: "main"
%%% End:

\begin{lemma}\label{thm:nc-display-1}
  If $\seq[{ \tm{\cpCut{x}{\cpPlug{G}{P}}{Q}} }]{ \Gamma }$ and
  $\notFreeIn{x}{G}$, then $\tm{\cpCut{x}{\cpPlug{G}{P}}{Q}} \equiv
  \tm{\cpPlug{G}{\cpCut{x}{P}{Q}}}$.
\end{lemma}
\begin{proof}
  By induction on the structure of the evaluation context \tm{G}.
  \begin{itemize}
  \item
    Case $\tm{\Box}$, $\tm{\cpCut{y}{H}{R}}$, and $\tm{\cpCut{y}{R}{H}}$. See \cref{thm:cp-display}.
  \item
    Case $\tm{\ncPool{G}{R}}$.
    \[\!
      \begin{array}{ll}
        \tm{\cpCut{x}{\ncPool{\cpPlug{G}{P}}{R}}{Q}} & \equiv \quad \text{by} \; \ncEquivPoolComm \\
        \tm{\cpCut{x}{\ncPool{R}{\cpPlug{G}{P}}}{Q}} & \equiv \quad \text{by} \; \ncRedKappaPool1 \\
        \tm{\ncPool{R}{\cpCut{x}{\cpPlug{G}{P}}{Q}}} & \equiv \quad \text{by} \; \ncEquivPoolComm \\
        \tm{\ncPool{\cpCut{x}{\cpPlug{G}{P}}{Q}}{R}} & \equiv \quad \text{by the induction hypothesis}\\
        \tm{\ncPool{\cpPlug{G}{\cpCut{x}{P}{Q}}}{R}} &
      \end{array}
    \]
  \item
    Case $\tm{\ncPool{R}{G}}$.
    \[\!
      \begin{array}{ll}
        \tm{\cpCut{x}{\ncPool{R}{\cpPlug{G}{P}}}{Q}} & \equiv \quad \text{by} \; \ncRedKappaPool1 \\
        \tm{\ncPool{R}{\cpCut{x}{\cpPlug{G}{P}}{Q}}} & \equiv \quad \text{by the induction hypothesis}\\
        \tm{\ncPool{R}{\cpPlug{G}{\cpCut{x}{P}{Q}}}} &
      \end{array}
    \]
  \end{itemize}
  In each case, the side condition for \ncRedKappaPool1, $\notFreeIn{x}{R}$, can
  be inferred from $\notFreeIn{x}{G}$, and the side conditions for the induction
  hypothesis can be inferred from \cref{thm:nc-preservation-equiv} and
  $\notFreeIn{x}{G}$.
\end{proof}
%%% Local Variables:
%%% TeX-master: "main"
%%% End:
\input{thm-nc-display-2}
\input{thm-nc-display-3}
\begin{theorem}[Progress]\label{thm:nc-progress}
  If $\seq[{ P }]{ \Gamma }$, then $\tm{P}$ is in canonical form, or there
  exists a $\tm{P'}$ s.t.\ $\reducesto{P}{P'}$.
\end{theorem}
\begin{proof}
  By induction on the structure of derivation for $\seq[{ P }]{ \Gamma }$.
  There only interesting cases are when the last rule of the derivation is
  \textsc{Cut} or \textsc{Pool}. In every other case, the typing rule constructs
  a term in which is in canonical form. 
  \\
  If the last rule in the derivation is \textsc{Cut} or \textsc{Pool}, we
  consider the prefix of the derivation for $\seq[{ P }]{ \Gamma}$ which
  consists of all top-level cuts and pooling rules. A prefix of $n$ cuts and $m$
  pooling rules introduces $n$ variables, but composes $n+m+1$ actions, at most
  $m+1$ of which are on the same side of all cut rules.
  Therefore, one of the following must be true:
  \begin{itemize}
  \item
    One of these actions was introduced by an application of \textsc{Ax}.
    \\
    We proceed as in \cref{thm:cp-progress}. 
  \item
    Two of these actions, on different sides of a \textsc{Cut}, act on the same
    channel. Let us name these processes \tm{P_i} and \tm{P_j}, and their shared
    channel \tm{y}. We have
    $\tm{P} = \tm{\cpPlug{G}{\cpCut{y}{\cpPlug{H_i}{P_i}}{\cpPlug{H_j}{P_j}}}}$.
    We distinguish the following cases:
    \begin{itemize}
    \item
      We have either
      $\seq[{ \cpPlug{H_i}{P_i}}]{ \Delta, \tmty{y}{\take[n]{A}} }$ or
      $\seq[{ \cpPlug{H_j}{P_j} }]{\Delta, \tmty{y}{\take[n]{A}} }$.
      \\
      We rewrite by \cref{thm:nc-display-3}, then apply one of \ncRedBetaStar{1}
      and \ncRedBetaStar{n+1}. 
    \item
      Otherwise, we can infer $\notFreeIn{y}{H_i}$ and $\notFreeIn{y}{H_j}$.
      \\
      We proceed as in \cref{thm:cp-progress}. 
    \end{itemize}
  \item
    The process is in canonical form \tm{
      \ncPool{\ncCnt{x_1}{y_1}{P_1}}{
        \ncPool{\dots}{\ncCnt{x_n}{y_n}{P_n}}\dots}}. 
  \item 
    Otherwise (at least) one of the actions acts on a free variable.
    \\
    We apply one of the commutative conversions.
  \end{itemize}
\end{proof}
%%% Local Variables:
%%% TeX-master: "main"
%%% End:

\begin{theorem}[Termination]\label{thm:nc-termination}
  If $\seq[{ P }]{ \Gamma }$, then there are no infinite $\Longrightarrow$
  reduction sequences.
\end{theorem}
\begin{proof}
  Every reduction reduces a single cut to zero, one or two cuts.
  However, each of these cuts is \emph{smaller}, in the sense that the type of
  the channel on which the communication takes place is smaller.
  Each reduction either eliminates a connective, or decreases a resource index
  on the type of a shared channel.
  See
  \cref{fig:cp-preservation-1,fig:cp-preservation-2a,fig:cp-preservation-2b,fig:nc-preservation-1,fig:nc-preservation-2a,fig:nc-preservation-2b}.
  Therefore, there cannot be an infinite $\Longrightarrow$ reduction sequence.
\end{proof}
%%% Local Variables:
%%% TeX-master: "main"
%%% End:
%% * Equivalence with non-deterministic local choice
%%% Local Variables:
%%% TeX-master: "main"
%%% End:

\chapter{Discussion}\label{sec:discussion}
%% Discussion
%% - Interaction with recursion
%% - Interaction with affine and relevant sessions
%%% Local Variables:
%%% TeX-master: "main"
%%% End:


%% Bibliography
\printbibliography
\end{document}

%%% Local Variables:
%%% TeX-master: "main"
%%% End:
